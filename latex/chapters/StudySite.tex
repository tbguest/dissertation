\chapter{Advocate Beach}\label{Chapter:AdvocateBeach} 

The objectives of this thesis are addressed using data from two field experiments at Advocate Beach, Nova Scotia. This section begins with an overview of the study site. Though the relevant components of each experiment are described in each chapter, overviews of the 2015 and 2018 field experiments are also provided here for completeness and ease of reference. The relevant analysis methodologies are outlined in the individual chapters. Chapters \ref{Chapter:PorePressure} and \ref{Chapter:CuspDynamics} treat data collected during the 2015 field campaign, and Chapters \ref{Chapter:MSDBeach} and \ref{Chapter:MSDSwash} are associated with the 2018 campaign.

%From the first experiment, one manuscript has been published, and one is in preparation, treating pore pressure response and beach cusp morphodynamics, respectively.


\section{Site Description}

%\begin{figure}[tbp] 
%	%  	\includegraphics[width=0.5\columnwidth]{/home/tristan2/Documents/PhD/Dissertation/figures/chapter3/mean_MGS_crossshore.png}
%	\caption{\label{fig:AdvIntro}}
%\end{figure}

Advocate Beach is a mixed sand-gravel barrier beach positioned near the head of the Bay of Fundy in Nova Scotia, Canada (Fig. \ref{fig:Adv}, p. \pageref{fig:Adv}; Fig. \ref{fig:Advocate_map}, p. \pageref{fig:Advocate_map}). Based on the 8-12 m tidal range, Advocate beach can be classed as megatidal rather than macrotidal \citep{Levoy_etal2000}. The beach separates the headlands of Cape Chignecto to the west and Cape D'Or to the southeast, and is 5 km long with a nearly linear shoreline. The beach face is steep (approximately 1 in 10 slope) and the sediments poorly sorted, ranging from medium sand to cobbles and boulders greater than 20 cm in diameter. From the lower beach face to beneath lowest low water, the sediment composition transitions to cobble and boulder-sized material. From the southwest, the beach is exposed to the full 500 km fetch of the Bay of Fundy and adjacent Gulf of Maine, but is otherwise more fetch-limited. At low tide, the beach is observed to be uniformly planar with crest to low water distance as much as 100 m in spring tides \citep[see][]{Taylor_etal1985, Hay_etal2014}. 

During and after fairweather forcing, an active high tide berm can commonly be observed near the high water line. One or more relict berms may be present landward of the active berm, particularly during neap tides. The berm composition is generally of coarser material than that found in the intertidal zone, consisting of relatively well sorted gravel and cobbles. Following periods of energetic wave forcing, the beach appears free of distinct morphological features, and the beach surface sediments are predominantly sandy. Episodes of pattern formation and surficial sediment sorting often occur at Advocate Beach, commonly in the form of beach cusps on the upper beach face (e.g., Fig. \ref{fig:cusp_bay_overwash}, p. \pageref{fig:cusp_bay_overwash}). The cusps generally appear as well-organised sediment structures near the high water line, often extending tens of metres seaward, forming cross-shore bands of loose gravel and cobbles separated by sandy embayments. The beach experiences changes in forcing conditions over tidal time scales, which is reflected by daily changes in surficial sediment composition and topography. Consequently, cusp episodes at Advocate Beach are generally limited to a single tide.

During storms, peak incident wave periods typically fall within the wind-wave band ($T_p$ = 4-7 seconds), with longer period swell usually limited to the weaker wave forcing conditions between storms. The combination of a steep beach slope and typically short period, wind-generated incident waves result in a highly energetic shore break for offshore significant wave heights of \textit{ca}. 0.5 m and larger \citep{Hay_etal2014}. Under average forcing conditions, the beach is best described as dissipative rather than reflective, which is uncommon for beaches of similar composition and profile type \citep[e.g.,][]{Wright_etal1979}. For example, the mean reflectivity computed for a two week period in October 2018 was $R^2=0.14$, where $R^2$ is an approximation of the ratio of seaward to shoreward propagating wave energy, computed as a Miche number \citep{Miche1951}, following the method of \citet{Elgar_etal1994}. Driftwood and other mobile flotsam are often present in the shorebreak, providing additional hazards to \textit{in situ} instrumentation over and above the impacts from coarse-grained sediment projectiles propelled by the high water velocities in the shorebreak and surf.

For additional information on Advocate Beach, including its history and more detailed descriptions of hydrodynamic processes, see \citet{Taylor_etal1985}, \citet{Wilson_etal2014}, and \citet{Hay_etal2014}.


%[more history. add geol context?]

% from conclusions:

%The large tide range, steep slope, and broad, generally well-mixed grain size distribution make Advocate Beach advantageous as a study site for reasons both practical and scientific: in practice, the intertidal beach can be accessed during each low tide to deploy, retrieve, or maintain \textit{in situ} instrumentation. The selective transport properties of heterogeneous sediments in response to variations in forcing lead to spatially and temporally-varying patterns in sedimentation often manifesting as complex three-dimensional morphologies (e.g., beach cusps) and instances of grain size sorting. The large tide range means a reduction in the force applied per unit of beach surface area and the heightened potential of observable morpho-sedimentary signals. However, because of the typically energetic shorebreak at Advocate Beach and at coarse-grained beaches in general, where impulsive wave forcing and large mobile grains impose a hazard to instrumentation, conventional \textit{in situ} sampling methods employed at sandy beaches are not suitable.


%\begin{figure}[tbp] %/home/tristan/Documents/Projects/AdvocateBeach2018/src/visualization/plot_beach_profile_data.py
%%  	\includegraphics[width=0.5\columnwidth]{/home/tristan2/Documents/PhD/Dissertation/figures/chapter3/mean_MGS_crossshore.png}
% 	\caption{Bulk grain size distribution from Advocate 2015 experiment. \label{fig:bulk_grainsize}}
%\end{figure}

%lognormal grainsize dist?
%Grain size distribution is broad, and is best described by a log-normal distribution. 


\section{2015 Field Experiment}\label{Intro:Adv2015}

The 2015 Advocate Beach field campaign (Fig. \ref{fig:Profile}, p. \pageref{fig:Profile}; Fig. \ref{fig:beach_profile}, p. \pageref{fig:beach_profile}) was carried out between 21 September and 9 October (i.e., yeardays 264-282), with the purpose of characterising the vertical pore pressure response of the bed under wave loading, and the morphodynamics of beach cusps. Data were obtained primarily using arrays of buried pressure sensors and a fixed-frame video system. The pore pressure response was evaluated in terms of the depth dependence of the attenuation and phase of the oscillatory component of the pore pressure signal through the mixed sand-gravel medium. Beach cusp morphodynamics were observed by tracking the time evolution of video-derived shoreline estimates.

Measurements of sediment pore water pressure were made using a vertical array of four MS5803-14BA high resolution (0.02 kPa) pressure sensors spanning the upper 50 cm of the sediment column (0, 15, 30, and 50 cm sediment depth), near the mid-tide level. Each sensor was secured in an acrylic housing, the openings of which were covered in aluminum mesh to prevent the mechanical force of sand grains from impinging on the sensing surface. The array was installed on yearday 265, and operated from yearday 266 to 281. The verticality of the array was established during installation using a spirit level, and was confirmed to still be vertical upon the excavation of the array at the end of the campaign. Data were sampled at a mean rate of 50 Hz, then resampled at 20 Hz for analysis. Two BeagleBone Black single-board computers were used for data logging. The BeagleBones were networked and time synchronised using Precision Time Protocol (PTP) via a junction box at the top of the berm to ensure accurate phase estimates. Power was supplied from a building behind the dune crest. Additional pressure measurements were made using sensors arranged in longshore and cross-shore arrays between the mid- and high tide levels for the purpose of observing low mode edge waves. The sensors in the longshore and cross-shore arrays were buried 30 cm deep in the beach and ported to the beach surface to avoid potential phase-shifting effects introduced by the sediment. 

Video imagery was recorded daily from a fixed-frame camera installed atop a 5.3 m tower on the beach crest, facing southward. The camera field of view contained the intertidal zone, and spanned approximately 30 m of shoreline at spring high tide and nearly 100 m at low tide. The images were georectified and rotated into local coordinates, with $x$ alongshore, positive to the northwest, $y$ positive offshore, and the origin defined as the position of the vertical pressure sensor array -- roughly 20 m seaward of the nominal high tide shoreline. Video was recorded with a resolution of 1920x1080 pixels at 2 Hz for roughly 6 hours per tide during daylight, corresponding when possible with the period before, during, and after high tide.  

Time-averaged images were used to estimate the shoreline position and track the evolution of the swash zone morphology over the course of selected high tides. The averaging interval was three minutes (360 frames). Here, the term ``shoreline'' is defined to be the $x$-$y$ position of the water's edge, as determined from the three-minute time-averaged imagery. The video-tracked shorelines were used to estimate the wavelengths and amplitudes of beach cusps. The instantaneous position of the water's edge was tracked using `timestack' images composed of cross-shore pixel transects (i.e., fixed $x$-coordinate) from the 2 Hz images stacked consecutively through time. The timestacks were used to estimate swash runup statistics.

Pressure measurements for monitoring incident wave conditions were obtained using an internally logging RBR Duo, deployed on an above-bed frame near lowest low water, sampling continuously at 6 Hz between yeardays 267 and 281. Laboratory measurements of sediment porosity and hydraulic conductivity were made following the experiment, using sediment samples collected from the upper beach face. Hydraulic conductivity was estimated using constant head permeameter tests, as outlined by \citet{Craig1974}. Grain size distributions were obtained following the method of \citet{Ingram1971}. 

Beach surveys were carried out at each daylight low tide using a cart-mounted RTK (real time kinematic) GPS.

The forcing conditions during the experiment were characterised by periods of wave inactivity (yeardays 265-270 and 275-277), owing both to light winds and to winds from the N-NE (offshore), and interspersed with wind events from the SW, leading to the steep, locally generated wind and swell waves that are characteristic of the Bay of Fundy. The first, and largest, of the wind events lasted two days (yeardays 270-271) with peak significant wave heights approaching 1.5 m. Two other SW wind events occurred on yeardays 274 and 281. During the period from yearday 277 to 280, wave energy spectra were dominated by long period (\textgreater 15 seconds), low amplitude (0.05-0.15 m) swell waves generated by Hurricane Joaquin which passed well to the south of Nova Scotia. The mean peak period, computed for storms (tide-averaged significant wave heights greater than 0.15 m) and intervals between storms, was 7.6 seconds and 9.7 seconds, respectively.




\section{2018 Field Experiment}\label{Intro:Adv2018}

The 2018 Advocate Beach experiment (Fig. \ref{fig:survey_grids}, p. \pageref{fig:survey_grids}; Fig. \ref{fig:frames} p. \pageref{fig:frames}) was conducted between 14 and 27 October, with an emphasis on the coevolution of bed level and mean surficial grain size both at the beach scale and at the scale of the swash zone, and on Lagrangian tracking of natural cobble tracers in swash flows. At the beach scale, bed level and grain size were observed using GPS and photo surveying each low tide. An array of ultrasonic range sensors and cameras was used to observe the subaerial bed level and grain size at the swash scale, and video data used for the Lagrangian particle tracking. The experiment spanned 27 tides, which are referred to by their low tide index within the experiment, i.e., 1 though 27.

Local grid coordinates were defined in keeping with a previous experiment at the site, with the origin defined as the former position of a vertical pressure sensor array (see Chapter \ref{Intro:Adv2015}), approximately 20 m seaward of the nominal high tide shoreline. $x$ is alongshore, positive to the northwest, and $y$ is positive offshore.

Beach-scale monitoring of bed elevation and beach surface grain size was carried out using RTK (real time kinematic) GPS and camera surveys. A survey grid (Fig. \ref{fig:survey_grids}) was established consisting of one cross-shore beach transect spanning 90 m and two longshore transects spanning 75 m, all sampled at 3 m intervals, as well as a more densely sampled grid, consisting of six 24 m longshore transects spaced at 1 m intervals, and 2 m intervals across-shore. The cross-shore transect spanned from the beach crest to the mean spring tide low water shoreline along the $x=0$ m coordinate, though the number of surveyed points differed from tide to tide depending on the shoreline position at the time of the survey. The two longshore transects were positioned on the upper beach face, but seaward of the nominal high water line. The most seaward transect was positioned at $y=-5$ m, and the more shoreward transect at $y=-13$ m. The densely sampled grid, and the larger grid layout in general, were designed to encompass the high tide shoreline, in order to emphasise high tide swash processes. 

The survey grid was sampled every low tide from tide 14 to 27, an RTK GPS position and photograph being taken at each grid point. Photographs were taken using a 20 megapixel Canon Powershot Elph 190 camera, mounted to a tripod at fixed height. Prior to tide 14, different grid configurations were surveyed, and only one survey was conducted per day (i.e., once per two tides). Survey data collected prior to tide 14 are not included in any analyses. 

A wavelet-based digital grain sizing package \citep[see][]{Buscombe2013}, implemented in Python, was used to estimate arithmetic grain size statistics from the survey photographs. The DGS algorithm does not require calibration, and takes as input a grain-resolving image containing only sediment. The survey images were cropped to half width and height, centred in the frame, corresponding to a field of view at the bed of 0.33$\times$0.25 m, assuming a camera height of 0.3 m above the bed. The pixel to physical unit scaling was computed by photographing an object of known length and width. The same scaling was used for all survey images.  

At the swash scale, data were collected using a four-element array of collocated ultrasonic range sensors and cameras, along with an overhead camera used for tracking the movements of tracer cobbles in the swash. Both systems were mounted over the swash zone during periods of low to moderate energy forcing conditions.

The coincident evolution of beach morphology and sediment properties in the swash zone was investigated using an array of collocated Maxbotix MB7383 HRXL ultrasonic range sensors (range resolution of \textit{ca}. 1 mm) and 5 megapixel Raspberry Pi cameras. The array consisted of four downward-facing range sensor-camera pairs, cantilevered approximately 2 m over the swash zone on an instrument frame that could be moved with the changing shoreline position. The pairs were separated by 0.9 m alongshore (i.e., having a total longshore span of 2.7 m), and at a nominal elevation of 0.75 m above the bed. The range data were sampled at 6 Hz, and the images at 0.2 Hz. Each of the four array element pairs were controlled by a Raspberry Pi single board computer, on which the data were also logged. The four computers were time-synchronised using network time protocol (NTP), and powered from a 12 V marine battery. The instrument frame was assembled near the high water line and data were collected at 3-5 ``stations" during late flood tide, high tide, and early ebb. At each flood tide station, sampling was initiated prior to the maximum swash runup position passing beneath the array, and continued until the bed was fully obscured by water (i.e., the swash zone was no longer in the instruments' field of view). At ebb stations, the frame was positioned so that the transition from full water cover to fully-exposed bed could be captured. The longest sampling periods corresponded to high tide, when the array stations could be held through the shoreline's advance and retreat, from late flood into early ebb. The position of each array element was recorded using RTK GPS at each station. Images of the exposed bed (i.e., between instances of swash runup) captured by the Raspberry Pi cameras were processed using the same digital grain sizing algorithm used to analyse the survey images.

An overhead Raspberry Pi camera used for monitoring tracer cobble transport in the swash zone was mounted to a second instrument frame. The frame consisted of a semi-stationary base which could be moved with the changing shoreline position, and a moveable arm which supported the camera, allowing it to view the swash zone from a height of \textit{ca}. 3 m without the frame base being in the image. The image field of view at the beach surface was approximately 2.4 by 4.3 m (longshore by cross-shore). During low to moderate wave energy conditions, the overhead camera frame occupied 3-5 stations near the high water line. A minimum of three ground control points were captured at each station using RTK GPS to provide a scaling between pixel and ground coordinates. Prior to deployment, cobbles were sieved into three different size classes: 22.4-31.5 mm, 31.5-45 mm, and 45-63 mm. The cobbles were painted blue, orange, and yellow, respective to each size class. The camera frame stations were typically chosen to capture cobble transport during high tide and early ebb, so the cobbles could be retrieved and redeployed after the shoreline had retreated. The frame held a station until the camera's field of view no longer contained the mean shoreline position. Video was recorded continuously throughout.  

The alongshore orientation of the four range sensor-camera elements on the array frame was chosen to capture the development of beach cusps, or other three-dimensional morphology in the alongshore, though no beach cusp events were satisfactorily captured. The geometries of both instrument frames (i.e., their fields of view in relation to their bases, which were in contact with the beach surface) were chosen so that the mid-swash zone could be sampled without the bases of the instrument frames interfering with either the data collection, or the swash processes being observed. This limited the use of the frames to `fairweather' conditions, during which the maximum swash runup distance was less than approximately 4 m, favourable for cusp formation \citep{Guest_Hay2019}. The majority of tides were characterised by high steepness, wind-band incident waves leading to an energetic shorebreak. There were three periods of fairweather forcing characterised by low amplitude, low steepness waves during which the swash array frame could be deployed: tides 15-16, tides 19-22, and tide 27.

An RBR Duo pressure sensor was deployed on a heavily weighted frame resting on the bed near the neap low water shoreline to observe the ``offshore" wave climate. Due to a prolonged period of high winds and energetic wave conditions at the outset of the experiment, the pressure sensor was not deployed until the sixth day of the experiment (prior to tide 10). Pressure data were recorded at 6 Hz.
