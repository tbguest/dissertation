\chapter{Morpho-sedimentary dynamics in the intertidal zone}\label{Chapter:MSDBeach}

%\begin{abstract}
%	
%	The field site is a steep (1:10 slope), mixed sand-gravel beach in Advocate, Nova Scotia, Canada. Located near the head of the Bay of Fundy, the tidal range at the site ranges from 8-12 m.
%
%\end{abstract}


%Pontee et al 2004 (3 MSG beaches, 2-3 m tide range, sampled 4 times/yr, or once per month for 1 yr)
%- a number of previous authors have suggested that the presence of both sand and gravel results in morphodynamics that are distinct from those of both pure gravel and pure sand beaches (ZENKOVITCH, 1967; MCLEAN, 1970; KIRK, 1975).
%- larger wave heights and longer periods produce a beach surface with smaller mean gravel size
%- larger wave heights produce beaches with higher percentage sand.
%(all with statistical significance in 1-2 of 3 cases)
%- states similarities with sand and pure gravel beaches
%	- profile response to semidiurnal and spring-neap tides
%	- dominance of waves in governing beach profile changes
%- profile characteristic similarities to gravel and other MSG beaches:
%	- abundance of gravel ridges landward of HWM
%	- high level ridges (berms) and wide, planar beach face region in storms	
%- berms owe their formation to thelandward movemwnt of gravel thru flood tide and subseq deposition during brief still stand of water levels at HWM.
%- tendency of gravel particles to move onshore due to:
%	- shallow water wave asymmetry prior to breaking 
%	- asymmetry of swash action due ot swash infilt losses
%	- increased mobility of lasrge relaticve to smaller pebbels
%	- concentraton of coarser particles at sed surface
%	(look up associated refs)
%(my words:) the shore ward movement of coarse material under wave action has been well documented. 
%- variability of beach composition under similar wave conditions suggests additoinal contrls such as feedback effects from antecedent ebach/nearshore areas
%- onshore transport of sand from the nearshore region during periods of post-storm swell (cf MASSARI and PAREA, 1988) may explain the sandy characteristics of post-storm profiles
%- fining of beach surface after larger, longer period waves may be due to the concentration of larger gravel sizes in smaller portions of the beach profile (ie berm and step), coupled with the growth of the beachface composed of finer material between these two regions.

%Curtiss et al Mar Geo 2009
%- "A shift to a stable, coarse sand composition on the upper beach is adistinct characteristic of the storm condition on the beach. Theobservations are consistent with a reverse winnowing process, inwhich the gravel is more mobile than thefine fraction. However, we cannot be certain of the sand behavior without direct measurementsof thefine fraction. After the cessation of storm waves, gravel slowlyaccretes on the upper beach. Ferry wakes appear to be a mechanismfor post-storm recovery of gravel on the upper beach. Tidal currentsare not a likely mechanism because they are not strong enough tomobilize gravel-sized sediment. Such changes in beach composition,which may be somewhat unique to MSG beaches, are very challengingto model and difficult to observe due to their spatial heterogeneity andthe long timescales over which they occur. Although the dynamicaldetails related to this observed shift in beach composition are notwell-resolved in the current study, it is a striking feature of theseasonal variability on the beach and, thus, likely to be important inthe overall morphological and sediment transport regime."

%Miller et al Mar geo 2011
%- "We observe thathigh waves tend to be associated with an increase in the sand fractionvisible on the bed, suggesting thorough mixing of the beach substrate."
%- "This size-selective model suggests that identifying a significantdifference between size classes may be more likely in low energyconditions, when clasts in transport show a greater degree of size-selective transport."

%Nordstrom and Jackson (1993) (macrotidal) and Pontee et al (2004) (2-3 m tide ranges) both observed MSG surficial fining during storm conditions!
%
%Nordstrom and Jackson (1993) also used painted tracer stones (beach composed mainly of sand)! They propose a cobble transport conceptual model. During low energy conditions, pebbles are more likely to be transported to the step (inconsistent with our results).
%
%From Busc + Mass: Duncan (1964) describes stranding of coarse material at top  of swash leading  to berm building, supported by eriksen(1970), waddell(1976), Horn et al (2003), ausin and mass (2005)
%
%
%Horn and Li (2006): few measurements of swash zone bed level on gravel beaches. (Kulkarni et al, 2004 did for MSG)
%- also observed accretionary berm feature migrating landward with flood tide.
%- KULKARNI et al. (2004) reported accretion rates of 10-15 cm/h on a mixed sand and gravel beach (berm)
%- HORN and WALTON (2004) found that the measured high-frequency variation on a mixed sand and gravel beach was nearly as much as the overall amount of bed level change, measuring up to 7.5 cm of elevation change in 10 seconds.

% [Mass 2007: "The absence of distinct secondary morpho-
%logical features on macrotidal beaches and their morphological stability are primarily caused by the tide-induced migration of the different mor- phodynamic zones (swash, surf and shoaling) across the beach profile. The significance of tidal translation for beach morphological development can be quantified by the ratio of tidal range to wave height (Masselink, 1993)"]
%
%"Smearing" of features across the intertidal zone.


% Nordstrom ANd Jackson - tidal
% P D OSborne, Mar Geo 2005 - tidal, RFID, painted, etc
% Allan, Hart, Tranquili Mar Geo 2006, tidal, RFID 
% Dickson, Kench, Kantor, Mar Geo 2011, tidal
% Miller, Warrick, Mar Geo 2012, RFID, tidal
% Stark, Hay, tidal
% Curtiss et al Mar Geo 2009, RFID, tidal
% Bertoni, Sarti et al, Mar Geo 2012, RFID, tidal



\section{Introduction}

Mixed sand-gravel (MSG) beaches represent one of three geomorphic subtypes of gravel beach, the other two subtypes being pure gravel and composite gravel, as outlined by \cite{Jennings_Shulmeister2002}. MSG beaches are generally characterised by a swash-dominated hydrodynamic regime, and well-mixed sediments both across-shore and at depth. The need for improved understanding of MSG beach processes has been emphasised in the literature of the last two decades, given the relative stability of coarse grained beaches in the face of energetic conditions and rising sea levels, and the increasing usage of mixed-sized sediments in beach replenishment schemes \citep{Mason_Coates2001}. 

% p: Something about relationship between morphology and sediments

Of particular note in understanding the evolution of MSG beaches is the relationship between beach surface texture, as determined by properties of the surficial grain size distribution, and the hydrodynamics, groundwater dynamics (hydraulics), and morphology. The morphodynamic model \citep{Wright_Thom1977, Buscombe_Masselink2006}, which for decades has been used to describe the evolution of sandy coastlines, cannot account for the changes in the morphological response that result from textural differences in bed state. Though linkages between bed texture and morphology have been discussed in the literature for decades \citep[e.g.,][]{LonguetHiggins_Parkin1962, Moss1962, Isla1993, Carter_Orford1993, Sherman_etal1993, Masselink_Li2001, Buscombe_Masselink2006, Masselink_etal2007, Austin_Buscombe2008, VanGaalen_etal2011, Guest_Hay2019}, the relative importance of sedimentary feedbacks in morphological evolution remains unclear. In their gravel beach review paper, \citet{Buscombe_Masselink2006} proposed ``morpho-sedimentary dynamics'' (MSD) as a preferred framework within which to carry out future research on gravel beaches. The framework treats differences in bed texture resulting from variations in the characteristics of the surficial grain size distribution as an intrinsic component of the dynamics of morphological evolution. 

% p: transport in MSG

Sediment transport on MSG beaches is governed by complex interactions between forcing by waves and tides \citep{Nordstrom_Jackson1993}, grains of different size and shape \citep{Stark_etal2014, Stark_Hay2016}, and hydraulic effects \citep[in/exfiltration and groundwater response;][]{Horn2006}. Spatial and temporal variability in grain size properties, hydraulic conductivity, and mobilisation thresholds, along with transport regimes which may be characterised by conditions of fractional mobilisation of material \citep[e.g.,][]{Wilcock_Crowe2003}, or dependence of individual particle transport on the background substrate \citep{Carter_Orford1993}, pose a challenge for the incorporation of MSG transport in contemporary predictive models. Indeed, there are indications that the phenomenological response of MSG beach sediments to forcing may be fundamentally different from the response expected on a sand beach: Several authors reporting on MSG beach sedimentology \citep{Nordstrom_Jackson1993, Pontee_etal2004, Curtiss_etal2009, Miller_etal2011, Hay_etal2014} have observed decreases in the mean grain size of beach surface sediments in response to energetic wave forcing, and surficial coarsening following fairweather conditions. On sandy beaches, the conventional understanding is of offshore transport of finer material during instances of increased wave height and steepness, due to swash velocity asymmetry, and onshore transport during low energy forcing, leading to differences in beach characteristics seasonally, or with differences in orientation \citep[e.g.,][]{Bascom1951}. The typically anticipated positive relationship between forcing and surficial grain size on sandy beaches is implicit in recent grain size response modelling efforts \citep[see][]{Prodger_etal2016}, which have been shown to predict the mean surficial grain size with considerable skill.

There are few examples in the field literature of studies designed to investigate the correlation between morphological and sedimentary (hereafter, morpho-sedimentary) dynamics. Those studies that have sought signatures of correlation between bed level and grain size have had mixed results. \citet{Masselink_etal2007} surveyed bed level and grain size across the profile of a macrotidal sandy beach over a 20 day period. They reported changes in mean grain size and sorting that were mostly unrelated to the morphological response, with two exceptions wherein instances of erosion were associated with bed surface coarsening. \citet{Austin_Buscombe2008} collected observations of bed level and grain size properties across the intertidal zone of a macrotidal gravel beach over the course of a tidal cycle. They reported some positive correlation between mean grain size and bed level change at the locations of the berm and the step, though distinct phases of morphological change were not otherwise evident in the sedimentary signal. Both \citet{Masselink_etal2007} and \citet{Austin_Buscombe2008} acknowledged their sampling strategies as a potential limitation on revealing clearer trends in the grain size measures.

Mixed substrates have been suggested to favour the formation of transient, secondary morphological features or patterns on the beach surface, including beach cusps \citep[e.g.,][]{LonguetHiggins_Parkin1962, Guest_Hay2019}. The tendency for gravel (especially MSG) beach sediments to display heterogeneity in space and time, and to self-organise into sorted sediment structures with linkages to morphological features \citep[e.g.,][]{Sherman_etal1993, Austin_Buscombe2008} makes MSG beaches well suited to investigations of the coevolution of morphology and grain size. 

% p: Objectives + layout

Here, results are presented from a 2018 field study of a MSG beach. The study made use of survey observations of beach surface elevation and mean surficial grain size sampled at low tides over a two-week period. The objective of this study is to investigate the coevolution of bed elevation and mean grain size, seeking insight into the phenomenological role of grain size in swash zone morphological evolution, and to discuss the observed and uncommon phenomenon of MSG beach surficial sediment fining immediately following energetic forcing events. %The chapter is organized as follows... .




%\section{Site description}
%[not needed in chapter; will be needed in paper]
%
%With the exception of tidally transient sediment structures on the beach surface (e.g. beach cusps, patchiness), the beach is alongshore uniform, and generally well-described in terms of the across-shore profile. 
%
%Incident waves are generally broad banded, wind-generated. waves break directly on the beach surface, with no clearly defined surf zone. As such, swash periods generally reflect the period of the incident waves, with little or no opportunity for long wave generation. 
%
%% forcing section?:
%Despite the steep slope, the conditions at Advocate Beach can generally be described as dissipative,  owing to the typically steep, sea-band incident waves ($R^{2} \approx 0.14$ for the Advocate 2018 experiment, approximated using the Miche equation: Miche (1951), Elgar et al (1994)[herbers guza JPO])
%
%Grain size distribution is broad, and is best described by a log-normal distribution. 


\section{Methods}\label{section:ch3_methods}

\subsection{Experiment Overview}\label{subsec:ch3_experiment_overview}

The field experiment was conducted between 14 and 27 October, 2018 at Advocate Beach, located in Nova Scotia at the head of the Bay of Fundy. The 27 tides spanned by the experiment are referred to throughout this chapter by their low tide index: tides 1 through 27. %The focal point of the experiment was to investigate the coevolution of morphology and sedimentology, with an emphasis on processes in the swash zone.

Beach-scale monitoring of bed elevation and beach surface grain size was carried out using RTK (real time kinematic) GPS and camera surveys. Local grid coordinates were defined in keeping with previous experiments at the site (Chapters \ref{Chapter:PorePressure} and \ref{Chapter:CuspDynamics}), with the origin defined as the former position of a vertical pressure sensor array (see Chapter \ref{Chapter:PorePressure}), approximately 20 m seaward of the nominal high tide shoreline. The coordinate system is such that $x$ is alongshore, positive to the northwest, and $y$ is positive offshore. A survey grid was established (see Fig. \ref{fig:survey_grids}), consisting of one cross-shore beach transect spanning 90 m ($y=$-30 to 60 m in local grid coordinates) and two longshore transects spanning 75 m ($x=$-50 to 25 m), all sampled at 3 m intervals, and a more densely sampled grid consisting of six 24 m longshore transects spaced at 1 m intervals, and 2 m intervals across-shore ($x=$-24 to -1 m, $y=$-15 to -5 m). The cross-shore transect (hereafter referred to as CT) extended from the beach crest to the mean spring tide low water shoreline along the $x=0$ m coordinate, though the number of surveyed points differed from tide to tide depending on the shoreline position at the time of the survey. The two longshore transects were positioned on the upper beach face, but seaward of the nominal high water line. The most seaward transect (LT1) was positioned at $y=-5$ m, and the more shoreward transect (LT2) at $y=-12$ m. The densely sampled grid (DG), and the larger grid layout in general, were designed to encompass the high tide shoreline, in order to emphasise high tide swash processes. 

\begin{figure}[tbp] %/home/tristan/Documents/Projects/AdvocateBeach2018/src/visualization/plot_beach_profile_data.py
  	\includegraphics[width=\columnwidth]{figures/chapter3/plan_view/survey_grids.png}
 	\caption[Advocate Beach aerial photograph and survey grid configuration]{Aerial photograph of Advocate Beach, with the survey configuration overlaid. A GPS point and photograph were taken at each survey station (yellow dot), each low tide. The survey layout consisted of four subsets: one cross-shore transect (CT), two longshore transects (LT1 and LT2), and a more densely sampled grid (DG). The red dot indicates the position of the pressure transducer (PT). \label{fig:survey_grids}}
\end{figure}

The survey grid was sampled every low tide from tide 14 to 27, an RTK GPS position and photograph being taken at each grid point. Photographs were taken using a 20 megapixel Canon Powershot Elph 190 camera, mounted to a tripod at fixed height. Prior to tide 14, different grid configurations were surveyed, and only one survey was conducted per day (i.e., once per two tides). Data collected prior to tide 14 are not included here. 

The cross-shore coordinate of the high water line (HWL; herein defined as the cross-shore coordinate of the high tide swash runup maximum) for each tide was determined post-experiment using the survey photography, by identifying features associated with the maximum swash runup limit: seaweed or similar flotsam, the transition from wet to dry substrate, or clear boundaries in sediment size or shape. Thus, given the cross-shore spacing of stations in grid DG, the HWL coordinates are considered accurate to within $\pm1$ m. 

An RBR Duo pressure sensor was deployed on a heavily weighted frame resting on the bed near the neap low water shoreline to observe the ``offshore" wave climate. Due to a prolonged period of high winds and energetic wave conditions at the outset of the experiment, the pressure sensor was not deployed until the sixth day of the experiment (prior to tide 10). Pressure data were recorded at 6 Hz.

The coevolution of beach morphology and sediment properties in the swash zone was investigated using an array of collocated ultrasonic range sensor and camera pairs, as well as an overhead camera used for monitoring tracer cobble transport. These data are presented in Chapter \ref{Chapter:MSDSwash}.

\subsection{Digital Grain Sizing}

A wavelet-based digital grain sizing (DGS) package \citep[see][]{Buscombe2013}, implemented in Python, was used to estimate arithmetic grain size statistics from the camera imagery. The DGS algorithm does not require calibration, and takes as input a grain-resolving image containing only sediment. 

The survey images were cropped to half width and height in the centre of each frame, corresponding to a field of view at the bed of 0.33$\times$0.25 m, with the camera height set at \textit{ca}. 0.3 m above the bed. Input parameters for the algorithm include a pixel to physical unit scaling, a maximum feature diameter to be resolved, and a dimensional scaling factor. The pixel to physical unit scaling was computed by photographing an object of known length and width. The same scaling was used for all survey images. The maximum feature diameter, defined as the inverse ratio of the pixel width of the frame to the width of the largest feature to be resolved, was set at 56 mm. The dimensional scale factor was set at 0.8. See Appendix \ref{Appendix2} for a discussion of the choice of dimensional scale factor, as well as for more detailed descriptions of the remaining input parameters. 

Though the algorithm is capable of returning a full grain size distribution, validation of the output against the distributions from both sieve and manual point count analyses \citep[see][or Appendix \ref{Appendix2} for a description of the point count method]{Barnard_etal2007, Buscombe_etal2010} indicated that only the lowest moment of the grain size distribution (mean grain size) was captured satisfactorily. The validation of the higher order moments (i.e., sorting, skewness, kurtosis) revealed that they were not well estimated by the algorithm, so only the mean grain size results are included here. We attribute the algorithm's poor representation of the higher order moments to the wide grain size distribution. See Appendix \ref{Appendix2} for further discussion of the validation procedures.


\subsection{Correlation Analysis}\label{subsec:ch3_correlation_methods}

Pearson correlation coefficients were computed between measures of bed level, $z$, and mean grain size, MGS, both defined at a given set of space and time coordinates: e.g., $z=z_{ij}(t)$, where the indices $i,j$ indicate the $x,y$ (longshore, cross-shore) coordinates, and $t$ is an integer low tide index. MGS and other, similar measures are equivalently defined. Changes in bed level and mean grain size from one tide to the next are denoted using $\Delta$ notation, e.g., $\Delta z = z_{ij}(t) - z_{ij}(t-1)$. Primed values denote observations associated with the previous time (i.e., tide) step, e.g., $z'=z_{ij}(t-1)$, and overbar notation is used where spatial averaging was done. That is, the spatial mean of the MGS observations for a survey transect with $N$ stations is given by 

\begin{equation}\label{eq:mean}
\overline{\mathrm{MGS}}_j = \frac{\sum_{i=1}^{N} \mathrm{MGS}_{ij}(t)}{N}. 
\end{equation}

\noindent Spatial means were computed only over longshore survey transects (note the summation over $i$ alone in equation (\ref{eq:mean})), since the observed properties -- notably MGS -- had a characteristic (i.e., non-Gaussian) structure in the across-shore. The subscript $j$, indicating the cross-shore coordinate associated with the transect over which the computation was made, is implicit in the equations below, but is hereafter omitted. The Pearson correlation coefficient for a given cross-shore coordinate associated with a given tide is

\begin{equation}\label{eq:pearsons_r}
r = \frac{\sum_{i=1}^{N}(X_i-\overline{X})(Y_i-\overline{Y})}{\sqrt{\sum_{i=1}^{N}(X_i-\overline{X})^2 (Y_i-\overline{Y})^2}},
\end{equation}

\noindent where $X$ and $Y$ are the observed properties.

Correlation coefficients were computed between changes in bed elevation from the previous tide's survey ($\Delta z$) and: (1) the mean grain size (MGS), (2) the change in mean grain size from the previous survey ($\Delta$MGS), (3) the mean grain size observed during the previous survey (MGS$'$), and (4) the change in bed elevation from the preceding time step ($\Delta z'$). Correlation coefficients between hydrodynamic parameters, bed level, and mean grain size were also computed. Statistical significance was evaluated based on whether confidence intervals contained zero at an $\alpha=0.05$ significance level.

The $\Delta z$, $\Delta z'$ correlations were carried out based on the assumption that changes in bed level represent, on average, deviations from the mean profile. In other words, a positive $\Delta z$ value is indicative of a topographic high at a given location, relative to the mean profile. This correlation is susceptible to negative bias if noise levels are high relative to the actual topographic change. It can be shown using a synthetic series of Gaussian noise that the Pearson correlation coefficient of the series with itself at the previous index approaches -0.5 for a large number of samples. The RTK GPS is taken to be accurate to within $\pm$0.02 m in the $z$-direction. With the exception of points surveyed above the HWL, the distribution of $\Delta z$ values for a given survey generally exceeded the expected distribution of values due to instrument inaccuracy alone. As verification, correlation coefficients between $\Delta z$ and $\Delta z'$ were computed for stations landward of the HWL. These were found not to be significantly different from zero, suggesting that the potential for negative bias at stations seaward of the HWL -- where the range of true $\Delta z$ values would be expected to be larger -- is low.

Because the distribution of mean grain sizes had a characteristic structure in the across-shore direction (coarser sediments at, and shoreward of, the berm region), only data from the longshore transects were cross-correlated (i.e., LT1, LT2, and each of the six rows of DG), to obtain meaningful correlations. In order to carry out correlations with large numbers of samples, $\Delta z$ and $\Delta$MGS observations for each longshore transect were combined, with their means removed at each tide step, and correlations carried out with the combined data sets. In other words, beach-scale changes in bed volume and mean grain size that may have resulted from, e.g., high energy forcing events, were removed, leaving only variations in longshore space relative to the mean of each transect.

Measures for which subtracting the longshore spatial mean would be less meaningful, i.e., MGS and MGS$'$, were correlated with $\Delta z$ only for individual tides, and not combined. This treatment also omits large scale changes in bed level or mean grain size from tide to tide, instead emphasising spatial variations in the longshore direction.

Temporal correlations were carried out between bed level, grain size, and hydrodynamic parameters by averaging the values from individual tides. Only data from the longshore survey transects were used for the bed level and grain size data. Thus,

\begin{equation}\label{eq:pearsons_r_time}
\hat{r} = \frac{\sum_{t=t_0}^{T}(\overline{X} - \langle \overline{X} \rangle)(\overline{Y}-\langle \overline{Y} \rangle)}{\sqrt{\sum_{t=t_0}^{T}(\overline{X}-\langle \overline{X} \rangle)^2 (\overline{Y}-\langle \overline{Y} \rangle)^2}},
\end{equation}

\noindent where the $\hat{\cdot}$ indicates temporal correlation, $\overline{X}=\overline{X}(t)$, the spatial mean of $X$ at tide index $t$, $t_0$ and $T$ are the indices of the first and last tides in the sequence, and $\langle \overline{X} \rangle$ is a temporal mean, given by

\begin{equation}\label{eq:time_mean}
\langle \overline{X} \rangle = \langle \overline{X(t)} \rangle = \frac{\sum_{t=t_0}^{T} \overline{X(t)}}{T-t_0},
\end{equation}

\noindent where $T-t_0+1$ is the total number of contiguous tides.


\subsection{Wave Parameter Estimation}

Data from the pressure transducer deployed near the neap low water line were processed in 12 min segments to compute standard wave statistics: significant wave height, $H_s$, computed as $4\sigma_p$ with $\sigma_p$ being the square-root of the variance of the pressure time series; and the peak wave period, $T_p$, corresponding to the location in inverse frequency space of maximum spectral density in the pressure spectrum. %Break-point significant wave heights were computed following the method of \citet[][p. 115]{Dean_Dalrymple1984}. 

% , and Iribarren number $\xi_0=\tan \beta / \sqrt{H_0/L_0}$, with $\beta = 0.12$ being the beach slope, were
% \citep[$\xi_0 > 3.3$,][]{Battjes1975}

The `deep water' wave steepness $H_0/L_0$ was computed to help draw a distinction between forcing states leading to an energetic shore break ($H_0/L_0 > 0.01$, $H_s > ~0.5$ m) and more reflective conditions characterised by collapsing- or surging-type wave breaking ($H_0/L_0 < 0.01$) more typically associated with the formation of a berm, cusps, or other patterned bed states at Advocate Beach. In the wave steepness calculation, the deep water wave height, $H_0$, was taken to be equivalent to $H_s$, and the deep water wavelength, $L_0$, was computed using the linear surface gravity wave dispersion relation at the frequency of peak forcing ($=1/T_p$).


\section{Results}\label{section:ch3_results}

\subsection{Overview of Forcing and Response}\label{subsection:beachMSD}

%[might need to include more morphodynamic stuff here -- vol change and hydrodynamic stuff? This will allow more comparison with austin, mass, busc work in the discussion. Can also acknowledge in the discusion that morophologicla comparisons (esp wrt volume) are made difficult by the changing positoinof the HWL relative to the survey grid -- surveying differetn parts of the profile with each tide.]

% Do need to produce plots of volume change (essentially mean dz) over LT1, so the relnship between dz and dmgs isn't muddied by that (having thought about this, I think I'm satisfied that my tide by tide breakdown approach ensures we're looking at changes about a 'master' bed profile, independent of net volume change.). 

A summary of the forcing conditions during the Advocate 2018 experiment is presented in Fig. \ref{fig:wavestats}. The experiment was conducted during a transition from neap to spring tides, with the tidal range increasing from \textit{ca}. 7.5 m at the outset to 11 m by the end of the experiment. The majority of tides were characterised by high steepness, wind-band incident waves leading to an energetic shorebreak. Sample energy spectra from high and low energy days are shown in Fig. \ref{fig:psd}.

\begin{figure}[tbp] %/home/tristan/Documents/Projects/AdvocateBeach2018/src/visualization/plot_beach_profile_data.py
	\includegraphics[width=\columnwidth]{figures/chapter3/wave_stats.png}
	\caption[Wave data: Advocate 2018 experiment]{Summary wave data from the Advocate 2018 experiment. (a) Height of water, $h$, above the pressure transducer. (b) Significant wave height. (c) Peak wave period. (d) Wave steepness. The red dots indicate first data points of each tide.} %(e) Iribarren number.}
	\label{fig:wavestats}
\end{figure}

\begin{figure}[tbp] %/home/tristan/Documents/Projects/AdvocateBeach2018/src/visualization/plot_beach_profile_data.py
	\begin{center}
	\includegraphics[width=0.75\columnwidth]{figures/chapter3/psd.png}
	\caption[Energy spectra for high and low energy wave forcing]{Sample energy spectra for characteristic low energy (tide 19; blue) and high energy (tide 23; black) forcing days, computed using pressure data from the low tide frame-mounted pressure transducer.}
	\label{fig:psd}
	\end{center}
\end{figure}

The transition between neap and spring tides had important implications for the swash zone morpho-sedimentary processes: most notably, the presence of a coarse-grained high tide berm which persisted during fairweather conditions, migrating landward throughout the experiment. A second, inactive spring tide berm was also present throughout the experiment (Fig. \ref{fig:alltides_profiles}). Throughout this chapter, use of `berm region' is in reference to the region extending from the high water line to the base of the seaward face of the active high tide berm, where a natural break in the morphological and sedimentary profiles (positive to near zero-valued $\Delta z$, and decreased MGS in Fig. \ref{fig:alltides_profiles}) separated the berm from the remainder of the intertidal zone. The intertidal zone seaward of the berm region is hereafter referred to as the `mid-intertidal zone'.

\begin{figure}[tbp] %/home/tristan/Documents/Projects/AdvocateBeach2018/src/visualization/plot_beach_profile_data.py
  	\includegraphics[width=\columnwidth]{figures/chapter3/surfaceplots_dz_grainsize.png}
 	\caption[Surface plots of bed level change and mean grain size by tide]{Beach morpho-sedimentary profile evolution, recorded each low tide between tides 14 and 27 on the cross-shore survey transect (CT). (a) Cumulative change in bed elevation from RTK GPS observations. (b) Mean grain size from photographs of the bed surface. Black triangles represent the estimated cross-shore coordinate of the high water line.}
 	\label{fig:alltides_profiles}
\end{figure}

The beach surface sediments were generally coarser near, and landward of, the high tide shoreline, the coarsest sediments corresponding to the nominal position of the active berm (Fig. \ref{fig:mean_MGS_profile}). On average, the mean grain sizes were finer in the intertidal zone seaward of the berm region. Note that the extent of the cross-shore profiles included in Figs. \ref{fig:alltides_profiles} and \ref{fig:mean_MGS_profile} represents only approximately one third of the full intertidal beach profile. In general, the grain size distribution became coarser near the low water shoreline, being characterised by a mixture of coarse sand and cobble- and boulder-sized material (not represented in the included data). 

\begin{figure}[tbp] %/home/tristan/Documents/Projects/AdvocateBeach2018/src/visualization/plot_beach_profile_data.py
	\begin{center}
  	\includegraphics[width=0.5\columnwidth]{figures/chapter3/mean_MGS_crossshore.png}
 	\caption[Cross-shore profile of mean grain size]{Profile of mean grain size, computed at each station in the cross-shore survey transect (CT) averaged over tides 14-27. The horizontal black lines indicate the standard deviation of the mean grain size over time, and the vertical lines indicate the minimum and maximum mean grain size at each cross-shore station. The grey-shaded region encompasses the range of high water line values during the experiment. The cross-shore coordinates of the two longshore survey transects, LT1 and LT2, are indicated by the dashed and dash-dotted lines, respectively. \label{fig:mean_MGS_profile}}
 	\end{center}
\end{figure}
%[If I'm going  to include a volume change plot, I should probably group it in with this. Also, I should make the max/min envelope grey]. 

Fig. \ref{fig:alltides_profiles} shows the change in bed elevation and mean grain size for the cross-shore survey profile over all the tides that were surveyed consecutively (tides 14-27). The shoreward migration of the high tide berm is visible, and shows the greatest morphological relief nearer the end of the experiment. Coarser grain sizes correspond to the landward extent of the high tide berm, which also generally corresponded to the HWL. The distribution of coarse and fine surficial sediments in the mid-intertidal zone displayed some spatial structure, though this structure did not persist from tide to tide. In the upper intertidal zone, within \textit{ca}. 10 m seaward of the HWL, coarse and fine sediment `patches' occasionally appeared to persist for longer (i.e., multiple tides). Unlike the previous experiments at Advocate Beach, there were no instances of well-defined cusp formation during the 2018 field experiment. 

A wide variety of features were observed within the mid-intertidal zone. In general, the features had low morphological relief (\textit{ca}. 5 cm maximum), and were more identifiable as coherent but disorganised sedimentary structures, which manifested as irregular alternating bands of coarse- and fine-grained sediments along- or across-shore, grain size `patchiness', or larger undulations in morphology and grain size along- or across-shore. Some examples of these features are shown in Fig. \ref{fig:patchiness_photo3}.

\begin{figure}[tbp] %/home/tristan/Documents/Projects/AdvocateBeach2018/src/visualization/plot_beach_profile_data.py
	\includegraphics[width=\columnwidth]{figures/chapter3/photos/collage6.png}
	\caption[Photographs of grain size segregation at Advocate Beach]{Photographs of features exhibiting grain size segregation at Advocate Beach during low tide 11 (a,b,c), 16 (d), 17 (e), and 19 (f).}
	\label{fig:patchiness_photo3}
\end{figure}



\subsection{Correlation Results}

%\begin{figure}[tbp] %/home/tristan/Documents/Projects/AdvocateBeach2018/src/visualization/plot_beach_profile_data.py
%	\includegraphics[width=\columnwidth]{figures/chapter3/photos/coarse_fine_photos_vertical2.png}
%	\caption{Photographs of Advocate Beach following periods of energetic wave forcing (top, tide 13) and low-energy forcing (bottom, tide 27).}
%	\label{fig:coarse_fine_photos}
%\end{figure}

The mean grain size data from the mid-intertidal zone beach face are strongly anti-correlated with significant wave height, with wide-scale beach fining being observed following energetic wave conditions. Time series of the mean significant wave height and spatially-averaged mean grain size, $\overline{\mathrm{MGS}}$, for each tide are shown in Fig. \ref{fig:hsig_mgs} using grain size values from the most seaward longshore survey transect, LT1. The Pearson correlation coefficients computed between the mean grain size and mean wave statistics are listed in Table \ref{table:temporal_correlations}. The spatially averaged mean grain sizes are highly correlated with both the significant wave height ($\hat{r}=-0.79$) and the deep water wave steepness ($\hat{r}=-0.84$). Scatter plots of the spatially-averaged mean grain size versus the mean significant wave height and mean wave steepness, shown in Fig. \ref{fig:hsig_steepness_mgs_scatter}, indicate that both relationships are well-described as linear under low to moderate energy forcing conditions (e.g., $H_s < 0.5$ m, $H_0/L_0 < 0.017$), but with an apparent threshold forcing value above which the spatially averaged mean grain size exhibits greater variability. The values of the correlation coefficients for both $\overline{\mathrm{MGS}}$ and mean significant wave height, and $\overline{\mathrm{MGS}}$ and mean wave steepness are not significant at the $\alpha=0.05$ level if computed using data from the more shoreward survey transect, LT2.

\begin{table}[tbp!]
	\caption[Temporal correlation coefficients: mean grain size and wave parameters]{Temporal correlation coefficients, $\hat{r}$, between spatial averages of mean grain size along the indicated transect, $\overline{\mathrm{MGS}}$, and wave forcing parameters.\label{table:temporal_correlations}} 
	\centering
	\begin{tabular}{lccc}
		\hline
		correlates & survey transect & $\hat{r}$ & 95\% confidence interval\\
		\hline
		$\overline{\mathrm{MGS}}$, $4\sigma_p$ & LT1 & -0.79$^{*}$ & (-0.39, -0.94)\\
		$\overline{\mathrm{MGS}}$, $T_p$ & LT1 & 0.04 & (-0.55, 0.60)\\
		$\overline{\mathrm{MGS}}$, $H_0/L_0$ & LT1 & -0.84$^{*}$ & (-0.51, -0.95)\\
		$\overline{\mathrm{MGS}}$, $4\sigma_p$ & LT2 & -0.24 & (-0.70, 0.36)\\
		$\overline{\mathrm{MGS}}$, $T_p$ & LT2 & 0.08 & (-0.60, 0.49)\\
		$\overline{\mathrm{MGS}}$, $H_0/L_0$ & LT2 & -0.24 & (-0.70, 0.36)\\
		\hline
		\multicolumn{4}{l}{$^{*}$ Statistically significant at the $\alpha=0.05$ level.}
	\end{tabular}
\end{table}

\begin{figure}[tbp] %/home/tristan/Documents/Projects/AdvocateBeach2018/src/visualization/plot_beach_profile_data.py
	\begin{center}
	\includegraphics[width=0.75\columnwidth]{figures/chapter3/grain_size_and_waveheight_timeseries.png}
	\caption[Time series of mean significant wave height and mean surficial grian size]{Time series of mean significant wave height and $\overline{\mathrm{MGS}}$ for transect LT1.\label{fig:hsig_mgs}}
	\end{center}
\end{figure}

\begin{figure}[tbp] %/home/tristan/Documents/Projects/AdvocateBeach2018/src/visualization/plot_beach_profile_data.py
	\begin{center}
		\includegraphics[width=0.7\columnwidth]{figures/chapter3/Hsig_and_steepness_vs_MGS_scatter.png}
		\caption[Mean surfical grain size versus significant wave height, wave steepness]{Scatter plots of the spatially averaged mean grain size $\overline{\mathrm{MGS}}$, computed over the seaward-most longshore survey transect LT1 for each of the surveyed tides, versus (a) the mean significant wave height $H_s$ and (b) the deep water wave steepness $H_0/L_0$. \label{fig:hsig_steepness_mgs_scatter}}
	\end{center}
\end{figure}

Fig. \ref{fig:corr_coeffs_spatial} shows scatter plots of all bed level change, $\Delta z$, and mean grain size change, $\Delta$MGS, observations between tides 14 and 27, for LT1, the most seaward longshore transect, and LT2, the longshore transect roughly corresponding to the mean position the high tide berm. A positive correlation is apparent in the LT1 data ($r=0.37$). Little visually discernible correlation is evident from the LT2 data ($r=0.16$), though the correlation increases significantly if only examining tides characterised by high steepness waves. A summary of the spatial correlations between changes in bed level and mean grain size is presented in Table \ref{table:spatial_correlations}. The correlation output for the LT2 data is nearly identical if stations landward of the HWL are omitted from the analysis (true only for tide 16; see Fig. \ref{fig:alltides_profiles}).

\begin{table}[tbp!]
	\caption[Spatial correlations between changes in bed level and changes in mean grain size]{Spatial correlations, $r$, between changes in bed level, $\Delta z$, and changes in mean grain size, $\Delta$MGS.} 
	\label{table:spatial_correlations}
	\centering
	\begin{tabular}{lcc}
		\hline
		survey transect & $r$ & 95\% confidence interval\\
		\hline
		LT1 & 0.37$^{*}$ & (0.27, 0.47)\\
		LT1 ($H_0/L_0 < 0.01$) & 0.36$^{*}$ & (0.22, 0.49)\\
		LT1 ($H_0/L_0 >= 0.01$) & 0.39$^{*}$ & (0.25, 0.52)\\
		LT2 & 0.16$^{*}$ & (0.05, 0.26)\\
		LT2 ($H_0/L_0 < 0.01$) & 0.06 & (-0.08, 0.21)\\
		LT2 ($H_0/L_0 >= 0.01$) & 0.30$^{*}$ & (0.15, 0.44)\\
		\hline
		\multicolumn{3}{l}{$^{*}$ Statistically significant at the $\alpha=0.05$ level.}
%		\footnote{$^{*}$ Statistically significant at the 95\% level.}
	\end{tabular}
\end{table}

A summary of the correlation coefficients between $\Delta z$ and $\Delta$MGS for each of the longshore transects is shown in Fig. \ref{fig:corr_coeffs_spatial}. The correlation values increase positively with increasing distance to seaward. The strongest correlations are associated with LT1 -- the seaward-most longshore transect. The correlation coefficient values are not significantly different from zero for transects located landward of the nominal HWL.

Correlation coefficients associated with data from individual tides are shown in Fig. \ref{fig:corr_coeffs_all}, for LT1 and LT2. Since these correlations were limited by the number of samples in the survey transect, few of the results are statistically significant at the $\alpha=0.05$ level; however, there are some notably consistent trends. Similar to the correlation results above, higher correlations are evident for data from the seaward-most survey transect. From the LT1 data, the correlation coefficients between $\Delta z$ and MGS, and between $\Delta z$ and $\Delta$MGS were consistently positive, having centroid values that are commensurate with the longer-term correlation results (Fig. \ref{fig:corr_coeffs_spatial}), which have higher degrees of statistical significance. With two exceptions, the correlation coefficients between $\Delta z$ and the mean grain size observed at the previous low tide, MGS$'$, were negative, having a centroid value of $r\approx -0.2$. The correlation of $\Delta z$ with $\Delta z$ from the previous low tide, $\Delta z'$, was consistently negative, with a centroid value of $r\approx -0.5$. The centroid values of the correlation coefficients associated with LT2 exhibit similar trends, though are closer to zero and have a higher degree of variation. 

With the exception of the strong anticorrelation between $\overline{\mathrm{MGS}}$ and the significant wave height, comparisons between $\overline{\Delta z}$, $\overline{\mathrm{MGS}}$, and mean wave forcing parameters (wave height, peak period, sea-band wave energy, and steepness) did not yield any statistically significant temporal correlations or clear visual trends. The potential influence of the wave forcing on the spatial correlation between $\Delta z$ and $\Delta$MGS (Fig. \ref{fig:corr_coeffs_spatial}) was investigated using data from LT1 by separating tides characterised by high wave steepness forcing, $H_0/L_0 > 0.01$, from those characterised by low steepness forcing, $H_0/L_0 < 0.01$, and carrying out the correlation computations on each. A steepness value of 0.01 is often used to demarcate accretive from erosive conditions \citep[e.g.,][]{Masselink_etal2007}. The correlation coefficients are nearly identical for both datasets: $r=0.39$ with a 95\% confidence interval of (0.25, 0.52) for the high steepness set, and $r=0.36$ in (0.22, 0.49) for the low steepness set. The result is comparable when a significant wave height threshold is used instead (threshold values of $H_{s} = 0.5$ m and $0.2$ m were tried).

%The correlation values between the surveyed morphology and sedimentology varied considerably on a tide-to-tide basis. No relationship could be identified between the degree of morpho-sedimentary correlation and hydrodynamic factors, including significant wave height, period, total wave energy, steepness, or Iribarren number. Similarly, the degree of correlation did not appear to co-vary with the MHW level, nor with the mean grain size over the length of the transect. See section \ref{section:discussion} for further discussion. 

The data from LT1 were also analyzed following the method of \citet{Masselink_etal2007} to facilitate comparison with their results. Bed level change ($\Delta z$) at each survey station was classed as accretion ($\Delta z \geq 0.02$ m), no change ($-0.02 < \Delta z < 0.02$ m), or erosion ($\Delta z \leq -0.02$ m). The mean and standard deviation of the MGS associated with the $\Delta z$ instances were computed for each of the three categories. The results are listed in Table \ref{table:dz_mgs}. The trend of change in MGS was examined by tabulating the proportion of fining ($\Delta$MGS $< 0$) versus coarsening ($\Delta$MGS $> 0$) instances within the accretion and erosion categories. As summarised in Table \ref{table:dz_dmgs}, surficial sediment coarsening occurred during 86\% of the instances of accretion, and fining occurred during 56\% of the erosion instances. %These results, and those of \citet{Masselink_etal2007}, are discussed further in section \ref{section:discussion}. 

\begin{figure}[tbp] %/home/tristan/Documents/Projects/AdvocateBeach2018/src/visualization/plot_beach_profile_data.py
	\begin{center}
		%	\includegraphics[width=0.75\columnwidth]{/home/tristan2/Documents/PhD/Dissertation/figures/chapter3/dz_dmgs_scatter.png}
		\includegraphics[width=0.8\columnwidth]{figures/chapter3/combined_scatter_vertical.png}
		\caption[Cross-shore dependence of correlations between bed level and mean grain size change]{Scatter plots of the change in bed level ($\Delta z$) versus the change in mean grain size ($\Delta$ MGS) using data from: (a) longshore survey transect LT2, corresponding roughly to the time-averaged cross-shore location of the high tide berm ($r=0.17$, with a 95\% confidence interval of (0.06, 0.27)); and (b) longshore survey transect LT1, the most seaward survey transect, positioned down-slope of the nominal berm location, in the mid-intertidal zone ($r=0.38$, 95\% confidence interval of (0.28, 0.47)). The data from all consecutively surveyed low tides are shown, with the mean subtracted from both $\Delta z$ and $\Delta$MGS for each tide time step. (c) Pearson correlation coefficients for $\Delta z$ and $\Delta$MGS for all of the longshore survey transects, i.e., LT1 (yellow), LT2 (blue), and each of the six longshore transects comprising the densely surveyed grid, DG (black). The correlations are based on all samples collected on each of the eight longshore transects throughout the experiment, with tide-specific means removed. The tails indicate 95\% confidence intervals. The horizontal black line, dark grey-shaded region, and light grey-shaded region indicate the respective mean, standard deviation, and range of the cross-shore coordinate of the high water line over the 13 tides. Note that the $y=-13$ m coordinate was sampled twice: i.e., both LT2 and the third-most shoreward transect of DG.}
		\label{fig:corr_coeffs_spatial}
	\end{center}
\end{figure}

%\begin{figure}[tbp] %/home/tristan/Documents/Projects/AdvocateBeach2018/src/visualization/plot_beach_profile_data.py
%	\begin{center}
%%	\includegraphics[width=0.75\columnwidth]{/home/tristan2/Documents/PhD/Dissertation/figures/chapter3/dz_dmgs_scatter.png}
%	\includegraphics[width=\columnwidth]{figures/chapter3/dz_dmgs_scatter.png}
%	\caption{Scatter plots of the change in bed level ($\Delta z$) versus the change in mean grain size ($\Delta$ MGS) using data from (a) longshore survey transect LT2, corresponding roughly to the time-averaged cross-shore location of the high tide berm ($r=0.17$, with a 95\% confidence interval of (0.06, 0.27)); and (b) longshore survey transect LT1, the most seaward survey transect, positioned down-slope of the nominal berm location, in the mid-intertidal zone ($r=0.38$, 95\% confidence interval of (0.28, 0.47)). The data from all consecutively surveyed low tides are shown, with mean subtracted from both $\Delta z$ and $\Delta$ MGS for each tide time step.}
%	\label{fig:corr_coeffs_spatial}
%	\end{center}
%\end{figure}
%
%\begin{figure}[tbp] %/home/tristan/Documents/Projects/AdvocateBeach2018/src/visualization/plot_beach_profile_data.py
%	\begin{center}
%		\includegraphics[width=0.5\columnwidth]{figures/chapter3/dz_dmgs_correlations_vs_y.png}
%		\caption{Pearson correlation coefficients for $\Delta z$ and $\Delta$ MGS for all of the longshore survey transects, i.e. longshore transects 1 and 2 (LT1 and LT2), and each of the six longshore transects comprising the densely surveyed grid (DG2). The correlation included all samples collected on each of the eight longshore transects thoughout the experiment, with tide-specific means reomved. The black lines indicate the 95\% confidence interval. The grey-shaded region indicates the range of positions of the high water line throughout the experiment. Note that the $y=-13$ m coordinate was sampled twice -- corresponding to LT2 and the third most shoreward transect of DG2.}
%		\label{fig:corr_coeffs_vs_y}
%	\end{center}
%\end{figure}

\begin{figure}[tbp] %/home/tristan/Documents/Projects/AdvocateBeach2018/src/visualization/plot_beach_profile_data.py
	\begin{center}
	\includegraphics[width=\columnwidth]{figures/chapter3/correlation_spatialonly.png}
	\caption[Spatial correlation coefficients]{Pearson correlation coefficients, $r$, associated with longshore survey transects (a) LT2, and (b) LT1. The correlations are between bed elevation change ($\Delta z$) and: mean grain size (MGS), the change in mean grain size ($\Delta$MGS), the mean grain size observed during the previous survey (MGS$'$), and the change in bed elevation observed during the previous survey ($\Delta z'$). Each `x' represents the correlation coefficient for data from the given survey transect for a single tide.}
	\label{fig:corr_coeffs_all}
	\end{center}
\end{figure}

\begin{table}[tbp!]
	\caption[Mean grain size correspoonding to bed accretion, no change, and erosion]{Average mean grain size (MGS) for three different classes of bed-level change: accretion, no change, and erosion associated with the most seaward longshore survey transect, LT1. The number of occurrences for each class, $N$, is also indicated.} 
	\label{table:dz_mgs}
	\centering
	\begin{tabular}{ccc}
		\hline
		bed state & mean size (mm) & $N$\\
		\hline
		%		accretion & 15.99 $\pm$ 8.29 & 151\\
		%		no change & 15.71 $\pm$ 7.55 & 292\\
		%		erosion   & 12.52 $\pm$ 7.31 & 210\\
		accretion & 25.7 $\pm$ 6.7 & 57\\
		no change & 20.7 $\pm$ 7.5 & 165\\
		erosion   & 19.6 $\pm$ 7.5 & 90\\
		\hline
	\end{tabular}
\end{table}

\begin{table}[tbp!]
	\caption[Occurrence counts of bed fining and coarsening during bed accretion and erosion]{Occurrence counts and occurrence ratios, in parentheses, between morphological response and change in mean sediment size for the accretion and erosion classes of bed level change associated with the most seaward longshore survey transect, LT1.} 
	\label{table:dz_dmgs}
	\centering
	\begin{tabular}{cccc}
		\hline
		bed state & finer & coarser & $N$\\
		\hline
		accretion & 8 (0.14) & 49 (0.86) & 57\\
		erosion & 50 (0.56) & 40 (0.44) & 90\\
		\hline
	\end{tabular}
\end{table}


\section{Discussion}\label{section:ch3_discussion}

%[If I need to discuss higher correlations associated with high wave energy, can discuss decoupling of the lower and upper swash zone, which also may provide a mechanism for beach surface fining during high energy events: the larger settling lag associated with the finer sand particles causes them to be deposited later. The decoupling of the swash manifests as a demarcation between regions characterized by transport of all kinds (bed load, suspended load, saltation), and transport of finer material (suspended load - upper swash).]

The morpho-sedimentary evolution of a mixed sand-gravel beach was investigated over 14 consecutive tidal cycles at the scale of the intertidal zone ($O$(10-100) m), through bed level and photographic grain size sampling over a fixed survey grid during low tide when the intertidal zone was exposed. The forcing conditions were dominated by steep, wind-band incident waves leading to an energetic shore break, interspersed by periods of low steepness wave incidence and berm building (Figs. \ref{fig:wavestats} and \ref{fig:alltides_profiles}). 

\subsection{Response of Surficial Sediments to Varying Wave Forcing}

The time series of spatially-averaged surficial sediment grain size in the mid-intertidal zone indicates a clear dependence on the significant wave height and wave steepness (Fig. \ref{fig:hsig_mgs}). High wave energy conditions resulted in surficial sediment fining across the intertidal zone, and low energy conditions resulted in bed surface coarsening. This result is counter-intuitive in the context of our broader understanding of (especially sand) beach response to changes in forcing \citep[e.g.,][]{Bascom1951, Masselink_etal2007}. However, this response has been previously observed at Advocate Beach \citep{Hay_etal2014}, as well as at other MSG sites reported on in the literature \citep{Nordstrom_Jackson1993, Pontee_etal2004, Curtiss_etal2009, Miller_etal2011}. Typical forcing conditions and beach characteristics for each of the sites reported in the literature are summarised in Table \ref{table:beach_summary}. Like Advocate, all of these beaches are steep, with slopes of at least 1:10.

%The beach surface grain size in the mid-intertidal zone displayed a clear dependence on the incident wave forcing, parameterized by the significant wave height (Fig. \ref{fig:hsig_mgs}). Larger wave heights resulted in surficial sediment fining across the intertidal zone, and low energy conditions resulted in bed surface coarsening. This result is counter-intuitive in the context of our broader understanding of (especially sand) beach response to changes in forcing \citep[e.g.][]{Bascom1951, Masselink_etal2007}. However, this response has been previously observed at Advocate Beach \citep{Hay_etal2014}, as well as at other MSG sites reported on in the literature \citep{Nordstrom_Jackson1993, Pontee_etal2004, Curtiss_etal2009, Miller_etal2011}. Typical forcing conditions and beach characteristics for each of the sites reported in the literature are as follows: The study by \citet{Nordstrom_Jackson1993} took place on a relatively steep (6.2 degree foreshore slope) MSG beach with flat, low tide terrace and a mean tide range of 1.6 m. The beach is located in Delaware Bay, USA, the dominant wave forcing thus being locally generated wind waves. \citet{Pontee_etal2004} reported on three MSG beaches on the East Anglian coast of the UK. All three beaches experience average tide ranges of 2-3 m, with fetch being limited by a series of sand banks off the coast. The mean annual wave height and period are 0.4-0.5 m and approximately 6 s, respectively. \citet{Curtiss_etal2009} made observations on a steep (1:5-1:7 slope) MSG beach on the shore of a narrow channel in Puget Sound, WA, USA, exposed to passing vessel wakes, wind waves, tides, and tidal currents. The mean tide range is 3.5 m. \citet{Miller_etal2011} reported on a MSG beach near the Elwha River delta on the Juan de Fuca Strait, WA, USA. The site experienced bimodal wave forcing, comprised of wind waves generated within the strait, and attenuated oceanic swell propagating eastward through the strait, having peak periods of 4 and 10 s, respectively. They observed a mean significant wave height of 0.47 m, and a mean tide range of 1.4 m. 

\begin{landscape}
%\rotatebox{90}{
%\centering
\begin{table}[tbp!]
	\caption[Mixed sand-gravel beaches: summary of reported beach surface fining during storms]{Summary of studies from the mixed sand-gravel beach literature reporting fining of beach surface sediments during periods of high energy wave forcing.\label{table:beach_summary}} 
	\centering
	\begin{tabular}{llccl}
		\hline
		study & location & tide range & slope ($\tan\beta_s$) & wave height; period\\
		\hline
		\citet{Nordstrom_Jackson1993} & Delaware Bay, USA & 1.6 m & 0.11 & $>0.5$ m (high energy); NA\\
		\citet{Pontee_etal2004} & 3 sites in East Anglia, UK & 2-3 m & 0.10-0.15 & 0.4-0.5 m (annual); \textit{ca}. 6 s\\
		\citet{Curtiss_etal2009} & Puget Sound, WA, USA & 3.5 m & 0.14-0.20 & 0.10-0.24 m (during 4 storms); 1-3 s\\
		\citet{Miller_etal2011} & Juan de Fuca Strait, WA, USA & 1.4 m & 0.12-0.13 & 0.47 m (6 month); 4 s (wind)\\%; 10 s (swell)\\
		\citet{Hay_etal2014} & Advocate Beach, NS, CA & 10 m & 0.12 & 0.3-0.5 m; 4-6 s\\
		\hline
	\end{tabular}
\end{table}
\end{landscape}
%}

The relationship between wave forcing and mean surface grain sizes is far from clear. For example, though \citet{Pontee_etal2004} observed an inverse correlation between mean grain size and wave height in general, they only claimed statistical significance of the result at one of their three study sites, and in some cases observed an opposite response. It is interesting to note that all of the sites described above experience some degree of wave energy limitation; the sites studied by \citeauthor{Pontee_etal2004} are the only ones which face the open ocean, though the wave energy is limited by a series of offshore sand bars. The sites listed in Table \ref{table:beach_summary} span a wide range of tide regimes, from micro- to megatidal. The similarities and differences between the sites may provide some insight into the mechanisms for fining during more energetic wave forcing, about which a range of explanations have been proposed: \citeauthor{Pontee_etal2004} suggested that onshore transport of sand from the nearshore region during periods of post-storm swell could explain the sandy characteristics of post-storm profiles at two of their beaches. They also suggested that intertidal fining could be a result of larger particles being concentrated in the berm and step. \citeauthor{Curtiss_etal2009} noted that the observed surficial fining during storm conditions is consistent with a `reverse winnowing' process, wherein larger particles are selectively removed from the bed as a result of enhanced mobility. \citeauthor{Miller_etal2011} noted that the increase in sand visible at the bed surface following high waves was suggestive of thorough mixing of the beach substrate. At Advocate Beach, during highly energetic wave conditions and on the rising tide, \citeauthor{Hay_etal2014} observed the formation of metre-scale wave orbital ripples with ripple crests built of fine sediments and ripple troughs containing a coarse sediment lag. These ripples were not present on the exposed beach face after the subsequent ebb tide, the beach surface instead being both planar and covered with a thin layer of fine material. \citeauthor{Hay_etal2014} attributed this fining of the beach face sediment to the ripple crests being planed off by the swash during the falling tide, burying the coarser-grained material in the troughs.    

Past observations of the grain size distribution at Advocate Beach have indicated that the sediments beneath the surficial layer remain well-mixed, regardless of the forcing conditions, suggesting that the surf and swash zones can be approximately considered as a closed sedimentary unit. Thus, the mechanisms suggested by \citeauthor{Pontee_etal2004}, and \citeauthor{Curtiss_etal2009}, involving removal of sediments from the beach face to the nearshore, or vice versa, do not seem likely at the Advocate site. The factor of 10 differences among the tidal ranges listed in Table \ref{table:beach_summary} suggests, given the factor of only 2 differences in slope, that cross-shore translation of the shoreline may not be a requirement for beach surface fining in response to energetic waves. The implication is that beach surface fining in response to energetic waves is primarily a swash zone process. The process may be influenced by, while not being dependent upon, morphodynamic processes in the surf zone, as required for the \citeauthor{Hay_etal2014} ripple planing mechanism. 

The following alternative mechanism is proposed (see the accompanying schematic in Fig. \ref{fig:storm_fining}): During periods of high energy wave forcing, the swash zone increases from its characteristic width of $O$(1 m) during fairweather forcing to a width of $O$(10 m). Mobilisation of all grain sizes occurs following bore collapse at the shorebreak; gravel- and cobble-sized particles, which are typically transported through rolling, sliding, or occasional saltation \citep{Carter_Orford1984}, are deposited earlier in the uprush phase, making them susceptible to remobilisation during the downrush phase. The sand size-fraction is transported shoreward as suspended load. The lower settling velocity of sand-sized particles in suspension contributes to a larger settling lag -- the time required for suspended particles to settle to the bottom through slowly flowing water \citep{Masselink_Puleo2006} -- meaning the finer particles are deposited later in the swash uprush phase. Increased infiltration near the landward edge of the swash zone, due to the swash surpassing the water table exit point, contributes to the deposition of any material still in saltation or suspension. The result is the deposition of a layer of finer-grained material near the shoreward-edge of the swash zone, atop the more well-mixed substrate. During fairweather forcing, the shorebreak and swash zone are more closely coupled. Mobilisation of most grain sizes occurs following bore collapse, with gravel- and cobble-sized particles being transported primarily through rolling. The coarse-grained particles may remain mobile due to rejection by the finer substrate leading to overpassing, or may be deposited near the landward- or shoreward edge of the swash zone. The sand-sized fraction may be transported via saltation and some suspended load. Unlike in the energetic forcing case, the characteristic transport range of the coarser particles is of the same order as the swash zone width, meaning the settling leg mechanism for surficial fining is not effective (note Fig. \ref{fig:cusp_bay_overwash} in Chapter \ref{Chapter:CuspDynamics} as an exception to this, where fine-grained material was deposited landward of cusp bays during fairweather forcing). The tendency for gravel- and cobble-sized particles to propagate shoreward due to swash velocity asymmetry has been acknowledged elsewhere \citep{Carr1983}. The sequences described above are independent of shoreline translation induced by tides. However, tides may serve to visually reinforce the signatures of the mechanisms described above on the beach face, since the beach surface coarsening or fining associated with the leading edge of the swash is spread over the subaerial intertidal zone following the translation of the shoreline during ebb tide. Furthermore, the presence of a nascent coarse-grained beach step during fairweather forcing, or metre-scale wave orbital ripples with fine-grained crests during energetic forcing \citep[see][]{Hay_etal2014}, may provide source regions of coarse-grained and fine-grained particles, respectively, during ebb tide. Note also that the surficial fining and coarsening mechanisms proposed above are a result of local processes, i.e., they do not require an exchange of material between the surf- and swash zones.

\begin{figure}[tbp] %/home/tristan/Documents/Projects/AdvocateBeach2018/src/visualization/plot_beach_profile_data.py
	\begin{center}
		\includegraphics[width=\columnwidth]{figures/chapter3/storm_fining_berm.png}
		\caption[Schematic description of surficial sediment fining mechanisms during energetic wave forcing]{Schematic description of the proposed mechanisms leading to surficial sediment fining during energetic wave forcing (left) and surficial sediment coarsening during fairweather wave forcing (right).}
		\label{fig:storm_fining}
	\end{center}
\end{figure}

The proposed surficial fining mechanism is similar to the beach recovery mechanism described by \citet{Bramato_etal2012} on a MSG beach following a storm-associated erosion event, in which, during periods of accumulation, the longer suspension time of the finer sediments causes them to be deposited in a layer atop the coarser sediments. However, their mechanism is associated with fairweather forcing, offshore transport of sand and beach surficial coarsening being associated with erosive storm conditions.

The data presented in Fig. \ref{fig:hsig_steepness_mgs_scatter} are suggestive of a threshold set of forcing conditions, below which the relationships between the spatially averaged mean grain size and both the significant wave height and the wave steepness appear linear, and above which the mean grain size displays a higher degree of variability. Two potential explanations are suggested: (1) The stranding of coarse-grained material at the leading edge of the swash zone, i.e., the proposed surficial sediment coarsening mechanism associated with fairweather forcing, is a more effective and consistent grain size sorting mechanism than the surficial fining process associated with energetic forcing. In other words, the continued presence of coarse-grained particles on the beach surface following energetic forcing conditions, despite the fining of the beach surface sediments in a mean sense, results in a broader surficial grain size distribution following energetic forcing that is more variable in space for a given tide, and consequently also exhibits more variation among different tides. (2) The low variability of spatially averaged mean grain size estimates following fairweather forcing and the higher variability associated with energetic forcing is an artefact of the digital grain sizing methodology. The validated best-fit parameter set used in the digital grain sizing algorithm underestimates mean grain sizes above \textit{ca.} 20 mm (see Appendix \ref{Appendix2}). This may result in an effective `ceiling' on the mean grain size estimates causing the larger grain size estimates to appear more closely grouped than the finer mean grain size estimates.

There may be other site-specific mechanisms which depend upon a complicated time history of antecedent morphology, grain size, and forcing. It would be of interest to carry out a full sedimentological investigation at a MSG site that includes high resolution sampling in space (vertical and horizontal), at tidal time scales, in order to further constrain the potential mechanisms.


% correlation results:
\subsection{Morpho-Sedimentary Correlation at the Beach Scale}\label{subsection:ChMSDBeach_corrDiscussion}

The results from Section \ref{subsection:beachMSD} clearly indicate that a spatial correlation exists between changes in bed level and mean surface grain size. Though the existence of this correlation is perhaps unsurprising given the acknowledged influence of bed texture on near-bed hydrodynamics and the prevalence of grain size sorting and pattern formation on MSG beaches, it has been difficult to establish in other cases. Indeed, similar studies by \citet{Masselink_etal2007} and \citet{Austin_Buscombe2008} found little evidence of correlation between morphological and grain size responses on coarse sand and gravel beaches, respectively. \citeauthor{Masselink_etal2007} sampled bed elevation and grain size over a \textit{ca}. 100 m cross-shore transect, over 40 tidal cycles. In most cases, they reported no correlation between bed level change and mean grain size or sorting, with the exceptions of coarse sediment patches near the beginning and end of their study which were associated with net erosion (i.e., correlations with opposite sign to those presented here) and decreases in sorting. In their gravel beach study, \citet{Austin_Buscombe2008} collected coincident observations of bed level, using manual surveying, and grain size from imaging methods. They observed some positive correlation between bed level change and mean grain size at the position of the berm and beach step, though a clear relationship was not otherwise observed. 

A notable difference between Advocate Beach and the sites investigated by \citet{Masselink_etal2007} and \citet{Austin_Buscombe2008} is the much wider grain size distribution at Advocate: i.e., from less than 1 mm to greater than 200 mm at Advocate compared to coarse sand, $0.5 <$ MGS $< 1.0$ mm; and pure gravel, $5 <$ MGS $< 20$ mm. The presence of combined sand and gravel fractions has dynamical implications for sediment transport \citep[e.g.,][]{Wilcock_McArdell1993, Wilcock_McArdell1997, Wilcock_Crowe2003}, and may enhance morpho-sedimentary feedbacks leading to pronounced size-segregated pattern formation \citep[e.g.,][]{LonguetHiggins_Parkin1962, Guest_Hay2019}. 

%%[this is redundant]
%An effort was made in this work to present an analysis that can be easily compared with others reported in the literature. \citet{Masselink_etal2007} tested for correlations between bed level change and grain size properties by separating instances of bed level change into `accretion', `erosion', and `no change' bins ($|\Delta z|<0.02$ m), mean grain size and sorting statistics then being computed within each. Trends of mean grain size and sorting change were examined by binarizing the grain size measures within each of the bed-level change bins to positive ($\Delta$MGS $> 0$, coarsening) or negative ($\Delta$MGS $< 0$, fining). Their results showed no difference between grain size characteristics in the three bed level change bins, and instances of coarsening or fining (and better or poorer sorting) that were not significantly different from 50\% in either of the accretion and erosion bins. Though we were not able to quantify higher order moments of the grain size distribution (i.e. sorting), the same analysis applied to mean grain size shows moderate differences between the accretion ($24.0 \pm 6.7$ mm) and erosion ($17.5 \pm 7.4$ mm) bins. Instances of accretion were coupled with bed coarsening in 72\% of cases, and erosion coupled with fining in 63\% of cases.

% The observed positive correlations between bed level change and both mean grain size and the trend of grain size change, and the negative correlations between bed level change and both mean grain size ...

%[mechanistic differences between temporal and spatial variation]

The correlation coefficients are highest for data sampled farthest to seaward of the high water line. There are two reasons why this might have been the case. The first is that five of the eight longshore survey transects were above the HWL for at least one survey. Omitting above-HWL data does little to change the correlation results, indicating that sediments at or near the HWL (interpreted as the swash runup maximum) were not acted upon by sufficient swash energy for a signal to emerge. This is supported by the notable increase in the correlation coefficient between $\Delta z$ and $\Delta$MGS at the more shoreward LT2 transect during high wave steepness forcing, when the swash zone width would have been greater. A different shoreline metric, e.g., the mean swash runup limit, might therefore be a more appropriate boundary. The second reason is that the beach surface near the nominal high water shoreline was typically characterised by the coarse sediments -- gravel and cobbles -- associated with the berm region. This coarse material near the HWL persisted through time (see Fig. \ref{fig:alltides_profiles}), migrating with the high tide shoreline during the neap-spring tide cycle. Conversely, the beach surface in the mid-intertidal zone, seaward of the berm region, was typically finer-grained with a greater degree of temporal variability. It seems likely that the inherent short-term variability of the bed level and mean grain size in the mid-intertidal zone, along with the presence of a more well-mixed surficial grain size distribution, was more conducive to the emergence of an observable signal.

%\subsection{Feedback Mechanisms}
%
%The correlations between bed level change and mean grain size at the beach surface (positive between $\Delta z$ and MGS, $\Delta$MGS; negative between $\Delta z$ and MGS$'$, $\Delta z'$) can be explained in terms of mutually reinforcing feedbacks between flow and morpho-sedimentology, and the diffusive influence of the horizontally translating shorebreak. In the ebb-tide swash zone, emerging coarse patches enhance deposition: of fine material through kinetic sieving (effectively removing the fine particles from the beach surface), and of coarse material through grain interlocking and increased angles of pivot required for mobilization. Away from the berm region, where sediments tended to remain coarse, it is logical that where grain sizes were large, there was an increase in mean grain size (e.g., from the time-averaged mean). Conversely, coarse-grained topographic highs or fine-grained lows which formed on the mid-intertidal beach face during the previous ebb tide were subject to the diffusive influence of the translating surf zone and shorebreak. The diffusive influence of the surf zone on transient morphological beach face features has been previously observed in relation to other morphological features \citep[e.g. beach cusps][]{Coco_etal2004}. The morphological smoothing implied by the negative correlation between changes in bed elevation from one tide to the next is unsurprising (consider, for instance, footprints in intertidal sand). More interesting is the persistent co-variation of mean grain size and bed elevation.
%
%Stations surveyed near the HWL may be associated with different morpho-sedimentary processes (e.g., berm, cusp formation) than the mid-tide beach face, and may not be acted upon by the diffusive influence of the translating shorebreak. At the high tide shoreline, the mechanism for mobilizing and transporting accumulations of coarse grains (i.e., swash) is weakened through hydraulic drag and infiltration, increasing the likelihood that a morphological feature would persist through successive tides.
%
%The implication of the correlation results is that mutually supporting feedbacks between morphology, sedimentology, and hydrodynamics/hydraulics are responses to swash processes and are not extensible to the surf zone, where the diffusive effect of the shorebreak becomes dominant. The result is the formation of transient morpho-sedimentary features throughout the mid-intertidal zone that form via positive morpho-sedimentary feedbacks in the ebb tide swash. These transient features might manifest as grain size `patchiness', or alternating bands of coarser- and finer-grained sediments with long- or cross-shore structure (see Fig. \ref{fig:patchiness_photo3}). \citet{Buscombe_Masselink2006} described similar features as `textural mosaics' which act as textural surrogates to morphological bedforms, the effect of texture on the nearbed flow being considered in terms of its `hydraulic equivalence' to morphological patterning.


\subsection{Feedback Mechanisms} % revised

At Advocate Beach, and on steep energetic beaches in general, the shorebreak does considerable reworking of the surficial sediments. This serves as a mechanism for morphological `smoothing', wherein features in the intertidal sediments are eradicated, particularly during periods of moderate- to high-energy wave forcing. The morphological smoothing implied by the negative correlation between changes in bed elevation from one tide to the next ($\Delta z$ and $\Delta z'$) is therefore unsurprising (consider, for instance, footprints in intertidal sand). More interesting is that the smoothing response of the mean grain size is similar -- the correlation between changes in bed elevation and the mean grain size from tide to tide ($\Delta z$ and MGS$'$) also being negative -- suggestive of the close linkage between spatial changes in morphology and grain size.

%The smoothing influence of the shorebreak was documented by \cite{Coco_etal2004} for the case of beach cusps on a macrotidal sand beach. They reported modulation in the amplitude of cusps with the tidal phase, the amplitude increasing during ebb and decreasing during flood. They attributed the infilling of cusp bays during the flood phase to the proximity of the impinging shorebreak and surf zone. However, their proposed mechanism is treated in terms of sediment diffusion, rather than the wholesale reset of the surficial sediments, as is generally the case at the Advocate site.

The positive correlations between changes in bed level and mean grain size ($\Delta z$ and MGS) and changes in bed level and changes in mean grain size ($\Delta z$ and $\Delta$MGS) at the beach surface can be explained in terms of mutually reinforcing feedbacks between flow, morphology, and the surficial grain size distribution. In the ebb-tide swash zone, emerging coarse patches enhance deposition: of fine material through kinetic sieving (effectively removing the fine particles from the beach surface), and of coarse material through grain interlocking and increased angles of pivot required for mobilisation. Away from the berm region, where sediments tend to remain coarse, it is logical that where grain sizes are large, there has been an increase in mean grain size (e.g., from the time-averaged mean). Conversely, coarse-grained topographic highs or fine-grained lows which form on the mid-intertidal beach face during the previous ebb tide are subjected to the destructive influence of the translating shorebreak during the subsequent flood. The result is the formation of transient morpho-sedimentary features throughout the mid-intertidal zone that form in the ebb tide swash, but which do not persist for longer than one tide. These transient features might manifest as grain size `patchiness', or alternating bands of coarser- and finer-grained sediments with long- or cross-shore structure (see Fig. \ref{fig:patchiness_photo3}). \citet{Buscombe_Masselink2006} described similar features as `textural mosaics' which act as textural surrogates to morphological bedforms, the effect of texture on the nearbed flow being considered in terms of its `hydraulic equivalence' to morphological patterning.

Stations surveyed near the HWL may be associated with different morpho-sedimentary processes (e.g., berm, cusp formation) than the mid-tide beach face, and may not be acted upon by the translating shorebreak. At the high tide shoreline, the mechanism for mobilising and transporting accumulations of coarse grains (i.e., swash) is weakened through hydraulic drag and infiltration, increasing the likelihood that a morphological feature would persist through successive tides.

% old, wrong
%The implication of the correlation results is that mutually supporting feedbacks between morphology, sedimentology, and hydrodynamics/hydraulics are responses to swash processes and are not extensible to the surf zone, where the shorebreak imposes a different regime of transport. 

%% what do I want to say here...
%- correlations are only possible (here) with subaerial observations
%- therefore it is not possible to observe signatures of feedbacks (with the methods described in this thesis) occurring (1) in the flood tide swash, or (2) seaward of the swash zone.
%- therefore it is not possible to report whether morphodynamically important sedimentary feedbacks occur anywhere but in the ebb tide swash. 
%- the fact that temporal correlations (from one tide to the next) are not significant indicates only that the sampling interval was not sufficiently short to resolve the processes.
%
%The implication of the correlation results is that signatures of mutually supporting feedbacks between morphology, the surficial grain size distribution, and hydrodynamics/hydraulics are 
%
%The ebb tide swash processes associated with the coevolution of bed level and grain size necessarily act over short timescales governed by the swash zone width and the rate of change of the mean shoreline position. The forcing associated with the passage of the shorebreak, surf zone, and flood tide swash zone leads to prolonged periods of sediment redistribution, resulting in spatial smoothing of relict morphological and sedimentary features in the intertidal zone, at least when observed with a sampling resolution of once per tide. 
%
%The implication of the correlation results is that ebb tide swash processes 


\subsection{Implications for Cusp Formation}

The conspicuous lack of well-developed beach cusps during the 2018 Advocate Beach experiment is in contrast to previous experiments at Advocate Beach. It is proposed that the reason for this is related to the timing of the experiment within the spring-neap cycle: i.e., the range of days over which data were collected corresponded to the transition from neap to spring tides (see the water column heights registered by the pressure sensor, Fig. \ref{fig:wavestats}), as noted in Section \ref{subsection:ChMSDBeach_corrDiscussion}. As a result, a coarse-grained high tide berm consistently coincided with the high water line (see Fig. \ref{fig:alltides_profiles}) -- at least during tides characterised by low- to moderate energy wave forcing, when cusps might have otherwise been expected to emerge. The coarse-grained berm migrated shoreward with the translation of the high tide shoreline.

As suggested by the results of the correlation analysis, the coarse-grained berm region was less conducive to the evolution of spatially correlated morphological and sedimentary features than the more well-mixed mid-intertidal zone. A wide surficial grain size distribution promotes the rapid emergence of beach cusps; where a narrower range of surficial grain sizes exists, the variation in space of textural feedbacks on the flow is less pronounced, and the emergence timescale of features is much longer, or features may not form at all (Chapter \ref{Chapter:CuspDynamics}). A well-mixed substrate was also suggested by \citet{LonguetHiggins_Parkin1962} to favour the formation of beach cusps through feedback mechanisms: the relatively high permeability of incipient accumulations of coarse grains enhances deposition through the loss of swash energy, whereas the lower permeability of a sand-gravel mixture promotes the transport of overlying particles, since less swash energy is lost to infiltration. 



%Unlike other features of the beach profile, such as the berm, which may persist for days to weeks, the transient features do not persist for longer than a single tide. Other features of Advocate Beach's intertidal profile, including cusps or a beach step, may be characterized by persistence timescales similar to the transient features described, or by timescales of secondary profile features. Large wavelength cusps having persistence times commensurate with the spring-neap tidal cycle have been observed at Advocate Beach during periods of swell following southwesterly wind events, though cusps more typically present as bands of low amplitude, loosely organized gravel and cobbles extending down the foreshore, with persistence times of the order of tides. No bed level or sedimentological data have been collected in the vicinity of the step at Advocate Beach, though visual observations during low energy forcing conditions suggest that a coarse step forms at the seaward edge of the swash zone during the period encompassing high tide, when the cross-shore translation rate of the shoreline is low. Though the evolution of the step is influenced by surf- as well as swash zone forcing -- unlike the features described here -- the step has been suggested to play an important role in the morphodynamic evolution of coarse-grained beaches through its influence on wave breaking \citep{Austin_Buscombe2008}, and its potential role as a source region for coarse material during the formation of bed forms in the swash \citep[e.g][]{Guest_Hay2019}.

%The results of the correlation analysis emulating that of \citet{Masselink_etal2007} suggest that the correlation between bed level change and mean grain size change is favoured during instances of accretion and coarsening: from the LT1 survey data, 86\% of accretion events corresponded to an increase in mean grain size. Correlation between erosion events and sediment fining was lower, having a 56\% co-occurrence. This may suggest that the coarser sediment fraction (gravel and cobbles) at the beach surface has a disproportionate effect on the morpho-sedimentary dynamics, relative to the finer fraction. However, strong conclusions here are precluded by the relatively small sample size.

%Relationships between the morpho-sedimentary correlation results and hydrodynamic parameters including significant wave height, peak period, wave steepness, and Iribarren number could not be established. This seems surprising, given the well-established relationship between forcing conditions and other instances of pattern formation and sediment size sorting: cusp formation, for example, which is favoured under low steepness, shore normal incident wave forcing. However, the lack of apparent relationship may be due to the sampling strategy -- the image-based grain size measure being prone to high noise levels -- or to the limited record length, or both. 




%The features do not persist for longer than a single tidal cycle, delimiting them from secondary features of the beach profile, such as the berm, and in some cases cusps and the step, which may persist for longer. The transient features might manifest as grain size `patchiness', or bands of graded sediments with long- or cross-shore structure (see Fig \ref{fig:patchiness_photo}). \citet{Buscombe_Masselink2006} described such features as `textural mosaics', and described the effect of texture on nearbed flow in terms of its `hydraulic equivalence' to morphological patterning (i.e., bedform surrogacy). %Though cusps might generally be described as a secondary beach feature, with persistence times of days to weeks, or even months \citep[e.g.][]{Carter_Orford1993, Vousdoukas_etal2011}, the MSG cusps at Advocate Beach might be better described as tertiary in terms of temporal persistence, typically presenting as bands of coarse material extending down the foreshore that are reworked during the following tide.

%\begin{figure}[tbp] %/home/tristan/Documents/Projects/AdvocateBeach2018/src/visualization/plot_beach_profile_data.py
%  	\includegraphics[width=\columnwidth]{figures/chapter3/photos/patchiness_19OctAM.JPG}
% 	\caption{Photograph of grain size `patchiness' at Advocate Beach, captured near low tide.}
% 	\label{fig:patchiness_photo}
%\end{figure}
%
%\begin{figure}[tbp] %/home/tristan/Documents/Projects/AdvocateBeach2018/src/visualization/plot_beach_profile_data.py
%  	\includegraphics[width=\columnwidth]{figures/chapter3/photos/patchiness_22OctPM.JPG}
% 	\caption{Photograph of grain size `patchiness' at Advocate Beach, captured near low tide.}
% 	\label{fig:patchiness_photo2}
%\end{figure}
%
%\begin{figure}[tbp] %/home/tristan/Documents/Projects/AdvocateBeach2018/src/visualization/plot_beach_profile_data.py
%  	\includegraphics[width=\columnwidth]{figures/chapter3/photos/patchiness_23OctAM.JPG}
% 	\caption{Photograph of grain size `patchiness' at Advocate Beach, captured near low tide.}
% 	\label{fig:patchiness_photo3}
%\end{figure}

%Surprisingly, the morpho-textural correlation does not appear to be strongly influenced by the forcing conditions. 
%...
%It might be expected that greater temporal persistence in the secondary morpho-sedimentology might result from periods of low energy wave forcing, though ... [I didn't really test this -- persistence/hysteresis is more dependent upon conditions (eg MGS) during the last tide, rather than dz, dmgs, for example]. 
	



%[Implications for longer-term profile response?]


%%- what is their significance?
%%	- highlight difficulty obtaining textural time series: surface GSD changes at swash timescales, but also inherently noisy. Useful for extracting trends if many data points are included.
%%	- dz and texture co-vary, but the relationship may not be simple, as in the case of berm rollover: fining on the face of the berm. coarsening may matter initially, but less so with continued morph growth? MGS time series are genereally coarse to begin with.
%%
%%- compare/contrast with AUstin + Mass (see their discussion on p.73)
%%- Pontee et al (2004) - also discussed by A+M 2006	
%%
%%- our results consistent with A and M's description:
%%"The foremost control on the cross-shore location of the berm is the shoreline elevation (Takada and Sunamura, 1982) and therefore the phase of the spring–neap cycle (Hine, 1979). During the rising phase of this cycle, vertical berm growth occurs; swash overtopping causes the berm to rollover landwards and the crest is re-established at the high water mark. This is consistent with recent observations of Weir et al. (submitted for publication), and some aspects of the beach response model for mixed beaches proposed by Pontee et al. (2004). However, unlike the responses observed by Pontee et al. (2004), where breaker heights exceeding 1m were required to cause accretion through the onshore migration of swash bars, onshore migration at Slapton occurred under low energy conditions (Hb<0.3m). The falling phase of the spring–neap cycle is also consistent with the observa- tions of berm morphodynamics byWeir et al. (submitted for publication). The accretion of gravel onto the seaward face of the berm increased its width, and caused seaward regression, as swash overtopping ceased."
%	
%
%% implications for long-term behaviour:
%
%- what do results mean for longer term profile evolution? 
%
%The results presented here describe a picture of spatio-temporally correlated `noise' about a dynamically evolving `equilibrium' profile. A larger question is the role of short-term (e.g. tide timescale) morpho-sedimentary responses in longer timescale responses of the beach. Namely, do secondary (berm, step), or `tertiary' (patchiness, zonation, in some cases cusps), morphological features of the beach profile influence the evolution of `primary' features (i.e. the slope)? Are there implications for the response of the beach to storm events, or the longer-term stability of the beach?
%
%- bed composition linked to pattern formation
%
%- [quote from Orford and Anthony Mar geo 2011]: 
%- "Orford and Carter (1984) provide a further connectionbetween short-term morphodynamics and long-term barrierbehaviour. They postulated that the rhythmic longshore posi-tioning of washover throats and back-barrier washover fansrelate to the rhythmic position of high-level beach cusps pro-duced through storm edge wave excitation on a reflectivemixed sand-gravel barrier."
%- size of clasts effects transport on upper beach
%- steepness of profile
%- "Any morphodynamic characterisation could also be conditioned by anumber of other parameters that include: the intensity and structureof the overflowfield; antecedent beachface characteristics inresponse to storms; the rate of relative sea-level change; barrierresistance to forcing, itself determined by a number of unknownsincluding ridge form and dimensions, sediment size and mosaics;and barrier resilience. We suggest that characterisation of GDCSmorphodynamics in terms of extreme events, which force morpho-logical change, will necessitate integrating some or all of theseparameters into a single model"
%
%
%%%
%longer term implications:
%
%Prodger et al , Mar Geo 2016:
%- Skilful model predictions of temporal evolution of MGS and sorting that incorporates antecedent disequilibrium wave steepness (i.e.: grain size and sortign purely a function of the hydrodynamcis). All study sites were medium sand -- narrow GSD.
%[likely does not hold in a MSG setting, and indeed, carries the opposite sign. suggests fundamental distinctions between the morpho-sedimentary dynamics of sand and MSG beaches.]
%
%[my words: though it would be interesting to pursue a similar analysis (to Prodger's) on MSG beaches, the reversal of sign represents a fundamental difference in the transport dynamics.]
%%%
%
%
%Re. bed level change. The results of the MSD correlation analysis -- i.e. positive between MGS, dMGS and dz, and negative between MGS(t-1) and dz (particularly the latter) -- is suggestive of the mechanism responsible for maintenance of the beach face in a state of dynamic equilibrium that is also shown here, and by others (turner, etc), in bed level change results. Namely, though the magnitude of bed level change associated with a single swash event may be on the order of cm, frequent reversals in the sign of bed level change inhibit large net changes. Net changes in bed level instead occur through the time integration of a large number of positive and negative changes of small magnitude.
%
%Given the correlation with seidmentology (grain size), it may be speculated that the forcing mechanisms that underlie this process (of dynamic equilibrium on inter-swash time scales (blenkinsopp et al , CEng2011, pg 43)) include textural feedbacks, as well as morphological and hydrodynamic.
%
%The suggestion that morphology and sedimentology display correlated time histories has been suggested by other authors (Pontee, Buscombe), but is difficult to demonstrate [bunk]. Buscombe and mass noted a disagreement in the past literature between those who believe ... and ... . The results presented herein clearly support the latter case. 
%
%
%Masselink et al Sedimentol. (2007)discuss : "The absence of distinct secondary morpho-
%logical features on macrotidal beaches and their morphological stability are primarily caused by the tide-induced migration of the different mor- phodynamic zones (swash, surf and shoaling) across the beach profile. The significance of tidal translation for beach morphological development can be quantified by the ratio of tidal range to wave height (Masselink, 1993). For values of this ratio in excess of 5–10, the morphodynamic effect of tidal translation is significant, giving rise to the occurrence of distinct tide-affected, but still wave-dominated, beach typologies (Masselink and Short, 1993)."
%




\section{Conclusions}

The morpho-sedimentary evolution of a mixed sand-gravel beach was investigated at the scale of the beach profile, through the sampling of bed level and mean grain size at \textit{ca}. 2500 points collected over 14 tidal cycles. The forcing conditions were dominated by large (\textasciitilde 10 m) tides and steep, wind-band incident waves leading to an energetic shore break, interspersed by periods of low steepness wave incidence and berm building. 

A pronounced negative correlation between wave height and mean surficial grain size was observed. Though at odds with the prevailing understanding of sand beach grain size change in response to forcing, a similar inverse correlation has been observed at other mixed sand-gravel beach sites, under a range of forcing and tide conditions. All the studies of which the author is aware that report fining of beach surface sediments under energetic wave forcing were conducted on steep MSG beaches experiencing some degree of wave energy limitation, though the tidal regimes varied from micro- to megatidal. The large differences in tidal range among these MSG beaches suggest that interplay between surf- and swash zone processes may not be of principal importance, and the dominant mechanism may be swash-related. However, given the relative scarcity of field studies at MSG beach settings which include concurrent observations of grain size and forcing conditions over timescales of several weeks, and the implications for the prediction of sediment dynamics in these environments, this suggestion warrants further investigation. 

A persistent positive spatial correlation was observed between tide-to-tide changes in bed level in the intertidal zone, and both the mean grain size and mean grain size change. Negative spatial correlations were established between bed level change and both the mean grain size observed during the previous tide, and the change in bed level observed during the previous tide. Both the positive and negative correlations were more evident in the mid-intertidal zone, seaward of the region typically occupied by the active berm. The correlation results were attributed to the formation of ephemeral morpho-sedimentary features in the ebb tide swash zone through mutually reinforcing feedbacks, and the subsequent destruction of the features through shore-break and surf zone morphological smoothing during the following flood tide. The sign of the correlation between bed level change and mean grain size is opposite to the phenomenological relationship expected on other beach types \citep[e.g.,][]{Masselink_etal2007}. The lack of significant temporal correlations between bed level and grain size (from one tide to the next) may indicate that the sampling interval was not sufficiently short to resolve time-coherent processes that occur over timescales which are commensurate with the forcing ($O$(10) s), or with the evolution of the morpho-sedimentary features ($O$(1000) s).

The persistent correlation between bed level and mean grain size changes supports the suggestion that sediment characteristics reinforce morphological change \citep[see][]{Buscombe_Masselink2006}, at least over tidal time scales.

%However, given the well-established relationship between it may be expected that  , or whether the processes are better described as being superimposed upon the larger process of dynamic equilibrium profile evolution.

%Clear shortcomings of this study are the limitation of sampling to the beach surface sediments, and the lack of observations of higher order moments of the grain size distribution, namely: measures of grain size sorting, skewness, and kurtosis. Also not included in this study was the role of particle shape, which has also been shown to play an important role in transport dynamics.





%[from curtiss et al, on hsig vs MGS:] Such changes in beach composition,which may be somewhat unique to MSG beaches, are very challengingto model and difficult to observe due to their spatial heterogeneity andthe long timescales over which they occur. Although the dynamicaldetails related to this observed shift in beach composition are notwell-resolved in the current study, it is a striking feature of theseasonal variability on the beach and, thus, likely to be important inthe overall morphological and sediment transport regime."
%
%[Miller et al Mar geo 2011] "This size-selective model suggests that identifying a significantdifference between size classes may be more likely in low energyconditions, when clasts in transport show a greater degree of size-selective transport."

%\section{Appendix}

%Validation of digital grain size results
%
%[plot: point counts, sieve data]
%[plot: -0.5 vs 0.5 to point counts, sieve data]


%\section{Notes}


%% Nordstrom and Jackson 1993
%- mean tide range 1.6 m
%- dominant waves wind generated
%- steep foreshore 6.2deg, flat low tide terrace
%- Grain-size curves of surface sediment de-
%posited as a result of storm wave action (Fig. 5A) reveal finer, better sorted
%sediments and greater homogeneity of characteristics across the beach than
%sediment deposited by low-energy, nonstorm waves

% Curtiss etal 2009
%- steep (1:5-1:7)
%- A unique feature of the beach is the median size of the gravel, which increases with decreasing elevation on the beach as also observed by Nordstrom and Jackson (1993) on a low energy estuarine beach
%- mean tide range 3.5 m
%- fetch restricted (in a passage exposed to vessel wakes, tides, wind waves)
% A shift to a stable, coarse sand composition on the upper beach is a
%distinct characteristic of the storm condition on the beach. The observations are consistent with a reverse winnowing process, in which the gravel is more mobile than the fine fraction. However, wecannot be certain of the sand behavior without direct measurements of the fine fraction. After the cessation of storm waves, gravel slowly accretes on the upper beach. Ferry wakes appear to be a mechanism for post-storm recovery of gravel on the upper beach. Tidal currents are not a likely mechanism because they are not strong enough to mobilize gravel-sized sediment. Such changes in beach composition, which may be somewhat unique to MSGbeaches, are very challenging to model and difficult to observe due to their spatial heterogeneity and the long timescales over which they occur. Although the dynamical details related to this observed shift in beach composition are not well-resolved in the current study, it is a striking feature of the seasonal variability on the beach and, thus, likely to be important in the overall morphological and sediment transport regime


%% MIller et al 2011
%- fetch limited - in strait of Juan de FUca (mean wave heigh 0.5 m)
%- bimodal wave forcing (10 and 4 s - ocean swell, local wind, )
%- mean tide range 1.4 m
%- We observe that high waves tend to be associated with an increase in the sand fraction visible on the bed, suggesting thoroughmixing of the beach substrate

%% Pontee et al 2004
%- 3 beaches, all semidiurnal with 2-3 m tides
%- avg annual wave heights 0.4-0.5 m
%- mean annual periods ~6 s
%- wave height + energy lower betwen Walberswick and Thorpness than from Aldeburg to Thrpness

%
%What do I know about the correlation between elevation and grain size at the beach surface?
%
%- A weak (r=0.2-0.5) positive correlation exists between changes in bed elevation and both MGS/sorting and MGS/sorting trends. 
%- A weak(er) negative correlation (r~0.2) exists between changes in bed elevation and MGS at the previous low tide.
%- The positive correlation is apparent in all survey datasets.
%- The positive and negative correlations appear to strengthen with distance seaward.
%	- The strongest positive and negative correlations exists for the longshore 1 transect.
%	- Near zero correlation on the most shoreward transect of the dense array 2 survey grid.
%- The degree of correlation within a given grid/transect does not correlate significantly with hydrodynamics:
%	- significant wave height
%	- peak wave period
%	- wind band wave energy
%	- wave steepness
%	- Iribarren number
%- The degree of correlation within a given transect does not correlate significantly with bed sediment properties:
%	- mean MGS over the transect
%	- mean sorting over the transect
%- The degree of correlation within a given transect does not correlate significantly with distance from the HWL, with the exception that transects above the HWL have near-zero correlation values (as should be the case -- correlating noise).
%
%Since most of the investigated transects were near, or above, the HWL some or all of the time (depending on cross-shore position), the logical conclusion is that near-zero-valued correlations associated with proximity to the HWL lowered correlation values for the more shoreward transects. Thus, I'll assume that the correlation holds throughout the intertidal zone. 
%
%The low correlation values in general can be attributed in part to the inherent noisiness of grain size estimates. In trying to explain the variation in degrees of correlation between tides, it is worth remembering that erroneous features (e.g. seaweed, footprints, boulders) in a few of the images on a given day may have a large impact on the resulting correlation coefficient.
%
%
%% moved from discussion:
%It seems likely that correlations existed with statistical significance where change consistently occurred, i.e., where the transects were below the HWL for all or most of the experiment. For example, DA2-1 -- the most shoreward longshore transect of DA2 -- was above the HWL until tide xx, and exhibited correlation values in the range of cc-dd. Conversely, LP1 -- the most seaward of the regularly sampled transects -- was below the HWL for the entire experiment and showed the highest correlation coefficients. 
%
%Debunked for this dataset:
%It may also be reasonably concluded that stronger correlations could exist where stronger contrasts existed within the sediment; that is, away from the high tide berm, where the bed was predominantly and persistently coarse. However, no significant correlation was found between the mean value of $M_0$ or $\Delta M_0$ for a given transect, and the morpho-sedimentary correlation coefficient. 
%
%Speculative discussion: LP1 cross correlations suggest that, for a given location on the beach surface, (1) if the mean grain size was larger, the bed elevation was likely higher; (2) If the trend of mean grain size change from the previous to the current tide was positive, the (current) bed elevation was likely higher; and (3) if the mean grain size during the previous tide was larger, there was likely a decrease in (current) bed elevation. [That all needs rewriting...]
%
%Though there are not enough data to draw any strong conclusions about why this might be the case, potential reasons for the apparent anticorrelation between mean grain size (t-1) and bed level change are described below.
%
%
%%% Misc
%Mass et al, Mar Geo 2010 (Swash zone sediment transport): H/L < 0.01 conducive to berm accretion (e.g. Komar 1998 - beach processes and sedimentation)
%
%
%%%
%
%It is important to note that, since all surveying was done at low tide (or at least while the upper beach was exposed), the surveyed morpho-sedimentology at a given point was most recently acted upon by ebb tide swash processes. Stations surveyed near the HWL are exceptions, as they represent high tide swash forcing, which can be associated with different morpho-sedimentary processes than the mid-tide beach face (e.g. the presence and evolution of a berm). At the high tide shoreline, the mechanism for mobilizing and transporting accumulations of coarse grains (i.e. swash) is weakened through hydraulic drag and infiltration (e.g. Guest and Hay, 2019). On the mid-tide beach face, transient accumulations of coarse material (which are also likely to be topographic highs) are acted upon during flood and ebb by the surf zone as well as the swash zone. The diffusive influence of the surf zone would act disproportionately upon the topographic highs. The diffusive influence of the surf zone on transient morphological beach face features has been previously observed for the cases of beach cusps (Coco et al, 2004), [berms? ridges? other?].  
%
%Other reasons why the action of ebb tide processes might be important here? What do we know about flood-ebb differences in beach profile evolution?
%- infiltration effects are exaggerated between coarse and fine substrates (see cusp paper)
%
%Reposing the above question... What effect might the rate of cross-shore shoreline change have on the morpho-sedimentary dynamics? Why is the relationship between bed level change and sedimentology apparent in our results, but not in others? The answers to these questions may be related. (May also tie in as an argument for the lower observed correlations near the HWL -- see notes below.) 
%Some answers:
%1. Masselinks (1993) tide-wave parameter. Here, we're looking at a heightened signal in the intertidal zone, where -- because of the influence of tides -- there are generally no distinct secondary morphological features (planar beach).
%2. Wide grain size distribution
%
%Likely largely to do with the larger range of grain sizes (gravel vs MSG beaches). This is the first analysis of its kind carried out at a MSG beach. There may, though, be some significance of processes in the shoreward edge of the swash zone, which for much of the intertidal gets forced for only a brief period during ebb, perhaps allowing the formation of incipient morphological features.
%
%Shortcomings:
%- only sampling the surface, not the full active layer
%- 
%
%
%
%The sign of the correlation is oppoiste that of sand beach settings (masselink et al Sedimnenoloy 2007, in their cases where a correlation was present).  Also note that higher waves would typically be expected to lead to bed coarsening as a result of preferential removal of fine fractions. Also, sorting processees during sediment trans are expecte dto effect temporal trends in sediment characteristics (also from mass paper), "At a most basic level, it can be hypothesized that areas of erosion should experience a sediment coarsen- ing because of the formation of a lag deposit"