\chapter{Introduction}\label{Chapter:Introduction}

%- sandy beach progress/history
%- gravel beaches
%- challenges?
%- feedbacks (equilibrium- sand beach, more complex with introduction of texture)
%- mixed sizes, self org, chaos
%- sorting
%- [holman bowen, coco, murray thieler - variable forcing condition sdrive spatiotemporal changes in sed transport + gradients, leading to 3d spatial apatterns/morphologies]. The presence of a wide range of grain sizes introduces sediment sorting processes as a result of differing mobilization/deposition characteristics. Particularly striking on coarse beaches, where vertical and horizontal size gradients can lead to complex (morpho-)sedimentary mosaics (cite busmbe?).
%- hydraulics - graoundwater, dependence on grain size, etc	  
%- definitions (in lieu of diagram, ... "step, the region...", berm, theregion (active/inactive))
%- relationship to sediment transport
%- the gap between small scale processes, understood on the basis of first principles, and phenomenological preocesse at the beach scale is not yet closed (can pontificate about this based on what I've thought about - doug hofstadter stuff?)
%- follow with talk about phenom vs modelling/first principles approaches.
%- where MSG beaches show up
%- why they are important (interst in quantitative prediction)


\section{Morpho-Sedimentary Dynamics}

% Beaches are dynamic and morphologically diverse regions of the nearshore zone, serving as the buffers between land and the tremendous energy of the sea. Shoreward-propagating wave energy is in some proportions dissipated or reflected back to sea. The energy not reflected is absorbed by the system through turbulent dissipation following wave breaking, and sediment transport. The resulting system is highly nonlinear, and like many damped, forced systems exhibits complex dynamics \citep{Lorenz1963}. The constant redistribution of sediments by the mechanical energy of wind, waves, currents, and tides 

%Beaches are dynamic and morphologically diverse regions of the nearshore zone, serving as the buffers between land and the tremendous energy of the sea. Shoreward-propagating wave energy is in some proportions dissipated or reflected back to sea. The energy not reflected is absorbed by the system through turbulent dissipation following wave breaking, and sediment transport. The resulting system is highly nonlinear, and like many damped, forced systems exhibits complex dynamics and chaotic behaviour, at times characterised by morphological features resulting from the imposition of hydrodynamic structures, or as a product of internal feedbacks leading to self-organisation. 

%Beaches are dynamic and morphologically diverse regions of the nearshore zone, where the tremendous energy of  .... is  shoreward-propagating wave energy are in some proportions dissipated or reflected back to sea. The energy not reflected is absorbed by the system through turbulent dissipation following wave breaking, and sediment transport. The resulting system is highly nonlinear, and like many damped, forced systems exhibits complex dynamics and chaotic behaviour \citep{Lorenz1963, Baas2002}, including signatures of fractal geometry (cite) and self-organisation (cite). ...[definitoin too stark here] Signatures of self organisation, which describes the emergence of structure or order in a system through its internal dynamics and feedback mechanisms [Baas], as opposed to the generation of regularity as a result of external forcing, arise throughout the geomorphological sciences, as nearly all dynamic landscapes exhibit some degree of nonlinear interaction between system elements \citep{Baas2002}. [examples?]

% at times characterised by morphological features resulting from the imposition of hydrodynamic structures, or as a product of internal feedbacks leading to self-organisation. 
%
%The relative importance of internal dynamics (i.e., self-organisation) and structure imposed by external forcing on coastal morphodynamics is difficult to quantify, and has proven in some cases to be contentious (cite).
%
%Beaches are dynamic and morphologically diverse regions of the nearshore zone. When the energy from the sea interacts with the beach, the mechanical energy associated with wave- and tide-induced currents is transferred to the beach via sediment transport, and heat energy through viscous dissipation following turbulent wave-breaking. The resulting system is highly nonlinear, and like many damped, forced systems exhibits \textit{complex} dynamics \citep{Lorenz1963, Phillips1992}, including signatures of self-organisation. Time- and space-varying hydrodynamics result in gradients of sediment transport that drive morphological change. .. the fluid -sediment interface..,  leading to three-dimensional spatial patterns or morphologies, in some cases having striking periodicity, as in the cases of sorted bed forms (cite), ripples \citep{Nishimori_Ouchi1993} [strictly speaking: windblown], rips, bars, crescentic bars, cusps (make sure what you include is self-org related, not templates) ... Positive internal feedbacks , e.g., consider the case of ... . Negative feedbacks limit instability ... equilibrium .. 
%
%Both have been acknowledged/shown ... deterministic complexity has seen greater applciaiton in the last ... , beginning with the work of Lorenz. 
%
%resulting systems may display self-organising behaviour, characterised by the grouping together of sediments into organised structures, sometimes with striking periodicity. 
%
%Internal dynamics give rise to negative feedbacks, which limit the range of states a system can occupy through self-regulation
%
% The process of morphological adjustment on beaches is typically described in terms of spatial and temporal variability in the hydrodynamics, which gives rise to gradients in sediment transport



Beaches are dynamic and morphologically diverse regions of the coastal zone. In the process of energy dissipation at a shoreline, the mechanical energy associated with wave- and tide-induced currents is transformed into heat through turbulent dissipation, for example, following wave breaking, or transferred to the bed facilitating sediment transport. The resulting system is highly nonlinear, and like many damped, forced systems exhibits complex dynamics. Complexity in this sense may be \textit{deterministic} \citep[e.g.,][]{Lorenz1963}, wherein nonlinear interactions between system elements are represented by relatively simple systems of deterministic equations, but with sensitive dependence on initial conditions leading to complex outcomes, or \textit{stochastic}, wherein complexity arises through the cumulative impacts of numerous process-response mechanisms, or through multiple controls on process-response relationships operating over ranges of spatial and temporal scales \citep{Phillips1992}. Many, or perhaps even most, geomorphic systems and landforms are forced and dissipative by nature, giving rise to complex dynamics and signatures of self-organising behaviour. Beaches are no exception: external forcing (e.g., wave- or tide-induced currents) elicits responses from fundamental system elements (e.g., grains of sand), which further influence the system state through internal dynamics and feedbacks (e.g., critical thresholds such as angles of repose), with outcomes ranging from the emergence of patterned bed states with striking regularity, as in the case of beach cusps \citep[e.g.,][]{Werner_Fink1993, Coco_etal2000}, to the characteristic shape of a beach as a whole \citep[i.e., its profile:][]{Dean1991, Turner1995}. 

Positive and negative feedbacks arise through asymmetries between the characteristic response times of morphology and hydrodynamics: change is induced though spatial and temporal variability in the hydrodynamics, which give spatial and temporal structure to erosion and deposition processes through gradients in sediment transport. Changes in the morphology, which occur over longer timescales, feedback on the flow by altering velocity, and therefore transport, gradients, in processes that may be constructive (positive feedbacks, self-organisation) or destructive (negative feedbacks, self-regulation). This process of morphological adjustment through mutual associations between form and flow, facilitated by sediment transport, is the essence of the morphodynamic model from \citet{Wright_Thom1977}, which emphasises the intrinsic interdependence of fluid dynamic and morphodynamic processes, such that cause and effect are not readily apparent.

% morpho sedimentary dynamics

The vast majority of beach research has been concerned with the morphodynamics of beaches composed of sand. On beaches where grain sizes are large, or where a broad range of grain sizes exists, texturing of the bed and changes in the hydraulic properties exert an added control on the hydrodynamics through increased roughness effects at the bed-fluid interface and changes in the vertical flow properties through the bed as a porous medium. The inclusion of textural feedbacks in the morphological evolution of the beach through flow-morphology interactions renders a morphodynamic model incapable of completely describing the system's dynamics. In their gravel beach review paper, \citet{Buscombe_Masselink2006} acknowledged this shortcoming, and urged the adopting of a `morpho-sedimentary dynamics' framework for the future study of coarse-grained beaches. \citeauthor{Buscombe_Masselink2006} \citep[see also][]{Carter_Orford1993} define morpho-sedimentary dynamics (MSD) as: ``the mutual association and feedbacks in operation between flows (hydrodynamics and hydraulics), and forms (morphological architectures and textural mosaics), mediated through selective sediment transport mechanisms acting upon the mechanical, hydrodynamic and hydraulic properties of sediments."

%% similarities + distinctions [might be better suited to MSG beach section]
%
%Mixed sand-gravel beaches have features in common with both pure sand and gravel beaches: Like sand beaches, they are characterised by lower hydraulic conductivities -- i.e., the rate at which water can pass through the pore spaces of the medium, having units of distance per unit time -- than pure gravel. Similar to gravel beaches, MSG beaches are generally steep-sloped, and often characterised by narrow surf zones leading to energetic shore breaks and morphodynamics dominated by processes in the swash zone -- the region of the beach intermittently covered and exposed by water following wave breaking. Despite any similarities, MSG beaches have been noted to be morphologically distinct from, and more complex than, both pure sand and gravel beaches \citep{Kirk1980}.

% grain size interactions; phenomenology vs first principles

Linkages between morphology and the grain size distribution have been noted in the literature for beaches of all types, most notably characterising features like high tide berms and beach cusps, where topographic highs typically correspond to coarser sediments, as well as the beach step -- a discontinuity in beach slope that corresponds to the region of bore collapse at the seaward edge of the swash zone -- which is generally observed to be composed of coarser material. The association between morphology and grain size can be particularly striking on mixed sand-gravel (MSG) beaches, where sedimentary structures which are well-organised in terms of grain size are often observed in addition to, or in conjunction with, morphological structures. \citet{Buscombe_Masselink2006} noted that textural patterning on coarse or mixed-grained beaches may serve, through their effects on the flow velocity gradients near the bed, as surrogates to the morphological bedforms of sand beaches, in other words, that the two types of feature are to some extent `hydraulically equivalent'.

% Gravel/MSG motivation to study

Gravel beaches are a common feature of the world's coastlines, particularly in the mid- to high-latitudes where active terrestrial weathering or glaciation in the recent geologic past have provided ample source material. MSG beaches form a subset of gravel beaches \citep[see][]{Jennings_Shulmeister2002} which have recently received increased attention in the literature. The hydraulically rough and permeable nature of coarse sediments provides an efficient mechanism for wave energy dissipation at the shoreline, thus providing a natural and sustainable form of coastal defence \citep{VanWellen_etal2000, Almeida_etal2014}. In the face of rising sea levels, the increasing demand for coastal stabilisation and erosion prevention has led to the use of gravel and sand-gravel mixtures as soft-engineering solutions, through replenishing vulnerable or eroding coastlines with imported material. MSG beaches in particular have seen a considerable increase in interest, since the imported material used in replenishment schemes increasingly consists of mixed-sized sediments dredged from offshore \citep{Mason_Coates2001}. However, sediment transport processes on MSG beaches are not well studied in comparison to sandy beaches, and the need for further research to support engineering designs, re-nourishment projects, and modelling efforts, is stressed in the current literature. % \citep{Mason_Coates2001, Ivamy_Kench2006}. 

On mixed-sediment beaches, the additional degrees of freedom associated with spatial and temporal grain size heterogeneity, and the propensity for mixed sediments to self organise into assemblages of like-sized grains, make characterising the response of the bed to forcing more difficult; in addition to the space- and time-varying influences of hydrodynamics and morphological feedbacks, the characteristics of grain mobilisation and deposition are also influenced by the `background' grain size distribution. For example, the likelihood of a coarse particle being deposited is decreased where the background material is relatively fine (the ease of rolling the particle causing it to be `rejected' by the substrate) and increased where a like-sized substrate `accepts' the particle due to grain interlocking. Similarly, larger particles serve as roughness elements, more susceptible to lift and drag forces, but capable of `sheltering' neighbouring grains \citep{Buscombe_Masselink2006}. Processes of self-organisation leading to spatial and temporal heterogeneity of the grain size distribution also have implications for the hydraulic properties of the sediment \citep[e.g.,][]{Horn2006}. The relative influence of the percolation of water through the beach surface on sediment transport, particularly in the swash zone, must also be assumed to vary in time and space.

The interactions between sediment size-fractions during transport are nontrivial, and along with hydraulic effects, impose a significant challenge for the prediction of MSG beach sediment transport. A sound observational basis is required in order to parameterise the dynamics in a predictive framework, including a phenomenological understanding of the influence of variations in the grain-size distribution on transport and morphology. Acknowledging the shortcomings of state-of-the-art sediment-water flow descriptions, \citet{Buscombe_Masselink2006} note: ``A sensible and pragmatic approach would be to first document field observations and phenomena, then work ‘backwards’ into the underlying physics.''

% observational challenges

The paucity of measurements of hydrodynamics and sediment transport on MSG beaches is due in large part to difficulties in obtaining observations \textit{in situ}. The steeper slopes characteristic of gravel and mixed beaches lead to narrow, energetic shore breaks capable of entraining gravel- and cobble-sized grains. The utility of sensors commonly employed in sandy beach settings is therefore limited. Remote observation methods making use of, e.g., video \citep{Holman_Stanley2007}, aero-acoustic sensing \citep{Turner_etal2008}, terrestrial lidar \citep{VanGaalen_etal2011}, and digital imagery \citep{Buscombe2013}, as well as instrument burial \citep[e.g.,][]{Raubenheimer_etal1998, Michallet_etal2009} provide alternatives. Also rare in the literature are observations of grain size dynamics sufficient to undertake meaningful investigations of morpho-sedimentary processes in the swash zone. Until recently, such observations were not obtainable in a non-intrusive manner. The increasing usage of digital grain sizing methods which are capable of producing estimates of grain size distributions from optical imagery is promising for the study of morpho-sedimentary dynamics.

% this thesis

In this thesis, field data are emphasised, with observations of hydro/morphodynamics, grain size, and sediment dynamics from the intertidal zone of a mixed sand-gravel beach. The study site is Advocate Beach: a steep mixed sand-gravel beach with a tide range often in excess of 10 m. The swash zone is a notoriously difficult region to obtain observations due to the presence of often energetic, aerated flows. The presence of a large tide range has several benefits in this regard: (1) the intertidal zone can be accessed each low tide to install, retrieve, or service \textit{in situ} instrumentation; (2) the force applied per unit area of the beach surface is reduced; and similarly, (3) the time resolution of observations, e.g., before and after, with regard to the passage of the swash zone, can be improved. In effect, these benefits result in an improved signal to noise ratio, since morpho-sedimentary signatures are spread over a large intertidal beach face, allowing a finer-scale look at transport processes in space and time. Though the presence of a large tide range, coupled with generally well-mixed beach sediment composition and fetch-limited forcing, is not representative of MSG beaches in general, the benefits of an improved signal to noise ratio on understanding small-scale transport processes make Advocate Beach a worthwhile study site.


% For application in engineering designs, management, coastal conservation, and ecology, the ability to predict the evolution of coastal morphologies is desirable, and has been actively researched for centuries, particularly since the second world war \cite{Johnson}. Despite substantial gains in our understanding of coastal morphodynamics and sediment transport [more?], the coastal zone (region -- depth of closure) and, in particular, the swash zone, remain a challenging place to make predictions, in large part due to the challenges inherent in obtaining observations. [more? details of sediment transport?]



%%%%%
% irrelevant
%The relative stability of coarse grained beaches in the face of energetic conditions and rising sea levels, along with an acknowledged deficit of process-based understanding of gravel and mixed sand-gravel beaches has led to an increased emphasis on gravel beach research over the last two decades. Promising advances in modelling (.. McCall XBeach-G) and directed field (Slapton stuff?) and lab (BARDEX?) campaigns (...) have led to meaningful gains in our understanding of hydrodynamics, groundwater interactions, and profile response, particularly for beaches composed of pure gravel. 

%"A number of
%factors are important on mixed beaches for determining rates and
%patterns of sediment transport, including spatial and temporal
%variability in hydraulic conductivity, swash and backwash hydrody-
%namics, the critical transport threshold for grain entrainment, the
%potential for fractional transport (Mason and Coates, 2001; Osborne,
%2005), and variations in grain size and grain size sorting (McLean and
%Kirk, 1969; Kirk, 1980; Adams et al., 2007)."
%
%"Contemporary sediment
%transport models derived from physical principles do not explicitly
%incorporate these parameters and empirical data from mixed beaches
%that can be used to calibrate global coefficients for analytical sediment
%transport models are scarce (Van Wellen et al., 2000; Mason and Coates,
%2001). As a consequence, there is uncertainty in the application of
%theoretical models of sediment transport to mixed beaches."

%Of particular note in understanding the evolution of gravel beaches is the role of beach surface texture, as determined by properties of the surface grain size distribution, on the hydrodynamics, groundwater dynamics (hydraulics) and morphology. Though the importance of bed texture is implicit in modern sediment mobilization and transport models via the their dependence on roughness lengthscales, the evolution of bed surface texture in time and space is not well understood, particularly on MSG beaches. 

%Carter orford, JCR1993
%- acknowledge lack of research of short-term morphodynamics on coarse beaches due to "difficulties placing sensitive instruments on steep narrow shorelinessubject to high energy plunging breakers".
%- discuss the role of background substrate on individual particle transport (MSD)



%%%%%

%Lay out gaps in understanding and model shortcomings as I described it to Alex. [see notes in notebook]
%- xbeach
%- phenomenology
%- how to include sorting, bimodality
%- current models limited to 2d, storm conditions

%The swash zone is the region of a beach intermittently covered and exposed by water folowing wave breaking. It is the region in the nearshore about which sediment transport processes are least understood, and was considered until recently to be beyond the state of the art to model swash zone seidment transport. Yet, the swash zone is one of the most dynamic regions of the nearshore
%
%It is the region in the nearshore about which sediment transport processes are least understood, though also one of the most dynamic.
%
%Despite increased interest, there still exists a dearth of field observations of sediment dynamics on MSG beaches. This is due in large part to the often energetic shore break, which can contain mobile flotsam and gravel to boulder sized particles, creating a hazard for in situ instrumentation. 
%
%The wide distributions of grain size lead to spatial and temporal variations in sediment mixture, which directly influence hydraulic conductivity and infiltration effects, both of which have been identified by Mason and Coates (2001) as first order factors influencing mixed beach transport. Instances of pattern formation and grain size sorting (e.g., beach cusps) are also common on MSG beaches, contributing to the spatial and temporal heterogeneity of beach surficial sediments. 
%
%the need for further study -- to support engineering designs, re-nourishment projects, and modelling efforts -- is stressed in the present literature (\cite{Ivamy2006}; \cite{Mason2001}). 
%
%Current numerical models ... Existing models are limited to 2-d, and are generally applicable only to storm conditions. How to include a bimodal grain size distribution?

%from MSc proposal:
%Shoreline morphology and hydrodynamics have implications for the transport of pollutants, beach ecology, and food webs in the nearshore. Bed formations in the nearshore (having potential dynamic similarities to beach cusps) also impact a number of coastal engineering applications, including pipeline and cable burial, channel dredging, and offshore gas mining (\cite{Coco2007}).

%The presence of gravel and (to a greater extent) MSG beaches in the above mentioned discussions has been notably absent in the literature. MSG and gravel beaches have received increasing consideration as sustainable coastal defenses (\cite{vanWellen2000}), with the ability to prevent flooding in low-lying back barrier regions and undercutting of coastal cliffs during storm events. Coastal engineering schemes involving beach replenishment with mixed sediments and gravel have become a preferred method for short-term stabilization of eroding coastlines (\cite{Mason2001}). However, process controls on mixed-sediment beach sediment transport and morphology are poorly understood, and the need for further study -- to support engineering designs, re-nourishment projects, and modelling efforts -- is stressed in the present literature (\cite{Ivamy2006}; \cite{Mason2001}). 


%from QE statement:

%Despite increased interest, there still exists a dearth of field observations of sediment dynamics on MSG beaches. This is due in large part to the often energetic shore break, which can contain mobile flotsam and gravel to boulder sized particles, creating a hazard for in situ instrumentation. The wide distributions of grain size lead to spatial and temporal variations in sediment mixture, which directly influence hydraulic conductivity and infiltration effects, both of which have been identified by Mason and Coates (2001) as first order factors influencing mixed beach transport. Instances of pattern formation and grain size sorting (e.g., beach cusps) are also common on MSG beaches, contributing to the spatial and temporal heterogeneity of beach surficial sediments. 

%Remote sensing techniques such as fixed camera video (Holman and Stanley, 2007), and more recently, UAV-based structure from motion photogrammetry (Turner et al., 2016) have proven to be valuable tools for observing coastal morphodynamics, and can provide cost-effective and low-risk alternatives to deploying instruments in situ. However, such techniques generally overlook small-scale underlying processes. Direct observations (e.g., pressure) of surf- and swash zone hydrodynamics can complement remote observations, but instruments must be robust enough to withstand the often harsh conditions.

%For my PhD research, I will combine remote and in situ field observation methods to investigate sediment transport and morphodynamic processes on a megatidal MSG beach. Specific areas of focus will include (1) beach cusp morphodynamics, (2) coarse particle (gravel-boulder) transport in surf and swash driven flows, and (3) pore/groundwater response to shallow water wave loading. Beach cusps will be discussed in terms of mechanisms (i.e., self-organization versus edge waves), forcing (e.g., short versus long-crested breakers, wave reflection, in/exfiltration), and response (e.g., coarse particle advection-diffusion). Given the important role of coarse particle transport in the formation and evolution of MSG cusps, items (1) and (2) are complementary, and together provide an accessible means for evaluating the influence of grain size sorting on transport. Item (3) constitutes past work (Guest and Hay, 2017), and highlights the influence of tidal range, via the introduction of pore-trapped air, and hydraulic conductivity on the pressure response of the intertidal bed. 




\section{Research Objectives and Thesis Outline}\label{section:Objectives}

%%%
%% [in  here?]: including a phenomenological understanding of the influence of variations in grain-size distribution (e.g. sorting, anisotropy) on transport and morphology.
%[cut, or moved to objectives section]
%The overarching objectives of my PhD research are to explore remote and \textit{in situ} methods for obtaining field observations at an energetic MSG beach site, to investigate links between variations in sediment size and morphological change, and to improve characterizations of swash-zone processes on MSG beaches. These will be addressed in three thesis chapters, treating: (1) vertical bed pore-pressure response under wave loading; (2) beach cusp morphodynamics; and (3) grain size-segregation during beach cusp formation. Substantial progress has been made on (1) and (2), respectively representing manuscripts, one published and one in preparation. 
%%%%

% General:
The purpose of this thesis is to contribute knowledge and understanding of morpho-sedimentary, hydrodynamic, and hydraulic processes on mixed sand-gravel beaches over sub-tidal timescales, while exploring remote and \textit{in situ} methods for obtaining field observations at energetic mixed sand-gravel beach sites. This is achieved with an emphasis on field observations, through a series of studies at Advocate Beach, Nova Scotia, Canada. The objectives, with associated questions, are as follows:

\begin{enumerate}

  % Obj. 1: hydraulics, pressure sensors  
  \item Characterise the depth dependence of the phase lag and attenuation of oscillatory pore water pressures induced by surface gravity waves.
  \begin{itemize}
    \item Can the pore-pressure amplitude and phase be predicted using a poro-elastic bed response model?
%    \item What are the implications of the pore-pressure response for the efficacy of observing surface gravity wave amplitude and phase with buried pressure sensors?
    \item Is the use of pressure sensors buried in the intertidal beach a viable means of accurately inferring surface wave height and phase in the surf and swash zones, where deploying \textit{in situ} instrumentation is otherwise logistically challenging?
  \end{itemize}

  % Obj. 2: cusp stuff 
  % [ contribute observations to the literature of mixed sand-gravel cusp evolution with high temporal resolution.?]
  \item Investigate the morpho- and hydrodynamic processes associated with mixed sand-gravel beach cusp evolution.
  \begin{itemize}
    \item How do mixed sand-gravel cusps fit into the existing cusp literature?
    \item What characterises the timescale of cusp evolution at Advocate Beach?
    \item What effect does large tidal range have on the timing of cusp events?
    \item What is the role of the beach surface grain size distribution in cusp morphological evolution?
    %\item Can signatures of standing edge waves be observed during cusp formation at Advocate Beach?
  \end{itemize}
  
  % Obj. 3: Sedimentology and sediment dynamics
  \item Investigate grain size dynamics and linkages to swash zone morphologic change.
  \begin{itemize}
  	\item Are there observable signatures of correlation between bed level change and the grain size distribution?
    \item Can Lagrangian tracking of gravel and cobble size-fractions help us characterise sorting processes in the swash zone?
  \end{itemize}
  
\end{enumerate}

The remainder of the thesis is organised as follows. In Chapter \ref{Chapter:Background}, some relevant background information is provided, including a general description of mixed sand-gravel beaches and their defining characteristics; a brief review of the beach cusp literature and theories for their formation; and an overview of the factors contributing to mixed sand-gravel sediment transport. A description of the study site is presented in Chapter \ref{Chapter:AdvocateBeach}, including summaries of two field campaigns, conducted in October-November of 2015 and October of 2018. The presentation of morpho-sedimentary and hydrodynamical data collected during the Advocate Beach field campaigns is presented in Chapters \ref{Chapter:PorePressure} through \ref{Chapter:MSDSwash}. In Chapter \ref{Chapter:PorePressure}, pressure data are used to assess the viability of using buried pressure sensors to accurately infer surface wave height and phase in the surf and swash zones, where deploying \textit{in situ} instrumentation is otherwise logistically challenging. Chapter \ref{Chapter:CuspDynamics} presents video-based observations of beach cusp dynamics. Chapters \ref{Chapter:MSDBeach} and \ref{Chapter:MSDSwash} present coincident observations of bed level and mean grain size at the beach surface: at the scale of the intertidal zone, using GPS and image-based survey data (Chapter \ref{Chapter:MSDBeach}), and in the swash zone, using acoustic bed level sensors and digital imagery (Chapter \ref{Chapter:MSDSwash}). In Chapter \ref{Chapter:Conclusions}, the key findings of the thesis are summarised, followed by some implications of the findings and a future work outlook.

Chapters \ref{Chapter:PorePressure} and \ref{Chapter:CuspDynamics} have been published as peer-reviewed articles \citep{Guest_Hay2017, Guest_Hay2019} in the \textit{Journal of Geophysical Reseach: Oceans} and \textit{Marine Geology}, respectively. The text and figures of each chapter have not been altered from the published versions, except where necessary for reference or navigation within the thesis. As lead author of the articles, I was responsible for the development and implementation of the analysis methods and the interpretation of the results. I also developed the figures, and wrote the corresponding manuscripts. Alex Hay, co-author of both articles, provided guidance in the research, and contributed to the editing of the manuscripts. I shared responsibility with Alex Hay and Richard Cheel (lab technician) for the experimental design.

Chapters \ref{Chapter:MSDBeach} and \ref{Chapter:MSDSwash} represent manuscripts in preparation for publication. As such, they emulate the article format of the previous two chapters, i.e., in layout and relative self-containment.

