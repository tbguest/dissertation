\chapter{Timescales of beach cusp evolution on a steep, megatidal, mixed sand-gravel beach}\label{Chapter:CuspDynamics}

%\begin{abstract}
%
%Field observations are presented of the morphological evolution of beach cusps on a 1:10 slope, megatidal (8-12 m range), mixed sand-gravel beach at the head of the Bay of Fundy, Nova Scotia. Cusps had mean wavelengths of 3-6 m and displayed pronounced horn/bay sediment size segregation, with sand-sized material in the bays and gravel-sized material in the horns. Cusp occurrence was limited to the upper third of the beach face. Shoreline position during the tidal cycle was estimated at three minute intervals from time-averaged video imagery. Three cusp events are examined in detail, two exhibiting pronounced topographic relief, and the third demonstrating sensitivity of the rate of cusp evolution to the beach surface grain size distribution. Forcing conditions were weak, with significant offshore wave heights of 10-20 cm and peak periods of 4-7 s. Relict cusp morphology was inundated with the rising tide and destroyed or reworked during high tide. New cusps formed during the falling tide through a combination of accretion at the horns and erosion in the embayments. Timescales of growth and decay were short, ranging from 10 to 30 mins. The location and dynamics of cusp horns appeared to depend on the high water line and its location relative to any pre-existing cusp morphology. The apparent sensitivity of cusp formation timescales to the local grain size distribution suggests that size segregation is intrinsic to the process of mixed sand-gravel cusp evolution.
%
%\end{abstract}

This chapter was first published in \textit{Marine Geology}\footnote{\textbf{Guest, T. B.}, and A. E. Hay, Timescales of beach cusp evolution on a steep, megatidal, mixed sand-gravel beach, \textit{Marine Geology}, 416, 105984, 2019.}.


\section{Introduction}

Beach cusps are spatially periodic sedimentary features common on beach foreshores. They consist of seaward-pointing sequences of ridges (horns), separated by topographic depressions (embayments), with quasi-regular spacings varying from $O$(10$^{-1}$) to $O$(10$^{2}$) m \citep{Coco_etal1999}. Cusps are generally acknowledged to be swash-generated features, though recent studies have drawn a distinction between beach cusps of short wavelength ($<$ 20 m), and the longer-wavelength large beach cusps (LBC) which may be tied to surf- as well as swash-zone morphodynamic processes due to their larger cross-shore extent \citep{Garnier_etal2010}. Cusps have been observed to form under a wide range of conditions and beach types, though are acknowledged to form most readily during low energy, shore-normal incident wave forcing on reflective, medium- to coarse-grained beaches \citep{Holland1998}. 

Despite their long history of study, focus on beach cusps has generally been limited to beaches comprised of a single sediment type (i.e., sand or gravel), with little emphasis on beaches composed of mixed sand and gravel (MSG). \citet{Nolan_etal1999} presented the only study we are aware of from the last 20 years treating mixed sand-gravel cusps specifically. Their focus was on cusp morphometry. The dynamics of the formation process were not investigated. 

Sediment dynamics on MSG beaches in general are less well studied than sand or gravel beaches, but have received increased attention in recent years owing to the increased use of sand and gravel mixtures in coastal engineering schemes, e.g., involving beach replenishment \citep{Mason_Coates2001}. Also, the relative efficiency of gravel, cobble, and mixed beaches for dissipating wave energy provides a natural form of coastal defence. 

The tendency for different sediment grain sizes and types on mixed beaches to self-segregate under some conditions is also of great interest, and has recently garnered attention in the context of other nearshore bedforms and processes \citep[e.g.,][]{Murray_Thieler2004, vanOyen_etal2010, Hay_etal2014}. Beach cusps are perhaps the most widely recognisable feature demonstrating size-segregation in the nearshore, having long been associated with grain size sorting between bays and horns \citep[e.g.,][]{LonguetHiggins_Parkin1962}. 

The modern (\textit{ca}. 1970s to present) cusp literature has been largely focused on understanding the mechanism(s) responsible for cusp formation and establishing predictive relationships between cusp spacing and hydrodynamic forcing. Efforts to establish predictive relationships between cusp spacing and the wave forcing were initially based on the concept of a hydrodynamic template, provided by standing low mode subharmonic or synchronous edge waves, whose spatial structure would become imprinted on the underlying sediment \citep{Guza_Inman1975, Guza_Bowen1981, Holman_Bowen1982}. Later models are based on free, self-organising behaviour, characterised by asymmetric feedbacks between hydrodynamics and morphology \citep{Werner_Fink1993, Coco_etal2000, Coco_etal2004, Sunamura2004, Dodd_etal2008}. Though both types of theory have been used to predict the spacing of beach cusps with reasonable skill in some circumstances \citep{Coco_etal1999}, the self-organisation mechanism has been found to be more widely applicable \citep{Coco_Murray2007}. 

Focus on cusp initiation mechanisms and predictors of spacing has given way in the last decade to renewed emphasis on cusp morphology and characterisation of dynamics \citep[e.g.,][]{Almar_etal2008, VanGaalen_etal2011, Vousdoukas2012, Poate_etal2014}. However, uncertainties remain about the nature of beach cusp morphological evolution. These include: varied and, in some cases, inconsistent conclusions regarding factors influencing cusp evolution timescales; the importance and/or role of grain size sorting; and a lack of consensus on the roles of accretion and erosion. 

There has been little emphasis on cusp formation timescale. The data indicate that beach cusp lengthscale and formation time are related. The observations by \citet{Komar1973} of beach cusps on a sandy lake shore often provide a lower limiting case in the literature for both cusp spacing and formation time -- as short as 11 cm and 10 mins, respectively. However, these observations are largely qualitative, and do not document the morphodynamic evolution of the cusp fields. \citet{LonguetHiggins_Parkin1962} reported similarly short formation timescales from a wave tank experiment, in which 1.5 s waves of $<$ 2 cm height led to the onset of cusp formation (mean wavelength of \textit{ca}. 50 cm) in fine sand within a few minutes. Cusp formation in the field is generally reported to occur over much longer timescales -- on the order of tidal time (hours to days) rather than minutes. \citet{Holland1998}, reporting on 57 cusp events spanning nearly a decade, observed that the transition from indistinguishable to well-formed cusps on a medium sand beach generally occurred over three consecutive days, with no indication of well-developed cusps forming in less than a few hours. The mean cusp wavelength was 25 m. \citet{Poate_etal2014} reported 25 m wavelength cusp formation on a high-energy, pure gravel beach occurring in under 12 hours. \citet{Garnier_etal2010} established formation timescales for large beach cusps (\textit{ca}. 30 m wavelength) on a medium-coarse sand beach that depended upon the forcing energy. Their shortest reported formation times (3-6 hours) occurred when offshore wave heights exceeded 2.5 m, 9-12 hours for wave heights between 1.5 and 2.5 m, and 2-3 days for wave heights less than 1.5 m. Model results from \citet{Dodd_etal2008} indicated a dependence of cusp growth rate on incident wave period, with growth rates increasing as period decreased. 

The relationship between timescale and beach composition is largely uninvestigated. Numerical modelling by both \citet{Coco_etal2000} and \citet{Dodd_etal2008} indicated a relationship between a dimensional transport constant in their sediment flux parameterisations and the number of swash cycles required for cusp formation. Fewer cycles were required when the transport constant was increased. As noted by \citet{Dodd_etal2008}, it is possible to approximate this constant as a function of $D_{50}$ using the Van Rijn sediment transport equation. The implication is that cusps would form more rapidly on beaches with a larger mean grain size. \citet{Dodd_etal2008} also concluded from their model results that beach permeability favours cusp development by enhancing the feedback mechanism (i.e., infiltration through the horns decreases the backwash volume, therefore decreasing offshore transport). For the case of mixed sand and gravel beaches, the gradation of fine and coarse material between bays and horns would be expected to further enhance feedbacks by increasing the asymmetry between erosion/depositional processes in bays and horns. Indeed, \citet{LonguetHiggins_Parkin1962} concluded from field observations that cusps form most easily given a vertical stratification of material: coarse sediments sitting atop a well-mixed, ``impermeable'' layer are more readily mobilised due to the reduced fluid infiltration. The heaping of sediments atop the horns makes them more permeable than the thinner layer in the bays, and therefore less subject to erosion. 
 
In their gravel beach review paper, \citet{Buscombe_Masselink2006} emphasised the importance of sediment characteristics in gravel beach processes, suggesting that morphodynamics alone are an inadequate descriptor of process on gravel (including MSG) beaches. They suggested a morpho-sedimentary dynamics framework, in which properties of the grain-size distribution (e.g., spatial and temporal heterogeneity, sorting) exert a fundamental control on morphological evolution, for further research and discussion around gravel beach processes. The role of size sorting in the process of cusp formation was investigated by \citet{VanGaalen_etal2011} for the case of a sandy, microtidal beach on the Atlantic coast of Florida. Using regular morphological observations from a terrestrial laser scanner and discrete surficial sediment sampling, they found no significant correlation between grain size and position between horn and embayment. This led them to suggest that horn coarsening and embayment fining occurs after cusp development, making them inherently different from other sorted bed forms whose formation depends on differential transport characteristics for different grain sizes. However, the sediment size distribution at their site was relatively narrow compared to MSG beaches, with typical grain sizes ranging from 0.17 to 0.35 mm: i.e., fine to medium sand. 

Results regarding the relative importance of accretion and erosion are varied. Instances of cusps formation have been reported for conditions dominated by sediment accretion, erosion, or a combination of both \citep[e.g., ][]{Antia1987, Masselink1997, Almar_etal2008}. In addition to their conclusions regarding size-sorting, \citet{VanGaalen_etal2011} demonstrated that the accretive/erosive properties of cusps observed on a sandy, microtidal beach were dependent upon the position of the cusps relative to larger-scale coastal morphology. The reader is referred to the van Gaalen et al. paper for a more complete review of past results treating cusp horn/bay sedimentation.
 
Video-based techniques are well-suited for observing short-term morphological change at the shoreline \citep[e.g.,][]{Holman_Stanley2007}. Video-based shoreline tracking has been employed to study beach cusps by \citet{Holland_Holman1996}, \citet{Holland1998}, \citet{Sunamura_Aoki2000}, \citet{Almar_etal2008}, \citet{Garnier_etal2010}, and \citet{Vousdoukas2012}. With the exception of \citet{Sunamura_Aoki2000}, all these studies were on sandy beaches. \citet{Almar_etal2008} and \citet{Vousdoukas2012} used shoreline tracking to characterise the evolution of cusp fields, and so are particularly relevant here. Both tracked daily changes in cusp horn positions over periods of months. \citet{Sunamura_Aoki2000} used images from a suspended camera to chronicle the formation of beach cusps on a steep gravel beach. Video-based methods are especially appealing for gravel or mixed sand-gravel environments where energetic shorebreaks impose challenges to observation by limiting the utility of \textit{in situ} instrumentation.

Here we present video-based observations from a mixed sand-gravel beach in Advocate, Nova Scotia, Canada (Fig. \ref{fig:Advocate_map}). The purposes of this paper are to investigate (1) the timescales of cusp evolution at Advocate Beach using high temporal resolution observations; and (2) the role of size-segregation in the cusp formation process on a mixed sand and gravel beach. The steep slope, large tide range, and very broad sediment size distribution at Advocate Beach enable closer investigation of processes influencing cusp evolution timescales, and the role of grain size sorting.

\begin{figure}[tbp] % from /home/tristan/Experiments/Advocate_2015/reports/Cusp_PaperII/figures
%%% 	\includegraphics[width=\columnwidth]{figures/submission/png/Advocate2015.png}
	\includegraphics[width=\columnwidth]{figures/chapter2/Advocate_map_with_inset.png}
	\caption[Map: Advocate Beach site]{Map indicating the location of Advocate Beach (red box), Nova Scotia, Canada, near the head of the Bay of Fundy.}
	\label{fig:Advocate_map}
\end{figure} 

The paper is organised as follows: Regional context is provided in Section \ref{section:Context}. Experimental setup and analysis methods are outlined in Section \ref{ch2:Methods}. Section \ref{ch2:Results} contains results, beginning with experiment scale forcing and response, followed by detailed summaries of individual cusp events, including the two events which were characterised by pronounced topographic relief, and a third event which fortuitously captured an instance of differing surficial sediment properties in the cross-shore, providing insights into the relationship between size distribution and cusp formation timescale. A discussion of the results is presented in Section \ref{ch2:Discussion}, with particular emphasis on horn dynamics, and the influence of a wide grain size distribution on cusp evolution timescales. 


\section{Regional Context}\label{section:Context}

The study site is Advocate Beach, Nova Scotia (Fig. \ref{fig:Advocate_map}), a mixed sand-gravel barrier beach positioned near the head of the Bay of Fundy. It separates the headlands of Cape Chignecto to the west and Cape D'Or to the southeast, and is 5 km long with a nearly linear shoreline. The beach faces southwest, exposing it to the prevailing winds during summer months, as well as the direction of maximum fetch--roughly 500 km along the Bay of Fundy and into the adjacent Gulf of Maine. From the south and west-northwest, the beach is more fetch-limited.

Tides in the region are semidiurnal, and at Advocate Beach have a range of 8-12 m. According to \citet{Levoy_etal2000}, beaches with a tidal ranges in excess of 8 m may be classified as megatidal rather than macrotidal. The intertidal beachface has a uniform slope of roughly 1:10 that has persisted nearly unchanged through surveys conducted since the early 1980s \citep{Hay_etal2014}. During spring tides, the beach crest to low-water distance is approximately 100 m. The beach sediments range in size from medium sand to cobbles and boulders greater than 20 cm in diameter, generally well-mixed both across-shore and at depth. From the lower beach face to beneath lowest low water, the beach composition transitions to cobble and boulder-sized material, and the slope decreases substantially.

The wave climate at Advocate Beach is generally dominated by locally-generated wind waves. Peak wave periods during a 2-week experiment in spring 2012 ranged from 5 to 7 seconds \citep{Hay_etal2014}. During periods of weak local wind forcing, low-amplitude swell waves propagating from outside the Bay of Fundy may dominate the wave energy spectra \citep[e.g.,][]{Guest_Hay2017}. The combination of a steep beach slope and typically short period, wind-generated incident waves result in a narrow surf zone and energetic shore break for offshore significant wave heights of \textit{ca}. 0.5 m and larger \citep{Hay_etal2014}. Wave incidence is predominantly shore-normal, in correspondence to the prevailing southwesterly wind direction and fetch limitation.

Episodes of cusp formation occur frequently at Advocate Beach, most commonly on the upper beach face (e.g., Fig. \ref{fig:cusp_bay_overwash}). The cusps generally appear as well-organised sediment structures near the high tide level, and as the tide recedes often extend tens of metres seaward, forming cross-shore bands of loose gravel and cobbles separated by sandy embayments. The beach experiences changes in forcing conditions on tidal time scales, which is reflected in daily changes in surficial sediment composition and topography. Consequently, cusp episodes at Advocate Beach are generally limited to a single tide. Sediment sorting has also been observed to follow energetic wave events in the form of widespread fining of the surficial material \citep{Hay_etal2014}.

\begin{figure}[tbp]%/home/tristan/Documents/Projects/AdvocateCuspMorphology2017/code/cusp_wavelength_so_edge_wave_comparison.m
 	\includegraphics[width=\columnwidth]{figures/chapter2/cusp_bay_overwash.png}
	\caption[Photograph of well-sorted beach cusps]{Well-sorted beach cusps observed during a 2012 field experiment at Advocate Beach. Note the sandy overwash (red arrow) at the top of several cusp embayments, atop the high tide berm. A discussion of mixed sand-gravel cusp formation mechanisms can be found in Section \ref{subsec:Timescales}.}
	\label{fig:cusp_bay_overwash}
\end{figure}



\section{Methods}\label{ch2:Methods} 

\subsection{Experiment Overview}\label{subsec:Overview}

Data were collected during a field experiment at Advocate Beach, between 21 September and 8 October (yearday 264-281) 2015. Video imagery was recorded daily using a Raspberry Pi single-board computer and 5 megapixel camera module, both installed atop a 5.3 m tower on the beach crest, facing southward. The camera field of view (shown in Fig. \ref{fig:beach_profile}) contained the intertidal zone, and spanned approximately 30 m of shoreline at spring high tide and nearly 100 m at low tide. Video was recorded with a resolution of 1920x1080 pixels at 2 Hz for roughly 6 hours per tide during daylight, corresponding when possible with the period before, during, and after high tide.  

\begin{figure}[tbp]% /home/tristan/Documents/Projects/AdvocateCuspMorphology2017/code/wavestats_slope_cusps.m
\begin{center}
%%% 	\includegraphics[width=0.75\columnwidth]{figures/submission/png/profile.png}
 	\includegraphics[width=0.75\columnwidth]{figures/chapter2/Advocate2015_instrument_laydown_sanscusps.png}
	\caption[Beach profile with instrument locations during 2015 Advocate Beach experiment]{(a) Aerial view of Advocate Beach, with instrument locations during the 2015 experiment shown. The location of the camera tower is indicated by the blue dot, the camera's approximate field of view (FOV) by the light-shaded triangle, and the pressure transducer (PT) by the red dot. (b) Profile of the Advocate Beach study site from yearday 274 of 2015. Dashed black lines indicate the spring/neap tidal range observed during the experiment. The yellow line (slope of 0.12) indicates the region where cusps were typically observed, corresponding roughly to the upper one-third of the beach face.}
	\label{fig:beach_profile}
\end{center}	
\end{figure}

Incident wave conditions were monitored using an internally logging RBR Duo pressure sensor, deployed on an above-bed frame near lowest low water (Fig. \ref{fig:beach_profile}), sampling continuously at 6 Hz between yeardays 267 and 281. 

Additional pressure measurements were made using high resolution (0.02 kPa) MS5803-14BA digital pressure sensors, arranged in longshore, cross-shore, and vertical arrays. The longshore and cross-shore arrays, spanning 16 m and 9 m respectively, were designed for observing low mode edge waves, and the vertical array, spanning the upper 50 cm of the sediment column, for observing sediment pore water pressure. The pressure array observations are not treated in this chapter. Results from the vertical array are presented in Chapter \ref{Chapter:PorePressure}.

A Hemisphere Model S320 RTK (real time kinematic) GPS was used to survey the beach profile at each daylight low tide, and to obtain ground control points for image rectification. A typical beach profile is shown in Fig. \ref{fig:beach_profile}. 


\subsection{Shoreline Tracking}\label{subsec:Tracking}

Video imagery was used to produce time exposure-type images for analysis of shoreline position. The image products were georectified onto a plane approximating the beach surface, and rotated into longshore and cross-shore coordinates. The georectification step was completed using the g\_rect toolbox for MATLAB \citep[e.g.,][]{Bourgault2008, Richards_etal2013}.  Thirty-four ground control points captured on yearday 281 were used for the rectification. The same coordinate transform was used for all image data. Lens distortion was mitigated using a second-order polynomial correction implemented within the g\_rect algorithm.

%The pixel timestack images used to track the instantaneous position of the water's edge were composed of cross-shore pixel transects (i.e. fixed $x$-coordinate) from the 2 Hz images, stacked consecutively through time. The cross-shore pixel transects were extracted at five alongshore locations separated by 4 m. Local maxima and minima of the cross-shore swash front coordinate were manually identified in each timestack image. Manual extraction of extrema was employed in lieu of an automated shoreline detection algorithm, which performed poorly due to the coarse timestack resolution resulting from the distance of the camera from the swash zone and the low sampling frequency. Horizontal swash excursion and swash period were estimated from the time series of extrema. Excursion was computed as the mean cross-shore distance from each local minimum to the following runup maximum. Swash period was computed as the mean time interval between consecutive runup maxima. Each time series spanned 12 mins. 

Time-averaged images were used to estimate the shoreline position and track the evolution of cusps on tidal and subtidal timescales. Shoreline contours were estimated from 3-minute (360 frame) average images during periods of interest. In this study the term `shoreline' refers to a beach contour defined by the $x$-$y$ position of the water's edge (i.e., the intersection of the sea surface and the beach, for fixed vertical coordinate), as determined from the time-averaged imagery. There are many techniques available for estimating shoreline position from time-averaged video, most commonly employing time exposure \citep[TIMEX,][]{Plant_Holman1997, Pearre_Puleo2009}, variance \citep{Holland_etal1997, Pearre_Puleo2009}, and colour balance imagery \citep{Almar_etal2008}, all of which utilise peaks in intensity or intensity gradients associated with the shorebreak to infer shoreline position. A quasi-variance technique similar to that used by \citet{Vousdoukas_etal2011} was found to give stable estimates over the widest range of conditions. This method uses differences in pixel intensity between each image and the preceding image summed over a desired image sequence to illuminate the regions of greatest change. The process is summarised in Fig. \ref{fig:extraction_workflow}. In keeping with the terminology of \citet{Vousdoukas_etal2011}, we refer to images produced in this way as SIGMA images. 

\begin{figure}[tbp]
\begin{center}
 	\includegraphics[width=0.75\columnwidth]{figures/chapter2/extraction_workflow_redline.png}
	\caption[Shoreline extraction from digital imagery]{(a) Still frame captured by the tower-mounted camera during event 281A on ebb tide. (b) Orthorectified SIGMA image corresponding to the same time period as panel (a), with the estimated shoreline position overlaid in red. Lighter shades represent regions of greatest change in pixel intensity over the 3 minute (360 frame) averaging period.}
	\label{fig:extraction_workflow}
\end{center}
\end{figure}

The algorithm for estimating shoreline position was designed as follows: Pixels outside a pre-defined region of interest were masked prior to georectification to reduce computation time. Each georectified SIGMA image was smoothed using a 2-dimensional, 3 point mean filter (0.15 m width across-shore, 0.24 m width alongshore) to remove high spatial frequency changes in pixel intensity. Each cross-shore pixel transect was normalised by the maximum intensity value in the transect, with resultant values exceeding a predefined intensity threshold being set to 1, and all others set to 0. The resulting binary image contained `structures' demarcating regions of high and low variability in pixel intensity. Unity-valued `islands' containing fewer than 100 pixels were removed. The shoreward edge of the unity-valued region associated with the surf and swash zone -- i.e., the first nonzero element of each cross-shore pixel column -- was defined as the shoreline. The intensity threshold was set at 0.35 for all cases.

Averaging periods of 10 minutes have been previously suggested to be appropriate for similar analyses on dissipative beaches exposed to swell band incident waves \citep{Plant_Holman1997}. Mean swash periods during cusp events at Advocate Beach were observed to fall within the range of 4-6 s. The short swash periods, coupled with the rapid rate of change of water level, led to the choice of a shorter 3-minute averaging period, which typically spanned 30-45 swash cycles.

Video-tracked shorelines were used to estimate bulk cusp properties. As indicated schematically in Fig. \ref{fig:cusp_schematic}, these variables were wavelength $\lambda_c$, the mean longshore distance between cusp horns; and cross-shore cusp amplitude $A_y$. $A_y$ is a measure of cusp prominence defined as half the difference between the mean cross-shore coordinates of the cusp horns $y_h$, and cusp bays $y_b$, for a given shoreline contour. The shoreline contours were smoothed using a 7 point mean filter (0.56 m width) prior to determining wavelength and amplitude. Cusps were distinguished from other topographic irregularities by defining a minimum cross-shore amplitude threshold of 7.5 cm for the local maxima (cusp horns). 

\begin{figure}[tbp]
\begin{center}
 	\includegraphics[width=0.75\columnwidth]{figures/chapter2/figure_cusp_schematic_image.png}
	\caption[Beach cusp dimensional measures]{(a) A schematic showing well-sorted beach cusps in plan view. The solid black line represents a fixed-elevation contour (i.e., the shoreline). Dimensions measured were cusp wavelength, $\lambda_c$, and cross-shore cusp amplitude, $A_y$, which was computed as half the difference between the mean cross-shore horn coordinate, $y_h$, and mean bay coordinate, $y_b$. (b) A photograph of well-organised cusps at Advocate Beach, taken from atop the spring tide berm during event 276B. The two horns are separated by approximately 6 m. Note the strip of fine material (grey-coloured) landward of the high water line, indicated by the clumps of flotsam (mainly seaweed).}
	\label{fig:cusp_schematic}
\end{center}	
\end{figure}

To infer bed level change (i.e., accretion or erosion), topographic mapping of the intertidal zone was carried out as an extension of the shoreline tracking method outlined above. With the low-tide, frame-mounted pressure sensor providing a vertical coordinate, video-tracked shoreline contours from respective flood and ebb tides were interpolated onto a regular 0.25 m square grid using the quadratic loess method outlined by \citet{Plant_etal2002} (loess filter half-width of 1 m).   



\section{Results}\label{ch2:Results}

\subsection{Forcing}

The hydrodynamic forcing conditions during the experiment, determined from the pressure sensor at the beach toe and representing ``offshore" wave conditions in the present context, are presented in Fig. \ref{fig:wavestats_slope}. Periods of low wave energy (yeardays 265-270 and 275-277), owing both to light winds and to winds from the N-NE (offshore), were interspersed with wind events from the SW, leading to the steep, locally generated wind and swell waves that are characteristic of the Bay of Fundy. The first, and strongest, of the wind events lasted two days (yeardays 270-271) with peak significant wave heights approaching 1.5 m. Two other SW wind events occurred on yeardays 274 and 281. During the period from yearday 277 to 280, wave energy spectra were dominated by long period (15-20 seconds), low amplitude (0.05-0.15 m) swell waves generated by Hurricane Joaquin which passed well to the south of Halifax (Fig. \ref{fig:Advocate_map}). The mean peak period, computed for storms ($4\sigma_{p} > 0.15 $ m) and intervals between storms, was 7.6 seconds and 9.7 seconds, respectively. Mean peak wave periods measured during two previous experiments at Advocate Beach were 5-7 seconds \citep[e.g.,][]{Hay_etal2014}. 

\begin{figure}[tbp]
 	\includegraphics[width=\columnwidth]{figures/chapter2/wavestats.png}
	\caption[Wave, tide, and cusp occurrence data from Advocate 2015 experiment]{Forcing conditions during the experiment, registered by the pressure sensor near lowest low water (Fig. \ref{fig:beach_profile}). (a) Tidal elevation, $h_{0}$. (b) Significant wave height, computed as 4$\sigma_p$. (c) Peak wave period, $T_{p}$. Shaded regions in panels b and c indicate tides during which cusps were identified.}
	\label{fig:wavestats_slope}
\end{figure}

In general, cusp events occurred during periods of low wave energy, with offshore significant wave heights of $O$(10 cm) and wind-band peak periods of 4-6 s. Incident waves were short-crested. Offshore wave heights of 10 cm correspond to break point wave heights of \textit{ca}. 17 cm based on the breaker criterion of \citet[][p. 115]{Dean_Dalrymple1984}. Wave incidence angle was not measured, but was generally observed to be shore-normal, owing to the prevailing southwestly wind and fetch limitation.

Tidal currents were not measured, but have previously been observed to be weak near the shoreline. \citet{Hay_etal2014} reported mean alongshore currents of less than 0.25 m s$^{-1}$, recorded at a mid-tide platform roughly 0.5 m above the bed. The maximum values occurred during high tide, when the water depth was greatest at the location of the instrument platform. The alongshore currents were much weaker--less than 0.08 m s$^{-1}$--immediately after the shorebreak passed the instrument platform on the rising tide, suggesting that the tidal currents are likely negligible in comparison to surf- and swash-driven flows. 


\subsection{Cusp Occurrence}

Over the experiment's two week duration, nine cusp events were identified (Figs. \ref{fig:wavestats_slope}b, \ref{fig:wavestats_slope}c), five of which occurred during daylight. Cusp events are herein defined to include instances of incipient cusp formation: i.e. longshore topographic undulations and sediment size-sorting were apparent, but where stable, uniform horn spacings may not have emerged. 

\begin{table}[tbp!]
	\caption[Beach cusp dimensions by tide: Advocate 2015 experiment]{Summary of beach cusp dimensions for the five events observed during daylight hours. Cusp wavelength, $\lambda_c$, and cross-shore amplitude, $A_y$, were computed from contiguous cusps near the time of peak pattern regularity for each event.} %[ADD SWASH DATA]}
	\label{table:cusps}
	\centering
	\begin{tabular}{ccc}
		\hline
		Event (yd, tide) & $\lambda_c$ (m) & $A_y$ (m)\\
		\hline
		273, B & 3.88 $\pm$ 1.10 & 0.32\\
		274, B & 3.13 $\pm$ 0.54 & 0.20\\
		275, B & 2.75 $\pm$ 0.63 & 0.13\\
		276, B & 6.29 $\pm$ 0.39 & 0.61\\
		281, A & 3.79 $\pm$ 0.53 & 0.46\\
		\hline
% 		\multicolumn{4}{l}{$^{*}$Computed for $D<22.4$ mm due to a higher percentage of coarse material.}
	\end{tabular}
\end{table}

Cusp presence during these events was limited to the upper third of the beach face, as indicated in Fig. \ref{fig:beach_profile}. The dimensions of the cusps observed during the five daylight events are summarised in Table \ref{table:cusps}. Cusps generally formed near the high water line (HWL) during high tide, when the rate of change of shoreline position was greatly reduced, and continued through early ebb, resulting in bands of well-sorted sediment extending up to 20 metres seaward. No well-formed cusps were observed at midtide, during which the rate of change of water level was as high as 3 metres per hour, or 1.25 metres across-shore between 3 minute SIGMA video images. The presence of newly-formed cusps was rarely observable in the SIGMA image shorelines during flood tides, due to the rapid shoreward advance of the swash zone. No cusp formation was observed at the low tide shoreline.


\subsection{Cusp Evolution}  

Of the five identified cusp events that occurred during daylight, three are examined in detail: (1) the first tide of yearday 281; (2) the second tide of yearday 276; and (3) the second tide of yearday 275, hereafter referred to as 281A, 276B, and 275B. The most pronounced horn-bay topographic relief was exhibited during 281A and 276B; 275B was characterised by incipient cusp formation and apparent sensitivity of formation rate to available grain size. The remaining two events were characterised by early-stage cusp formation, and are not examined here. During events 281A and 276B, existing cusps from the previous tide were inundated during flood tide and either readjusted or were planed nearly flat during high tide, with new cusps formed during the subsequent ebb. Timestack plots of the shoreline contours encompassing each of events 281A and 276B are shown in Fig. \ref{fig:shoreline_stacks_276B_281A}. In both cases, emergent horn positions appear to have been influenced by the pre-existing morphology.

\begin{figure}[tbp] % from /home/tristan/Documents/Projects/AdvocateCuspMorphology2017/code/filter_and_plot_extracted_shorelines.m, via shoreline_stacks_276B_281A.svg
 	\includegraphics[width=\columnwidth]{figures/chapter2/shoreline_stacks_281A_276B.png}
	\caption[Timestacked shoreline contours]{Timestack plot of the shoreline contours extracted from SIGMA images at 3-min intervals during cusp events 281A (left) and 276B (right). Orange dots mark local maxima exceeding a cross-shore amplitude threshold of 7.5 cm. Persistent maxima correspond to the locations of cusp horns. High tide was at approximately 12:30 during event 281A, and 19:45 during event 276B. The cross-shore spatial scale is indicated by the vertical black bars.}
	\label{fig:shoreline_stacks_276B_281A}
\end{figure}


\subsubsection{Event 281A}\label{281A}

Fig. \ref{fig:cusp_timeseries_281A} shows the forcing conditions and cusp parameters associated with event 281A. Significant offshore wave heights were 11-16 cm, corresponding to wave heights of 18-25 cm at the break point. Peak incident wave periods were 4-5 s. Relict cusps from the previous tide had a mean wavelength of 5 m and mean cross-shore amplitude of 0.5 m at maximum relief (Fig. \ref{fig:cusp_timeseries_281A}d-f, 12:00-12:40). The HWL did not reach the maximum shoreward extent of the relict cusps due to a decrease in mean high water (MHW) from 7.5 m during the previous tide to 7.3 m. During high tide, local adjustments to cusp spacing were observed, with some horns maintaining their alongshore positions and others shifting or branching until a new mean wavelength of 4 m was reached (12:40-13:30). The cross-shore amplitude of the cusps was reduced to a minimum of 0.2 m during the adjustment period, and returned to a maximum of 0.5 m. The transition time between maximum cross-shore amplitudes was \textit{ca}. 1 hour. The newly formed cusp field had regular horn spacing, with a standard deviation of 0.4 m during and immediately following maximum cross-shore amplitude (Fig. \ref{fig:cusp_timeseries_281A}d, 13:50-14:30). Cusp formation continued down the foreshore with the outgoing tide, resulting in well sorted bands of gravel and cobbles extending over 20 m seaward of the high tide level. 

\begin{figure}[tbp]% /home/tristan/Documents/Projects/AdvocateCuspMorphology2017/code/filter_and_plot_extracted_shorelines.m
 	\includegraphics[width=\columnwidth]{figures/chapter2/cusp_timeseries_281A.png}
	\caption[Cusp horn dynamics: yearday 281]{Time series of hydrodynamic conditions and morphodynamic response during cusp event 281A. (a) Water depth, $h$, registered by the low-tide pressure sensor. The colourscale indicates the cross-shore shoreline coordinate, estimated using the water depth $h$ and the mean beach slope ($\beta_s=0.12$), and translated to conform to the locally defined coordinate system. (b) Significant wave height, computed as $4\sigma_p$. (c) Peak wave period, $T_p$. (d) Longshore cusp horn coordinates, defined as any maxima in the video-tracked shoreline contours exceeding a cross-shore amplitude threshold of 7.5 cm. The cross-shore coordinate colourscale is the same as in panel (a). (e) Mean beach cusp wavelength, $\lambda_c$, computed as the mean separation between the maxima plotted in panel (d). The gray-shaded region indicates $+/-$ one standard deviation from the mean. (f) Cross-shore cusp amplitude, $A_y$. The vertical black line in all panels indicates high tide.} %[Note: from OSM practice talk: RIchard suggested colorscale indicating dy/dt rather than y.]}
	\label{fig:cusp_timeseries_281A}
\end{figure}

\subsubsection{Event 276B}\label{276B}

Forcing conditions and cusp parameters during event 276B are presented in Fig. \ref{fig:cusp_timeseries_276B}. Significant wave heights at the beach toe were 8-12 cm, corresponding to break point wave heights of 15-21 cm and peak periods of 5-6 s. During flood tide, the relict cusp topography from the previous tide was intersected by the advancing shoreline. These relict cusps had a mean wavelength of 4 m and peak cross-shore amplitude of 0.5 m (Fig. \ref{fig:cusp_timeseries_276B}d-f, 18:50-19:20). MHW during the previous tide was 8.1 m. The existing cusps were overtopped during high tide of 276B, during which MHW reached 8.2 m. The cusps were planed almost flat within \textit{ca}. 10 mins (19:20-19:30). The new cusp field emerged equally rapidly after high tide, and had a mean wavelength of 6 m and peak cross-shore amplitude of 0.8 m (20:00-20:15). The new cusp horns formed in locations where relict horns existed. The transition time between cusp fields, from maximum old to maximum new cross-shore amplitude, was \textit{ca}. 50 minutes (Fig. \ref{fig:cusp_timeseries_276B}f).

\begin{figure}[tbp] % /home/tristan/Documents/Projects/AdvocateCuspMorphology2017/code/filter_and_plot_extracted_shorelines.m
 	\includegraphics[width=\columnwidth]{figures/chapter2/cusp_timeseries_276B.png}
	\caption[Cusp horn dynamics: yearday 276]{Time series of hydrodynamic conditions and morphodynamic response during cusp event 276B. See Fig. \ref{fig:cusp_timeseries_281A} caption for panel descriptions.}
	\label{fig:cusp_timeseries_276B}
\end{figure}

Maintenance of the high-tide cusps did not continue as the water level receded during ebb. After the shoreline reached roughly 5 m seaward, the beach returned to an effectively planar state. It is important to note that the transition from cuspate to planar topography was not a result of cusp erasure or reworking, but rather a result of an increase in the rate of change of sea level such that continued cusp formation was not maintained.     


\subsubsection{Event 275B}\label{275B}

The cusp event 275B was preceded by a period of energetic waves on yearday 274, during which MHW reached 8.9 m. The wave event peaked early in the day and left a layer of sand on the beach surface. Though uncommon in general, this effect -- i.e., fining of the surficial sediments following energetic wave forcing -- is consistently observed at Advocate Beach, and attributed by \citet{Hay_etal2014} to the gathering of sand into the crests of metre-wavelength ripples to then be planed flat by the swash on the receding tide. During the first tide of yearday 275 (8.5 m at MHW), light, wind-band wave forcing returned the beach surface up to the high tide shoreline to a more typical, well-mixed state, leaving a strip of fine material extending 4-5 m landward of the HWL. This strip is visible in Fig. \ref{fig:cusp_schematic}b, landward of the well-sorted cusps. The incoming tide of 275B (8.6 m at MHW) inundated existing cusps at the upper extent of the well-mixed sediment region. No cusp formation was observed during high tide, while the shoreline coincided with the strip of fine material. Nascent cusp formation was observed on the ebb tide, initiation coinciding with shoreline recession to the well-mixed portion of the beach surface. This process is summarised in Fig. \ref{fig:yd275B_cusps}. The incident wave height and period measured at the beach toe ranged from 7-12 cm and 4.5-6.5 s, respectively, throughout. The possible effects of available grain sizes on cusp formation are discussed further in Section \ref{subsec:Timescales}.   

\begin{figure}[tbp]% /home/tristan/Documents/Projects/AdvocateCuspMorphology2017/code/yd275_cusp_plot.m
 	\includegraphics[width=\columnwidth]{figures/chapter2/size_dependent_sorting_with_matplot_reverse.png}
	\caption[Dependence of cusp formation on surficial grain size: timestacked shoreline contours and schematic]{Left panel: A timestack of video-tracked shoreline contours associated with cusp event 275B. Peaks (cusp horns) were marked with orange dots if more than three surpassed a cross-shore amplitude threshold of 7.5 cm. The gap of roughly 20 mins encompassing 19:00 UTC corresponds to high tide. No video was recorded during this time. The cross-shore spatial scale is indicated by the vertical black bar. Right panel: A cartoon illustrating the sequence of beach sediment and tide states leading to grain size dependent cusp formation during event 275B. No cusps formed during high tide, when only sand was present on the beach surface. Cusp formation was initiated during ebb, when the shoreline coincided with mixed sand and gravel at the beach surface.}
	\label{fig:yd275B_cusps}
\end{figure}



\subsection{Accretion and Erosion} 

In order to evaluate the contributions of accretion and erosion in the evolution of new cusps, bed levels prior to and following high tide were compared for one of the three cusp events outlined above. The video-tracked shoreline contours associated with event 281A were interpolated onto a regular grid so that changes in bed elevation between flood and ebb tide could be estimated. Each contour was assigned a $z$-coordinate using water depths registered by the pressure sensor. To maximise the in-frame shoreline at high tide, while mitigating the effects of changing light and forcing conditions on the estimated shoreline positions, the grid-domain was limited to a rectangular region near high tide with dimensions of 34 m in the alongshore and 7 m across-shore. The shoreward extent of the domain was chosen to omit the region nearest to high tide, where the topography likely remained dynamic through successive shoreline estimations due to the temporary stationarity of the sea level.% (i.e., when $|$d$h/$d$t| <$ 30 cm hr$^{-1}$).

The flood and ebb cusp topography, and the associated residual field, are shown in Fig. \ref{fig:flood_ebb_residual}. Though interpretation of the topographic change is complicated by the presence of cusps from the previous tide, it is apparent that accretion generally occurred at the cusp horns, and erosion in the embayments. The overall flood-ebb volume change in the control region was small (\textless 1 m$^{3}$), suggesting that sediment volume was conserved.

\begin{figure}[tbp] %/home/tristan/Documents/Projects/AdvocateCuspMorphology2017/code/intertidal_video_mapping_and_volume_change.m
  	\includegraphics[width=\columnwidth]{figures/chapter2/flood_ebb_residual_with_correction.png}
 	\caption[Beach surface elevation plots]{Beach surface elevations interpolated from video-derived shoreline contours and pressure data. (a) Relict topography, derived from event 281A flood tide shoreline contours, presented as deviation of the beach surface elevation from a best-fit plane. (b) New topography, derived from event 281A ebb tide contours, as deviation from the same plane used in (a). (c) Difference in bed elevation between flood and ebb.}
 	\label{fig:flood_ebb_residual}
\end{figure}

%Estimates of the rate of coarse particle accumulation at the cusp horns can be obtained from the video-derived shoreline contours. Beginning with the cusp spacing and cross-shore amplitude time series (e.g. Fig. \ref{fig:cusp_timeseries_281A}d, e), the difference in mean cross-shore shoreline coordinate between consecutive shoreline estimates can be applied with an assumption about the cusp shape to produce a volume estimate associated with a given 3-minute time interval. If the cusp shape is approximated as a sinusoid, and the contributions of accretion and erosion to cusp cross-shore amplitude are assumed equivalent, the mean rate of accumulation over the cusp formation period during ebb tide of event 281A was approximately 0.01 m$^3$ min$^{-1}$ per horn. Integrating these rates over the entire formation period gives an estimate of total horn volume -- approximately 1 m$^3$ per horn for event 281A. Comparable values were obtained using the interpolated surfaces in Figure \ref{fig:flood_ebb_residual}, within the 34$\times$7 m control region described above.  
	

\section{Discussion}\label{ch2:Discussion}

\subsection{Cusp Morphodynamics}\label{subsec:Morphodynamics}

The field measurements presented herein describe the short-term morphological evolution of beach cusps on a steep mixed sand-gravel beach. Cusps formed during periods of low wave energy, and were characterised by pronounced grain size sorting. The generation and evolution of cusps was strongly influenced by tide stage, and the location and dynamics of cusp horns (i.e., erasure/(re)emergence versus branching/merging/shifting) appeared to depend on the HWM and its location relative to any pre-existing cusp morphology. 

During instances of cusp formation, relict cusp horns branched (i.e., wavelength approximately halved), merged, or shifted in adjustment to the changes in forcing, or were planed nearly flat. The relict horn response appears to have been related to the range of the tide during adjustment: specifically, the changes in MHW within the spring-neap cycle, which resulted in up to 0.5 m differences between consecutive high tides, corresponding to a 4-5 m cross-shore change in the HWL. Because of a shoreward migration of the HWL during event 276B, the relict cusps were overtopped during high tide, and their cross-shore amplitude was rapidly reduced--the cusps appearing to be planed almost flat (see Fig. \ref{fig:cusp_timeseries_276B}d-f, \textit{ca}. 19:15-19:45). The relict cusps during event 281A were not overtopped, and the period of decrease in cross-shore amplitude coincided with localised branching and longshore adjustment of the horn positions (Fig. \ref{fig:cusp_timeseries_281A}d-f, 12:30-13:30). 

No cusp formation was observed near the low tide shoreline. There are two potential explanations: (1) The sediment size distribution tended toward increasingly larger grain sizes approaching the beach toe. Large, sometimes boulder-sized material may have stabilised the surficial sediments via high mobilisation thresholds and disruption of flow near the bed, preventing cusp formation. (2) Near saturation of the inter-granular pore spaces by ground water reduced the permeability of the beach surface relative to the upper beach, decreasing the influence of infiltration-related feedbacks.

The properties of cusp horn branching, merging, local adjustment, and interaction with relict morphology (Figs. \ref{fig:shoreline_stacks_276B_281A}-\ref{fig:cusp_timeseries_276B}) have been previously observed by \citet{Almar_etal2008} and \citet{Vousdoukas2012}, both using image-based shoreline tracking techniques. \citet{Almar_etal2008} tracked cusp horn positions (\textit{ca}. 20 m spacing) daily over periods of weeks to months on a steep, microtidal, medium sand beach. They noted that the observed merging and disappearance of cusp horns is consistent with cusp dynamics simulated by \citet{Coco_etal2000} using a model based on self-organisation. \citet{Vousdoukas2012} observed horn positions daily over a 5 month period on a steep (greater than 1:10 mean slope), mesotidal, medium to coarse sand beach. Cusps on the upper beach face had large wavelengths (mean of 50 m), and were stable, lasting from days to months. The horn positions were displaced coinciding with storm events, sometimes resulting in branching or merging, though the discrete nature of the sampling did not capture the transitions clearly. Cusps on the lower beach face were of shorter wavelength (mean of 38 m), and were more dynamic, often changing daily and lasting no longer than a week. Both \citet{Almar_etal2008} and \citet{Vousdoukas2012} emphasised the influence of existing morphology on cusp evolution, acknowledging that evolution occurred in most cases as a result of localised adjustment rather than through the emergence of a global pattern. 

The observed differences in horn response between events 281A and 276B can likely be attributed to the diffusive influence of the surf zone on the beach surface sediments, which has similarly been implicated in tidal modulation of cusp amplitude on sand beaches. \citet{Coco_etal2004} observed lower topographic relief of developed cusps during flood tide, and higher relief during ebb. They attributed this difference in relief to the smoothing of features seaward of the swash front, where the impinging surf zone during flood increasingly mobilises sediment, damping cusp growth. The observed modulation of relief was reproduced with a self-organisation-based numerical model by incorporating surf-zone morphological smoothing (sediment diffusion) and in/exfiltration. Their model results indicated that smoothing was of leading order importance in this process, with infiltration effects being of lesser importance. Our results show that cusp evolution at Advocate Beach is similarly modulated, with cusp formation being favoured during ebb tide. Though the relative importance of surf zone smoothing and infiltration is not known, one could speculate that infiltration is more important than in the sandy beach case due to higher hydraulic conductivity and grain-size sorting effects. 

\citet{Garnier_etal2010} noted that large beach cusps (25 m spacing or more) may be acted upon by surf as well as swash zone forces at different stages of the tide, due to their large cross-shore extent. Their results suggest that surf zone processes may not be purely destructive, and in some cases may reinforce cusp patterns. They ascribed this to a surf zone hydrodynamic instability independent of swash or tide processes, which they investigated using a numerical model. The cusps at Advocate Beach are morphologically quite distinct from the large beach cusps observed by \citet{Garnier_etal2010}, and though we have no observations of subaqueous morphology, it is apparent that swash processes dominated the evolution of cusps in our case--the impinging surf zone during flood appearing to act destructively as suggested by \citet{Coco_etal2004}. This is evidenced by the rapid destruction of relict cusps during event 276B, which were inundated during high tide. \citet{Almar_etal2008} similarly concluded that the overtopping of existing horns leads to decreased control exerted by the existing morphology on flow, reducing the tendency of horns to divert flow.


\subsection{Timescales and the Influence of Grain Size Distribution}\label{subsec:Timescales}

For cusp events 281A and 276B, the time between maximum relief (as determined using the cross-shore cusp amplitudes) of relict cusps and newly formed cusps was \textit{ca}. 1 hour. This period encompasses relict cusp amplitude decay and emergent cusp amplitude growth. Conceptually, cusps approaching equilibrium under steady forcing conditions would be expected to become growth-inhibited as a result of negative feedback processes. Here, the cross-shore amplitude growth and decay timescales are similar in magnitude and appear exponential in form, but both are different from the timescale representative of the process leading to equilibrium cusps (e.g., logistic growth). It is not apparent that equilibrium cusp conditions were met during the Advocate experiment. Hereafter, discussions of timescale refer to the emergence timescale associated with initial cusp growth, and the relict cusp decay timescale.  

A visual inspection of the cross-shore cusp amplitude time series for events 281A and 276B (Figs. \ref{fig:cusp_timeseries_281A}f, \ref{fig:cusp_timeseries_276B}f) indicates that the timescales associated with initial cusp growth and decay are of the order of tens of minutes. Indeed, fitting exponential curves to the initial cross-shore amplitude growth and decay periods for events 281A and 276B yields $e$-folding times between 10 and 20 minutes in 3 of the 4 cases: i.e., those qualitatively well-described by the fit. These timescales are short relative to those typically reported in the field literature. Previously identified factors influencing timescale include randomness in runup trajectory \citep{Coco_etal2000}, swash period \citep{Dodd_etal2008}, swash transport capacity -- presented in modelling terms via a sediment transport constant \citep{Coco_etal2000, Dodd_etal2008}, beach permeability \citep{Dodd_etal2008}, and energetics \citep{Garnier_etal2010}. Some of these factors are interrelated, as has been acknowledged in the context of cusp formation via (for example) surf scaling or similarity parameters. The importance of mean grain size in predicting beach cusp spacing was suggested by the model proposed by \citet{Sunamura2004}, in which cusp spacing is inversely proportional to grain size, but the possible effects of grain size and distribution on timescales of beach cusp evolution have not been discussed in the recent literature.

The general absence of cusps at mid-tide level on the beach face suggests a constraint on cusp formation imposed by the high rate of change of shoreline position, and allows us to establish an estimate of the minimum formation timescale. During maximum flood or ebb, the rate of change of water level was as much as 3 m hr$^{-1}$, or 0.007 m s$^{-1}$ in the across-shore. Swash zone widths during cusp formation of 3-5 m divided by the 0.007 m s$^{-1}$ rate of horizontal shoreline change gives a minimum timescale of $O$(10) minutes. This is close to our suggested timescale for initial cusp growth of 10-20 minutes. The often-observed banding of sorted material extending tens of metres seaward from the high tide level, but terminating before mid-tide, is consistent with these values being similar in magnitude. 

The event 275B results suggest sensitivity of timescale to the range of available grain sizes, with the presence of gravel and cobbles in addition to sand at the beach surface leading to shorter adjustment times. The morpho-hydrodynamic feedbacks often considered fundamental in the process of cusp formation appear to be bolstered by additional feedback mechanisms provided by the presence of a wide range of grain sizes and the process of size segregation: i.e., (1) increased and immediate feedback on flow due to larger volume displacement by individual grains, causing flow divergence and drag losses, and (2) increased infiltration losses through accumulation of coarse-grained material (i.e., gravel/cobble) in the horns. Sedimentary and hydraulic feedbacks are intrinsic in the morpho-sedimentary dynamics framework described by \citet{Buscombe_Masselink2006} for gravel beaches.

In the conceptual framework put forward by \citet{LonguetHiggins_Parkin1962}, cusps form most rapidly when a coarse-grained permeable surface layer sits atop a relatively well-mixed and impermeable subsurface layer. In this case, reduced infiltration through the subsurface layer would increase the ability of the swash to transport sediment. Accretion of coarse material on cusp horns would increase the thickness of the permeable layer, decreasing the strength of the backwash and thus the erodibility of the horns. Measurements of the hydraulic conductivity of Advocate Beach sediment reported by \citet{Guest_Hay2017} are consistent with the conceptual framework from \citet{LonguetHiggins_Parkin1962}: the maximum value of nearly 12$\times 10^{-4}$ m s$^{-1}$ was in the upper 5 cm of the sediment column, compared to values of less than 8$\times 10^{-4}$ m s$^{-1}$ in the underlying 45 cm.

Useful insights can also be obtained from the fluvial sediment transport literature. In a flume experiment involving mixtures of sand and gravel in unidirectional flow, \citet{Wilcock_etal2001} observed orders of magnitude increases in gravel transport rates when sand was introduced, despite a decrease in the overall proportion of gravel. They observed maximum gravel transport rates when sand content was between 15 and 27\%, and suggested that these ratios roughly corresponded to the transition from a framework- to a matrix-supported bed. This would provide support for the notion that well-sorted accumulations of coarse material are less likely to be mobilised.

Based on the above, the development of MSG cusps could progress in a sequence similar to the following (summarised in Fig. \ref{fig:MSG_cusp_schematic}): During the swash uprush phase, the larger-sized tail of the grain size distribution is deposited first, preferentially near incipient topographic highs where the fluid velocity decreases most rapidly. Incipient accumulations of gravel and cobbles protrude from the bed and are hydraulically rough, diverting and slowing flow. The divergence of flow around coarse accumulations during uprush strengthens the return flow in incipient bays during downrush. Thus, the capacity of the flow to mobilise large grains is reduced at the horns (lower fluid velocity), and increased in the bays (higher fluid velocity). Large grains in the developing bays, more readily mobilised by the strengthened seaward flow, are transported to the step region seaward of the shorebreak, where they may become source material for continued horn development as the tide recedes. Finer material may remain in suspension, become trapped in the interstitial spaces between larger grains, or be deposited at the shoreward edge of the developing embayment (see sand overwash in Fig. \ref{fig:cusp_bay_overwash}).

\begin{figure}[tbp] %/home/tristan/Documents/Projects/AdvocateCuspMorphology2017/code/intertidal_video_mapping_and_volume_change.m
  	\includegraphics[width=\columnwidth]{figures/chapter2/MSG_cusp_schematic_better.png}
 	\caption[Schematic: proposed mechanisms of mixed sand-gravel cusp formation]{Proposed mechanisms of mixed sand-gravel cusp formation, shown for incipient cusps (upper) and developed cusps (lower). Downward directed black arrows represent infiltration. White arrows indicate swash front velocity.}
 	\label{fig:MSG_cusp_schematic}
\end{figure}

In addition to preferential deposition at horns due to fluid deceleration, larger particles are more readily deposited amid accumulations of similarly sized grains due to higher angles of pivot required for mobilisation and a higher degree of grain interlocking \citep[see selection/rejection/acceptance, overpassing:][]{Buscombe_Masselink2006}. This stabilising mechanism at the cusp horns, reinforced by the hydraulic feedbacks described in the Longuet-Higgins and Parkin framework above, is comparable to armouring phenomena described more generally for sediment sorting and zonation on gravel beaches, river beds, and spits \citep[e.g.,][]{Isla1993}. Conversely, coarse particles in the cusp bays remain mobile, being acted upon by larger lift and drag forces, and rejected by (i.e., overpassing) the finer substrate. 

Though grain-size sorting between horns and bays has also been acknowledged in sand beach settings, the role of sorting in the process of sand cusp formation remains unclear. In one recent case, \citet{VanGaalen_etal2011} found no statistically significant correlation between grain size and position between horn and embayment on a sandy beach on the Atlantic coast of Florida. Their suggestion that horn-coarsening and embayment-fining likely occur only after cusps have developed is in contrast to the result presented herein, and suggests a distinction between sand and mixed sand-gravel cusp formation.

 
\section{Conclusions}

This study makes use of video-based observations of beach cusp formation episodes to characterise timescales of mixed sand-gravel (MSG) cusp evolution subjected to fetch-limited wind waves and 10 m range tides. The observations document the evolution of cusp fields with higher temporal resolution than has been previously reported in the literature.

Cusp evolution at Advocate Beach occurred over timescales that are an order of magnitude shorter than those typically reported in the field literature. The transition from relict to newly formed cusp fields occurred over \textit{ca}. 1 hour, with initial emergence and relict cusp decay timescales of 10-20 minutes. The general absence of cusps at mid-tide level on the beach face suggests a constraining formation timescale of $O$(10) minutes, or $O$(100) swash cycles, assuming nominal swash zone widths of 3-5 m at times of maximum rate of shoreline change. We provide evidence that timescale was influenced by the availability of larger grain sizes during formation, indicating the importance of the wide distribution of grain sizes to morpho-hydrodynamic feedbacks on MSG beaches. The cusp horn dynamics appeared to depend on the HWL and its location relative to any pre-existing cusp morphology. Horn accretion and embayment erosion were both shown to be important for at least one of the events, with local sediment volumes conserved. 

Sediment grain size sorting within the beach cusps was pronounced, with gravel-sized material concentrated in the horns, and sand-sized sediments in the bays. The apparent sensitivity of cusp formation timescales to the local grain size distribution, particularly the presence of surficial gravel and cobble, suggests that size segregation is intrinsic to the process of mixed sand-gravel cusp evolution. MSG cusps appear to bear similarities to other patterned bed states \citep[e.g., sorted bed forms,][]{Murray_Thieler2004} whose formation depends upon grain-size distribution dependent transport properties. However, more directed study is needed in order to establish the role of sorting in MSG cusp formation (i.e., cause or effect).









