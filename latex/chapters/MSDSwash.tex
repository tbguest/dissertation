
\chapter{Swash zone morpho-sedimentary dynamics}\label{Chapter:MSDSwash}

%\begin{abstract}
%	Field observations are presented of swash zone morpho-sedimentary evolution at a megatidal mixed sand-gravel beach. Swash zone sediment transport on mixed sand and gravel beaches is complicated by interactions between flow, morphology, and textural properties determined by the breadth of the grain size distribution. However, few observations of the coevolution of morphology and bed texture on mixed sand and gravel beaches exist in the literature. This is due in large part to the difficulty in obtaining sedimentological data with temporal resolution commensurate with timescales of morphological change. Often-energetic shorebreaks capable of entraining gravel- and cobble-sized grains limit the utility of in situ instrumentation, also contributing to the scarcity of observational data. Non-intrusive remote sensing methods provide logistically appealing alternatives. In particular, digital grain sizing techniques allow for the acquisition of grain size distributions with previously unattainable temporal resolution. The principal objective of the present study was to investigate linkages between grain size sorting processes and swash zone morphologic change using an array of collocated camera and aero-acoustic sensors for coincident bed level and sedimentological observations, and Lagrangian cobble tracking. Results from selected tides, all characterized by low energy wave conditions -- these being the conditions leading to cusp formation on Advocate Beach \ref{Chapter:CuspDynamics} -- and the formation of a high tide berm, are examined. In general, data from the range sensor and camera array show that bed accretion during berm formation was accompanied by coarsening and poorer sorting of beach surface sediments. Video imagery of tracer cobble movement was collected during high tide and early ebb. During at least one tide, the cobble trajectories indicated net shoreward  displacement during and immediately following high tide, while the substrate consisted, wholly or in part, of the coarse berm material. During ebb, seaward fining of the substrate coincided with net cobble displacement that diverged to seaward and landward from the mid-swash zone, having a mean of near zero. Cumulative cobble displacement increased with offshore distance, the greatest mobility being seen where the substrate was fine-grained. Cobble transport occurred predominantly in the cross-shore direction.
%\end{abstract}


\section{Introduction}

%[importance of texture para - ie motivation?, MSG]

Inter-relationships between bed level change and sediment properties in the swash zone are difficult to establish due to the challenges inherent in obtaining observations of bed level and sediments at the time scale of the swash forcing. The application of novel \textit{in situ} sensing techniques -- based on multi-element conductivity insertion probes -- has led to major advances in our understanding of swash zone sediment transport on sandy beaches \citep{Puleo_etal2013, Lanckriet_etal2013}. However, such methods are not suitable for coarse-grained beaches, where energetic shorebreaks capable of mobilising gravel- and cobble-sized grains pose a major hazard to instrumentation. Remote sensing methods provide a potential alternative. The recent application of remote sensing methods to the swash zone, including acoustic range sensors for observing bed level change \citep[e.g.,][]{Turner_etal2008} and image-based methods capable of estimating sediment grain size \citep[e.g.,][]{Rubin2004, Buscombe2013}, has also led to some important advances in our understanding of sediment transport processes in swash flows. However, previous efforts have been largely limited to pure sand or gravel beach types, with mixed sand-gravel beaches receiving little or no attention, particularly with regard to sediment properties.

%[search: sediment transport beach swash conductivity author:jack author:puleo]

%[sentence description of bed level sensing - swash + bed?]

The use of ultrasonic range sensors to obtain remote observations of bed level in the swash zone has provided insight into the dynamics of beach profile change. An important result borne of swash timescale bed level monitoring throughout the last 15 years is that of the inter-swash timescale of net bed level change, wherein a tendency toward a dynamic equilibrium profile \citep[see][]{Dean1991} is achieved via a balance between onshore and offshore sediment fluxes over many swash events \citep[e.g.,][]{Horn_Walton2004, Turner_etal2008, Masselink_etal2009, Russell_etal2009, Blenkinsopp_etal2011}. This is contrary to the previously prevailing notion that beach face equilibrium is the result of a balance between individual uprush and downrush events (i.e., intra-swash timescales). The use of arrays of bed level sensors has allowed for more comprehensive investigations of volume change across the beach profile, including the swash zone, than was previously possible \citep{Masselink_etal2009, Blenkinsopp_etal2011}. Studies of bed level change at swash or near-swash timescales on MSG beaches are few: \citet{Kulkarni_etal2004} used a manual post-and-ruler surveying method to examine bed level change on a MSG beach at time intervals of minutes; \citet{Horn_Walton2004} used a similar post-and-ruler method to measure bed level every 10-25 seconds over a three hour period encompassing high tide. They reported increases in bed elevation on the rising tide and decreases on the falling tide, with high frequency (i.e., one time step) bed level variations being nearly as large as the overall amount of bed level change. No MSG beach studies exist, to the author's knowledge, that make use of ultrasonic bed level sensors -- providing sub-swash timescale observations of bed level and swash height. 

Digital grain sizing methods enable the collection of grain size data non-intrusively with high temporal resolution. There are many automated methods for estimating a grain size distribution from imagery. Under the scheme proposed by \citet{Buscombe_etal2010}, digital grain sizing methods can be broadly classified as `geometrical' or `statistical'. Geometrical methods \citep[e.g.,][]{Chang_Chung2012} employ image processing techniques (e.g., segmentation, thresholding) to identify major and minor axis lengths of individual grains. Statistical methods \citep[e.g.,][]{Rubin2004, Warrick_etal2009, Buscombe2013} make use of time series analysis techniques (e.g. autocorrelation and Fourier or wavelet transforms) in the space domain to characterise texture in the image without attention to individual grains. Recent statistical methods \citep{Buscombe_etal2010, Buscombe2013}, which characterise the grain size distribution in terms of Fourier or wavelet derived power spectra, do not require calibration specific to a sediment population. Wavelet-based methods have the benefit of not requiring the image to be stationary or spatially homogeneous, making the method less sensitive to the number of grains in the image, their orientation, or the presence of voids (i.e., pore spaces between grains). These benefits make the wavelet method more suitable for applications involving images of poorly sorted sediments \citep{Buscombe2013}.

Though digital imaging techniques have been applied widely across the geological sciences, few studies have applied such methods to the swash zone. The only study to do so that the author is aware of is that of \citet{Austin_Buscombe2008}, who used the \citet{Rubin2004} method on images captured in the subaerial swash zone of a meso-macrotidal pure gravel beach. \citeauthor{Austin_Buscombe2008} used mean grain size data, collected at a minimum of 5 minute sampling intervals from both digital imagery (subaerial) and physical, grab-sampled sediments (subaqueous) to compare with bed level change from a manual, \textit{in situ} surveying method. The extent to which the image-derived grain size data were used in their analysis is unclear. Theirs is also the only study the author is aware of in the gravel beach literature which includes simultaneous observations of bed level and sediment properties in the swash zone. Their cross-shore bed level and sediment sampling transects were separated by 2 m alongshore. \citeauthor{Austin_Buscombe2008} observed that temporal changes in the grain size signal at several locations on the beachface were related to the morphological response: sediment coarsening being associated with accretion at the berm and step crest, and fining associated with accretion seaward of the step. They note, however, that distinct phases of the observed morphological change were not generally reflected in the grain size signal.

% p: Lagrangian cobble tracking

Lagrangian tracers have a long history of use for characterising sediment dynamics on beaches. In the gravel beach literature, particle tagging with fluorescent paint \citep{Nordstrom_Jackson1993, Ciavola_Castiglione2009, Stark_Hay2016}, or radio frequency identification tags \citep[RFID;][]{Osborne2005, Allan_etal2006, Curtiss_etal2009, Dickson_etal2011, Miller_etal2011, Miller_Warrick2012, Bertoni_etal2012, Grottoli_etal2015, Grottoli_etal2019} has lead to improved characterisations of transport dynamics on tide-to-tide timescales. No studies, to the author's knowledge, have made use of cobble-sized tracers to investigate transport at the shorter timescale of swash flows on a coarse-grained or MSG beach. 

%[above - mention msg work? noed and jack have results thst are important here.]

Here, results are presented from a field study at Advocate Beach, Nova Scotia. The study made use of collocated observations of bed level elevation using ultrasonic range sensors, and mean surficial grain size using digital imagery in the swash zone, at subsecond to several seconds resolution. The digital imagery and observations of bed level were both obtained using low-cost, commercially available sensing equipment. The observations are presented in the context of berm formation and evolution over two high-tide cycles. The objectives of this chapter are: (1) to investigate the coevolution of bed level and mean surficial grain size, seeking insight into the phenomenological role of grain size on swash zone morphological evolution, (2) to investigate the dynamics of individual particles in swash flows using video-based Lagrangian particle tracking, and (3) to assess the utility of low cost range sensing and video-based methods for quantifying bed level and mean grain size change in the sub-aerial swash zone. %The chapter is organized as follows... .



\section{Methods}\label{Adv2018_methods}

\subsection{Experiment Overview}\label{subsec:Experiment}

%[include a trimmed-down wave stats plot?]

A field experiment was conducted between 14 and 27 October, 2018 at Advocate Beach, Nova Scotia. The experiment spanned 27 tides, which are hereafter referred to by their low tide index within the experiment, i.e., 1 though 27. The focal point of the experiment was to investigate the coevolution of bed level and mean surficial grain size, with an emphasis on processes in the swash zone. Data were collected using a four-element array of collocated ultrasonic range sensors and cameras, along with an overhead camera used for tracking the movements of tracer cobbles in the swash. Both systems were movable, and mounted over the swash zone during periods of low to moderate energy forcing conditions, in anticipation of cusp formation (Chapter \ref{Chapter:CuspDynamics}). Beach-scale surveying of bed level and grain size was carried out on a tide-to-tide basis. Results from the survey component are presented in Chapter \ref{Chapter:MSDBeach}.

Coincident evolution of beach morphology and sediment properties in the swash zone was investigated using an array of collocated Maxbotix MB7384 ultrasonic range sensors (range resolution of \textit{ca}. 1 mm) and 5 megapixel Raspberry Pi cameras (Fig. \ref{fig:frames}). The array consisted of four downward-facing range sensor/camera pairs, cantilevered approximately 2 m horizontally over the swash on an instrument frame that could be moved as shoreline position changed with the tide. The pairs were separated by 0.9 m alongshore (i.e., having a total longshore span of 2.7 m), and had a nominal elevation of 0.75 m above the bed. The range data were sampled at 6 Hz, and the images at 0.2 Hz. Each of the four array element pairs were controlled by a Raspberry Pi single board computer, which also served as the data logger. A wireless router, connected to the Raspberry Pis via ethernet, enabled Wi-Fi communication with the Pis to initiate and terminate data logging. The four computers were time-synchronised using network time protocol (NTP), and powered from a 12 V marine battery. The instrument frame was assembled near the high water line and data were collected at 3 to 5 ``stations" during late flood tide, high tide, and early ebb. At each flood tide station, sampling was initiated prior to the maximum swash runup position passing beneath the array, and continued until the bed was fully obscured by water (i.e., the swash zone was no longer in the instruments' field of view). At ebb stations, the frame was positioned so that the transition from full water cover to fully-exposed bed could be captured. The longest sampling periods corresponded to high tide, when the array stations could be held through the shoreline's advance and retreat, from late flood into early ebb. The position of each array element was recorded using RTK (real time kinematic) GPS at each station. 

\begin{figure}[tbp] %/home/tristan/Documents/Projects/AdvocateBeach2018/src/visualization/plot_beach_profile_data.py
%  	\includegraphics[width=\columnwidth]{/home/tristan2/Documents/PhD/Dissertation/figures/chapter3/frames.png}
	\includegraphics[width=\columnwidth]{figures/chapter3/frames.png}
 	\caption[Instrument frame photographs: Advocate 2018 experiment]{Instrument frames used during the 2018 experiment at Advocate Beach. (A) The array frame, with its four ultrasonic range sensor and camera pairs used to observe bed level change and grain size, respectively, in the swash zone. Each pair was separated by 0.9 m in the alongshore from its nearest neighbour, at a nominal elevation of 0.75 m above the bed. (B) The overhead camera frame, from which a downward-looking camera was suspended in order to monitor the transport of painted tracer cobbles in the swash. The camera was elevated approximately 3 m above the bed. \label{fig:frames}}
\end{figure}

An overhead Raspberry Pi camera -- also network-connected and serviceable via the wireless router -- was used for monitoring tracer cobble transport in the swash zone. The camera was mounted to a second instrument frame, consisting of a semi-stationary base which could be moved with the changing shoreline position, and a moveable arm which supported the camera, allowing it to view the swash zone from a height of \textit{ca}. 3 m without the frame base being in the image. The image field of view at the beach surface was approximately 2.4 by 4.3 m (longshore by cross-shore). During low to moderate wave energy conditions, the overhead camera frame occupied 3 to 5 stations near the high water line (HWL). A minimum of three ground control points were captured at each station using RTK GPS to provide a scaling between pixel and ground coordinates. Prior to deployment, cobbles were sieved into three different size classes: 22.4 to 31.5 mm, 31.5 to 45 mm, and 45 to 63 mm. The cobbles in these size classes were painted blue, orange, and yellow, respectively. The camera frame stations were typically chosen to capture cobble transport during high tide and early ebb, so the cobbles could be retrieved and redeployed after the shoreline had retreated. The frame held a station until the camera's field of view no longer contained the mean shoreline position. Video was recorded continuously throughout.  
  
The alongshore orientation of the four range sensor-camera elements on the array frame was chosen to capture the development of beach cusps, or other three-dimensional morphology in the alongshore. The geometries of both instrument frames (i.e., their fields of view in relation to their bases, which were in contact with the beach surface) were chosen so that the mid-swash zone could be sampled without the bases of the instrument frames interfering either with the data collection, or with the swash processes being observed. This limited the use of the frames to `fairweather' conditions, during which the maximum swash runup distance was less than approximately 4 m, favourable for cusp formation (Chapter \ref{Chapter:CuspDynamics}). The majority of tides were characterised by high steepness incident waves leading to an energetic shorebreak. There were three periods of fairweather forcing characterised by low amplitude, low steepness waves during which the swash array frame could be deployed: tides 15-16, tides 19-22, and tide 27.

Local grid coordinates were defined in keeping with previous experiments at the site (Chapters \ref{Chapter:PorePressure} and \ref{Chapter:CuspDynamics}), with the origin defined as the former position of a vertical pressure sensor array (see Chapter \ref{Chapter:PorePressure}), approximately 20 m seaward of the nominal high tide shoreline. $x$ is alongshore, positive to the northwest, and $y$ is positive offshore.

\subsection{Bed Level and Swash Analysis}

The data from the ultrasonic range sensors represent first returns (i.e., the distance to the nearest object within the $O$(10 cm) radius beam pattern), which were either from the exposed beach surface in the absence of swash, or from the elevation of the water surface when swash was present. Processing of the range time series was carried out to isolate the bed level and swash signals. Spurious returns of 0.5 and 5 m -- the sensors' minimum and maximum sensing distances -- made an additional processing step necessary. The spurious returns were attributed to diminished or scattered acoustic reflections from aerated swash at leading edge of the swash front. No return, or returns not within the 0.5-5 m sensing range of the instruments, result in range output of 0.5 or 5 m. 

The bed level was extracted from the range data time series by identifying sequences of $N$ samples in which no sample differed from the first sample in the sequence by more than some predefined range threshold. The time threshold was set to 9 samples (1.5 seconds), and the range threshold set to 5 mm. For analysis applications requiring a uniformly sampled bed level time series, gaps in the series associated with swash were filled via linear interpolation. 

\begin{figure}[tbp] %/home/tristan/Documents/Projects/AdvocateBeach2018/src/visualization/plot_beach_profile_data.py
%  	\includegraphics[width=\columnwidth]{/home/tristan2/Documents/PhD/Dissertation/figures/chapter3/tide19/pi74_swash_peaks_bed_level.png}
  	\includegraphics[width=\columnwidth]{figures/chapter3/tide19/pi74_swash_peaks_bed_level.png}
 	\caption[Bed level and swash height time series]{Time series of elevation data, collected during high tide by the southmost array element. Orange dots represent the extracted bed level, and blue dots represent swash height maxima. Tide 19. \label{fig:swash_peaks}}
\end{figure}

The swash thickness was extracted from the range time series by isolating all values at ranges less than the range to the interpolated bed level, minus a buffering threshold of 2 mm to eliminate spurious low amplitude peaks due to instrument noise. The isolated segments were defined to be swash events if they had a minimum duration of 5 samples (0.83 s) and a local maximum at least 15 mm above the bed that did not exist at either endpoint of the segment. These criteria were implemented to exclude spurious events in the swash time series that did not exhibit the anticipated shape of a runup event (i.e., a sequence of increase, maximum, and decrease in swash height). A sample elevation time series, with the extracted bed level and swash height maxima, is shown in Fig. \ref{fig:swash_peaks}.

Bed level change between swash runup events was computed by differencing the final values in each bed level segment in the (non-interpolated) bed level signal, where a segment consisted of contiguous points meeting the exposed bed criteria described above. This definition of bed level change was chosen to be conceptually consistent with the definition proposed by \citet{Blenkinsopp_etal2011}. %The bed level difference was excluded from any analysis if the time between bed observations was greater than 5[?] s. Instances of bed level change were only associated with runup events if a swash height maximum could be identified no more than 5 s prior to the instance of bed level change.


\subsection{Digital Grain Sizing}

A wavelet-based digital grain sizing (DGS) package \citep[see][]{Buscombe2013}, implemented in Python, was used to estimate grain size statistics from the camera array imagery. The DGS algorithm does not require calibration, and takes as input a grain-resolving image containing only sediment. 

The image sets from the four array-frame cameras were manually curated to include only those images with fully exposed bed in the region of the image used for analysis. Each image was cropped to half width and height in the centre of the frame, corresponding to a field of view at the bed of 1.42 $\times$ 0.85 m, for a 0.75 m camera height above the bed. Input parameters for the algorithm include a pixel to physical unit scaling, a maximum feature diameter to be resolved, and a dimensional scaling factor. The bed-level signal extracted from the range sensor time series was used, along with the camera's known field of view specifications, to establish the pixel to physical unit scale factor for each image. The bed level signal was smoothed to eliminate short period changes in bed level attributable to individual grain movements. The maximum feature diameter, defined as the inverse ratio of the pixel width of the frame to the width of the largest feature to be resolved, was also dynamically assigned to maintain a maximum feature resolution of 56 mm -- a value chosen to balance output resolution at both small and large grain sizes. The dimensional scale factor was set as 0.8. See Appendix \ref{Appendix2} for a discussion of the choice of dimensional scale factor, as well as for more detailed descriptions of the remaining input parameters. 

Though the algorithm is capable of returning a full grain size distribution, validation of the output against distributions from both sieve and manual point count analyses \citep[see][or Appendix \ref{Appendix2} for a description of the point count method]{Barnard_etal2007, Buscombe_etal2010} indicated that only the lowest moment of the grain size distribution (mean grain size) was captured satisfactorily. We attribute the algorithm's poor representation of the higher order moments to the wide grain size distribution. See Appendix \ref{Appendix2} for further discussion of the validation procedures.


\subsection{Cobble Tracking}

Cobble trajectories were manually extracted from the video image sets captured by the overhead camera. To mitigate the difficulties posed by the partial or full occlusion of cobbles by bubbles and foam in the swash, pixels with high white content (red, green, and blue channel intensity values all exceeding 175 out of 255) were subtracted from each image, then each image was averaged with the 10 preceding images. The resulting composite images showed a partially reconstructed bed, with more tracer cobbles being visible than in the original images. Due to the regular occlusion of the cobbles by swash, the extracted trajectories are assumed to be accurate to within the timescale of the swash forcing (\textit{ca}. 6 s), i.e., any transport occurring while the cobbles were submerged was not captured. The 45-63 mm (yellow) size class was most easily identified in the images. Reduced visibility of the other size classes meant that cobble identities could not be maintained with confidence between instances of occlusion. Thus, only trajectory data from the 45-63 mm size class is presented in this chapter. Tracking of the cobbles began immediately after they were deposited in the swash. Tracking was stopped when cobbles were buried or transported out of the field of view. 


\section{Results}\label{section:results}

\subsection{Swash Zone Morpho-Sedimentary Dynamics}\label{subsection:swashMSD}

The frame bearing the range sensor and camera arrays was deployed during five high tides over the course of the experiment, each characterised by low to moderate energy forcing conditions. During tides 19 and 27, the array was favourably positioned relative to the HWL such that intermittent swash height and bed exposure could be observed for periods longer than one hour without moving the frame. For these cases, time series of the swash thickness, bed level, and image-derived mean grain size are available. During both tides 19 and 27, a pronounced berm developed near the HWL.

Applying the digital grain sizing algorithm to the images captured by the Raspberry Pi cameras enabled the examination of the coevolution of bed level and mean grain size at the bed beneath each array element. Figs. \ref{fig:MSD_timeseries_19} and \ref{fig:MSD_timeseries_27} show time series of the bed level and mean grain size for the periods of high tide data collection during tides 19 and 27. The grain size data are inherently noisy, so the individual data points are less valuable than the trends revealed by large numbers of data points. In both cases, the morphological context was the formation -- and in the latter case the formation and shoreward translation -- of a high tide berm. Visual inspection of the time series indicates that the bed level and mean grain size measures tend to co-vary. 

\begin{figure}[tbp] %/home/tristan/Documents/Projects/AdvocateBeach2018/src/visualization/plot_beach_profile_data.py
%  	\includegraphics[width=\columnwidth]{/home/tristan2/Documents/PhD/Dissertation/figures/chapter3/tide19/MSD_timeseries.png}
  	\includegraphics[width=\columnwidth]{figures/chapter3/tide19/MSD_timeseries_minus.png}
  	\caption[Time series of swash zone bed level and mean grain size, tide 19]{Morpho-sedimentary time series during high tide for tide 19, recorded by the four range sensor-camera array elements. (a) Swash height. (b) Bed level. (c) Mean grain size. The solid line is a locally weighted (loess) regression using a \textit{ca}. 10 min averaging window.}
 	\label{fig:MSD_timeseries_19}
\end{figure}

During tide 19, coarse material accumulated near the HWL, initially in a mound directly beneath the north-most array element ($x=-15.5$ m in Fig. \ref{fig:MSD_timeseries_19}). The mound initially resembled a large cusp horn, though its alongshore extent widened over the following tens of minutes, becoming steeper in the offshore direction and more berm-like. An incipient topographic low existed beneath the southmost array element ($x=-18.2$ m). By early ebb tide, a more uniform berm had formed, with irregular undulations in amplitude alongshore. The berm beneath the array sloped downward to the south (note the elevation differences between the array element at $x=-15.5$ and $x=-18.2$ m), with a maximum amplitude of nearly 20 cm, immediately beneath the northmost array element. The grain size time series are noisy, and sparsely sampled, especially at high tide (i.e., between 11:30 and 12:00) due to the more frequent swash events. However, common trends can be noted: namely, the upward trend in mean grain size as the swash zone first reached the sampling region (11:10-11:20), which precedes the onset of the upward trend in the bed level time series. The apparent fining of the surficial sediments at $x=-18.2$ m following the initial coarsening trend is also evident, and corresponds to a similar, but lagged, trend in the bed level time series.

\begin{figure}[tbp] %/home/tristan/Documents/Projects/AdvocateBeach2018/src/visualization/plot_beach_profile_data.py
%  	\includegraphics[width=\columnwidth]{/home/tristan2/Documents/PhD/Dissertation/figures/chapter3/tide27/MSD_timeseries.png}
  	\includegraphics[width=\columnwidth]{figures/chapter3/tide27/MSD_timeseries_minus.png}
 	\caption[Time series of swash zone bed level and mean grain size, tide 27]{Morpho-sedimentary time series during high tide for tide 27, recorded by the four range sensor-camera array elements. (a) Swash height. (b) Bed level. (c) Mean grain size. The solid line is a locally weighted (loess) regression using a \textit{ca}. 10 min averaging window.}
 	\label{fig:MSD_timeseries_27}
\end{figure}

During tide 27, a berm began to form seaward of the HWL (note the increase in $z$ in Fig. \ref{fig:MSD_timeseries_27}, 13:50-14:10). As the swash zone migrated landward, coarse material was pushed over the berm crest, leading to a shoreward migration of the berm by roughly 2 m. The decrease in bed elevation between 14:10 and 14:30 is a result of this berm translation. During early ebb tide, coarse material began to accrete on the seaward face of the berm, leading to the bed elevation increase observed between 14:30 and 14:50. The mean grain size was initially coarse (\textit{ca}. 30 mm) in all three of the sampled locations, but shows a downward trend in two of the three cases. Fining occurred in all cases after the initiation of berm growth seen in the bed level time series. In all cases, the minimum mean grain size occurred at or just after high tide, when the sampling location was near the mid-swash zone, and also nearest the base of steep berm face. Bed surface coarsening coincided with the increase in bed level on the seaward face of the berm during early ebb tide.

A phenomenon that was often observed, both visually and in the grain size time series, was the shoreward migration of a band of coarse material at the top of the swash zone during late flood tide (see Figs. \ref{fig:MSD_timeseries_19}, \ref{fig:MSD_timeseries_27}: `bumps' in the mean grain size time series). Similarly, a coarsening of the substrate was generally observed at the seaward edge of the swash zone during early ebb tides. Surficial fining was generally observed nearer the mid-swash zone. This fining is apparent in both Figs. \ref{fig:MSD_timeseries_19} and \ref{fig:MSD_timeseries_27}. 

%[Figure: NEW? images of (red stone) coarse lag being pushed shoreward by the top of the swash. Maybe this should be a discussion addition?]

%[MOVED FROM PRVIOUS SECTION: "Swash zone bed level change"]

\begin{figure}[tbp] %/home/tristan/Documents/Projects/AdvocateBeach2018/src/visualization/plot_beach_profile_data.py
%  	\includegraphics[width=\columnwidth]{/home/tristan2/Documents/PhD/Dissertation/figures/chapter3/modified/bed_level_change_histograms.png}
%	\includegraphics[width=\columnwidth]{figures/chapter3/modified/bed_level_change_histograms.png}
	\includegraphics[width=0.75\columnwidth]{figures/chapter3/tide19/joint_pdf.png}
% 	\caption{Observations of swash height and bed level change from the high tide data collection run of tide 19 (see Fig. \ref{fig:MSD_timeseries_19}). Data from all four sensor pairs are included. (a) Swash height plotted against absolute bed level change. (b) Distribution of bed level change events. The black dashed curve is a best-fit Gaussian distribution.}
	\caption[Joint distribution of bed level change and swash height, tide 19]{Joint distribution of bed level change between swash events and swash height from the high tide data collection run of tide 19 (see Fig. \ref{fig:MSD_timeseries_19}). Data from all four sensors are included.}
 	\label{fig:hist19}
\end{figure}

\begin{figure}[tbp] %/home/tristan/Documents/Projects/AdvocateBeach2018/src/visualization/plot_beach_profile_data.py
	%  	\includegraphics[width=\columnwidth]{/home/tristan2/Documents/PhD/Dissertation/figures/chapter3/modified/bed_level_change_histograms.png}
	%	\includegraphics[width=\columnwidth]{figures/chapter3/modified/bed_level_change_histograms.png}
	\includegraphics[width=0.75\columnwidth]{figures/chapter3/tide27/joint_pdf.png}
	% 	\caption{Observations of swash height and bed level change from the high tide data collection run of tide 19 (see Fig. \ref{fig:MSD_timeseries_19}). Data from all four sensor pairs are included. (a) Swash height plotted against absolute bed level change. (b) Distribution of bed level change events. The black dashed curve is a best-fit Gaussian distribution.}
%	\caption{Histograms of bed level change between swash events from the high tide data collection run of tides 19 (a) and 27 (b) (see Figs. \ref{fig:MSD_timeseries_19} and \ref{fig:MSD_timeseries_27}). Data from all four sensor pairs are included. The black dashed curves are best-fit Gaussian distributions.}
	\caption[Joint distribution of bed level change and swash height, tide 27]{Joint distributions of bed level change between swash events and swash height from the high tide data collection run of tide 27 (see Fig. \ref{fig:MSD_timeseries_27}). Data from three sensors are included.}
	\label{fig:hist27}
\end{figure}

The distribution of bed level changes (Figs. \ref{fig:hist19} and \ref{fig:hist27}) shows that the majority of changes between swash events were near zero, with larger changes -- both positive and negative -- that are loosely approximated by a Gaussian distribution, though with higher kurtosis values (kurtosis of 5.1 and 5.4 for tides 19 and 27, respectively, relative to a value of 3 for a Gaussian distribution). This finding is consistent with similar analyses in the literature \citep[e.g.,][]{Horn_Walton2004, Turner_etal2008, Masselink_etal2009, Blenkinsopp_etal2011}, which have demonstrated that bed level change over the course of a tidal cycle is the result of the cumulative effect of many instances of small accretion and erosion. The joint probability distributions of the swash height and changes in bed level between swash events do not indicate a clear relationship between the two. %The distribution of absolute bed level changes relative to the swash height (shown in the scatter plot in Fig. \ref{fig:hist}) suggests that swash height is not a clear predictor of bed level change.

%\begin{figure}[tbp] %/home/tristan/Documents/Projects/AdvocateBeach2018/src/visualization/plot_beach_profile_data.py
%  	\includegraphics[width=\columnwidth]{/home/tristan2/Documents/PhD/Dissertation/figures/chapter3/tide27/pi71_swash_peaks_bed_level_with_dz.png}
% 	\caption{(a) Time series of swash height and bed level during tide 27 at the $x=19.5$ m alongshore array element, and (b) the corresponding change in bed level, d$z$.}
% 	\label{fig:bed_level_change_pi71}
%\end{figure}
%
%\begin{figure}[tbp] %/home/tristan/Documents/Projects/AdvocateBeach2018/src/visualization/plot_beach_profile_data.py
%  	\includegraphics[width=\columnwidth]{/home/tristan2/Documents/PhD/Dissertation/figures/chapter3/tide27/pi73_swash_peaks_bed_level_with_dz.png}
% 	\caption{(a) Time series of swash height and bed level during tide 27 at the $x=21.3$ m alongshore array element, and (b) the corresponding change in bed level, d$z$.}
% 	\label{fig:bed_level_change_pi73}
%\end{figure}

\begin{figure}[tbp] %/home/tristan/Documents/Projects/AdvocateBeach2018/src/visualization/plot_beach_profile_data.py
%  	\includegraphics[width=\columnwidth]{/home/tristan2/Documents/PhD/Dissertation/figures/chapter3/tide27/pi74_swash_peaks_bed_level_with_dz.png}
  	\includegraphics[width=\columnwidth]{figures/chapter3/tide27/pi74_swash_peaks_bed_level_with_dz.png}
 	\caption[Time series of swash zone bed level and change in bed level]{(a) Time series of swash height and bed level during tide 27 at the $x=-22.2$ m alongshore array element, and (b) the corresponding change in bed level, d$z$.}
 	\label{fig:bed_level_change_pi74}
\end{figure}

Fig. \ref{fig:bed_level_change_pi74} shows time series of swash height, bed elevation, and the change in bed elevation during the high tide data collection of tide 27. In general, the envelope of bed elevation change is largest when bed elevation was higher (i.e., near the beginning and end of the time series). This may be due to the larger mean grain sizes associated with coarse lag at the leading edge of the swash during flood tide, and the accretion of coarse material on the seaward face of the berm during flood.



\subsection{Cobble Dynamics}

Cobble trajectory statistics were computed from data associated with tide 19. Video datasets of cobble transport in the swash zone were collected at four locations in the cross-shore (i.e., four distinct mean shoreline positions) during high tide and early ebb: 

\begin{itemize}
	\item Station 1: at high tide, when the shoreline position was nearest to the HWL. Here, the tracer cobbles were deployed atop the coarse berm material. The camera's cross-shore field of view spanned $y\approx$ -14.5 to -11 m in local grid coordinates, and contained almost entirely coarse berm material. 
	\item Station 2: 45 min after high tide. The shoreline and swash zone coincided with the region immediately seaward of the coarse berm. Coarse berm material was present in the landward one third of the camera's field of view, which spanned from $y\approx$ -9.5 to -6 m (i.e., 5 m seaward of station 1). The cobbles were deployed in the mid-swash, over a combination of the coarse seaward face of the berm and the finer material farther seaward. 
	\item Station 3: 75 min after high tide. The swash zone no longer coincided with any coarse-grained berm material, and the substrate was predominantly fine-grained and uniform. The cobbles were deployed near mid-swash and mid-camera frame. The camera field of view spanned $y\approx$ -5.5 to -2 m across-shore (9 m seaward of station 1). 
	\item Station 4: 85 min after high tide, with the bed conditions and the cobble deployment being similar to those described for location 3. The cross-shore field of view of the camera in this location was $y\approx$ -2.5 to 1 m (12 m seaward of station 1).
\end{itemize}	

\begin{figure}[tbp] %/home/tristan/Documents/Projects/AdvocateBeach2018/src/visualization/plot_beach_profile_data.py
%  	\includegraphics[width=\columnwidth]{/home/tristan2/Documents/PhD/Dissertation/figures/chapter3/tide19/cobble_transport/transport_stats.png}
  	\includegraphics[width=\columnwidth]{figures/chapter3/tide19/cobble_transport/transport_stats.png} 	
 	\caption[Net and cumulative cobble transport]{Mean and standard deviation of the (a) net and (b) cumulative cobble transport. Positive is shoreward.}
 	\label{fig:cobble_transport_stats}
\end{figure}

The net and cumulative transport statistics for all four stations are summarised in Fig. \ref{fig:cobble_transport_stats}. The net cobble transport was shoreward at stations 1 and 2, where the substrate consisted wholly or in part of coarse-grained berm material. At stations 3 and 4, where the bed surface was predominantly fine-grained, the net transport was near zero, but with a high degree of variation between individual cobbles. The cumulative transport of cobbles increased to seaward (i.e., cobbles were more mobile where the mean shoreline position was farther to seaward), corresponding to a decrease in coarse-grained material in the swash zone substrate. The longshore component of transport was small in comparison to the cross-shore transport at all the stations, consistent with previous observations at Advocate Beach \citep[e.g.,][]{Stark_Hay2016}. 

\begin{figure}[tbp] %/home/tristan/Documents/Projects/AdvocateBeach2018/src/visualization/plot_beach_profile_data.py
	%  	\includegraphics[width=\columnwidth]{/home/tristan2/Documents/PhD/Dissertation/figures/chapter3/tide19/cobble_transport/cross_shore_trajectories_combined_position3.png}
	\includegraphics[width=\columnwidth]{figures/chapter3/tide19/cobble_transport/cross_shore_trajectories_combined_position1.png}
	\caption[Cross-shore cobble transport trajectories and distances: Station 1]{Cross-shore cobble transport trajectories and transport distances at station 1 during tide 19. The bed composition was coarse-grained, leading to net onshore transport and low transport rates. (a) Time series of the cross-shore component of all the cobble trajectories. The grey line indicates the time-varying position of the swash front. Landward  is up. (b) Mean and standard deviation of the cross-shore transport distance in each each of eight bins corresponding to the the cross-shore coordinates in (a). The transport distance was assigned a bin based on its starting point. (c) Number of transport events used in the calculations of the means and standard deviations in (b).}
	\label{fig:cobble_transport_station1}
\end{figure}

\begin{figure}[tbp] %/home/tristan/Documents/Projects/AdvocateBeach2018/src/visualization/plot_beach_profile_data.py
%  	\includegraphics[width=\columnwidth]{/home/tristan2/Documents/PhD/Dissertation/figures/chapter3/tide19/cobble_transport/cross_shore_trajectories_combined_position3.png}
  	\includegraphics[width=\columnwidth]{figures/chapter3/tide19/cobble_transport/cross_shore_trajectories_combined_position3.png}
 	\caption[Cross-shore cobble transport trajectories and distances: Station 3]{Cross-shore cobble transport trajectories and transport distances at station 3 during tide 19. The bed composition was predominantly fine-grained, leading to higher transport rates than at other stations. (a) Time series of the cross-shore component of all the cobble trajectories. The grey line indicates the time-varying position of the swash front. Landward  is up. (b) Mean and standard deviation of the cross-shore transport distance in each each of eight bins corresponding to the the cross-shore coordinates in (a). The transport distance was assigned a bin based on its starting point. (c) Number of transport events used in the calculations of the means and standard deviations in (b).}
 	\label{fig:cobble_transport_station3}
\end{figure}

Closer inspection of the transport characteristics within each station reinforces the finding of low cumulative transport and net onshore transport of cobbles in the presence of the coarse-grained berm material. For example, data from station 1 are shown in Fig. \ref{fig:cobble_transport_station1}. At the more seaward stations, a general trend of divergence of the cobbles away from the mid-swash zone is observed: shoreward transport above the mid-swash level, and seaward transport below. This was true in particular for stations 2 and 3, for which the swash zone was well-centred in the camera's field of view, and the surficial sediments were predominantly fine-grained. Data from station 3 illustrating this divergence are shown in Fig. \ref{fig:cobble_transport_station3}. %A less clear relationship between transport and cross-shore position was observed at station four


\section{Discussion}\label{section:discussion}

The morpho-sedimentary evolution of a mixed sand-gravel beach was investigated at the swash scale ($O(10^{-2}-10^0)$ m) through point observations of bed level and grain size with temporal resolution on the order of seconds. Data were collected during selected tides characterised by low steepness wave incidence and berm building.  % (Fig. \ref{fig:wavestats}). 
% The forcing conditions were dominated by steep, wind-band incident waves leading to an energetic shore break, interspersed by periods of low steepness wave incidence and berm building.

The bed level and mean grain size signals displayed qualitative similarities. Point observations during berm formation and evolution were captured by the ultrasonic range sensor and camera pairs. In general, accretion at the berm corresponded to surficial sediment coarsening, though finer structure in the grain size signal was also apparent, particularly as the leading edge of the swash zone passed beneath the instrument array on flood tide (Figs. \ref{fig:MSD_timeseries_19} and \ref{fig:MSD_timeseries_27}), and during the observed instance of shoreward berm translation during tide 27 (Fig. \ref{fig:MSD_timeseries_27}). 

The shoreward migration of a coarse band of sediments with the leading edge of the swash appeared to precede any substantial changes in bed elevation during tides 19 and 27. This precursor to berm formation has been reported in the literature \citep[e.g.,][]{Austin_Buscombe2008, Duncan1964}, and has been attributed to the temporary stranding of coarse material at the landward edge of the swash. At high tide, the slowdown and arrest of the swash zone's shoreward translation leads to continued accretion at the leading edge. The berm crest migrates shoreward during periods of berm overtopping, wherein the coarsest mobile fraction is saltated, or `thrown', over the crest. This is followed by accretion on the seaward face of the berm during the seaward regression of the shoreline during early ebb tide, leading to an increase in berm width \citep[e.g.,][]{Pontee_etal2004, Austin_Masselink2006}. The array data are consistent with this conceptual model. They indicate that the coarsest sediments correspond to the berm crest, nascent or developed, with fining occurring on the seaward face of the berm. Note also that the peak in mean grain size associated with the berm in the Fig. \ref{fig:alltides_profiles}b profiles generally appears one station shoreward of the $\Delta z$ peak in Chapter \ref{Chapter:MSDBeach}, Fig. \ref{fig:alltides_profiles}a, and appears to be more closely aligned with the HWL.

The above results and interpretation of the swash zone bed level and mean grain size signals are supported by the cobble tracer results. The divergence of the cobbles from the mid-swash toward the seaward and shoreward edges of the swash zone is consistent with the formation of a coarse deposit that migrates with the leading edge of the swash zone, as well as with the observed fining of the surficial material in the mid-swash zone in many cases. Though not verifiable with the observations here, it is likely that seaward transported material accumulated in the beach step region associated with bore collapse at the shore break, which also likely migrated with the cross-shore translating swash zone. The beach step has been demonstrated elsewhere to play an important role in controlling wave breaking on steep beaches, and has also been shown to migrate with the translating swash zone \citep{Austin_Buscombe2008}. The net shoreward transport of cobbles in the vicinity of the berm (Fig. \ref{fig:cobble_transport_stats}, stations 1 and 2) suggests that the beach profile at high tide was in disequilibrium with the forcing in this case.

The cobble transport results highlight the influence of the substrate on the transport dynamics of coarse particles; transport is favoured on a fine substrate, where low angles of pivot and higher exposure to lift and drag forces cause the coarse particles to overpass the finer ones. Transport is inhibited where the substrate is coarse, due to higher angles of pivot required for mobilisation and decreased return flow velocities resulting from increased infiltration and hydraulic roughness.

The low-cost ultrasonic range sensors, though yielding a lower resolution data product than similar sensors used in other studies, were capable of characterising bed level changes with probability distributions that are comparable to those reported elsewhere \citep[e.g.,][]{Turner_etal2008, Blenkinsopp_etal2011}, including at least one gravel beach \citep{Russell_etal2009}: namely, having a quasi-Gaussian distribution that is indicative of an inter-swash timescale for profile evolution, where net changes to the profile are a result of time integration of small changes in bed elevation (both positive and negative) over many swash cycles. The ultrasonic range sensors used in this study would be less suitable where resolution finer than $\pm$ 1 mm is required (e.g., in a pure sand setting where average grain sizes are less than 1 mm).

The high kurtosis of the bed level change probability distributions (kurtosis of 5.1, 5.4 for tides 19 and 27, respectively) relative to a Gaussian distribution (kurtosis value of 3) may be attributable to the `armouring' effect of the coarse bed: low energy swash events, i.e., small runup events having low velocities, may not have lead to mobilisation or deposition in the range sensor sampling region, resulting in a bed level change probability distribution more heavily zero-weighted than a Gaussian distribution. This is in contrast to a sandy bed, where some degree of bed level change might be expected with each swash event due to the higher mobility of sand grains. Instances of greater bed level change associated with the transport of gravel- or cobble-sized grains into or out of the sampling region may also have contributed to a higher kurtosis value through increased weighting of the tails of the bed level change distribution.

The Raspberry Pi cameras were adequate for capturing images and video of the bed for digital grain sizing and cobble tracking. The difficulties resolving higher order moments of the grain size distribution (see Appendix \ref{Appendix2}) can more likely be attributed to the wide grain size distribution at Advocate Beach, which spans three orders of magnitude.

%[discuss pros and cons of low cost options. eg cobble tracking was labour intensive, and didn;t lend itself well to automation.]

\section{Conclusions}

The morpho-sedimentary evolution of a mixed sand-gravel beach was investigated at the swash zone scale, through point observations of bed level and mean grain size with temporal resolution on the order of seconds. Data were collected near high tide during periods of fairweather forcing characterised by low steepness wave incidence and berm building. The shoreline position changed rapidly due to the large (\textit{ca}. 10 m) tidal range.

Point observations of swash zone bed level and mean grain size at the timescale of the swash forcing displayed qualitatively similar trends. In general, increases in bed level (e.g., during berm building) corresponded to increases in mean grain size. Finer structure in the grain size signals was also observed. The largest mean grain sizes were generally associated with the leading edge of the mean swash front, and the smallest sizes with the mid-swash zone. This was interpreted as cross-shore divergence of gravel and cobble-sized material out of the mid-swash zone.

The tracer cobble trajectory results support the above interpretation. Cobbles initially positioned above the mid-swash zone were generally transported shoreward, and those positioned below the mid-swash were transported seaward. The shoreward transport in the upper part of the swash zone is consistent with the formation of a coarse-grained deposit, or incipient berm, at the leading edge of the swash. The downslope-transported cobbles were likely deposited near the seaward edge of the swash zone, contributing to the formation of a transient coarse-grained beach step. The step would then serve as a source of coarse material, which would be reintroduced to the swash zone during the ebb-tide seaward translation of the shoreline, contributing to the coarsening of the beach surficial sediments during fairweather forcing (Chapter \ref{Chapter:MSDBeach}). The substrate composition affected the magnitude of transport: tracer cobbles were less mobile, but experienced net shoreward transport in the vicinity of the coarse berm material. 

%However, given the well-established relationship between it may be expected that  , or whether the processes are better described as being superimposed upon the larger process of dynamic equilibrium profile evolution.

%Clear shortcomings of this study are the limitation of sampling to the beach surface sediments, and the lack of observations of higher order moments of the grain size distribution, namely: measures of grain size sorting, skewness, and kurtosis. Also not included in this study was the role of particle shape, which has also been shown to play an important role in transport dynamics.

The low-cost, commercially available range sensors used in this study were successful in resolving signals of morpho-sedimentary change in the subaerial swash zone. This finding suggests great potential for the use of close-range remote sensing techniques to examine the coevolution of bed level and grain size in response to swash processes, at least in macrotidal mixed sand-gravel settings. More precise instrumentation would reduce the need for added processing, and would be required where a greater proportion of the grain size distribution is less than \textit{ca}. 1 mm.

In considering future work, it would be of interest to make similar measurements with greater spatial coverage, particularly in the cross-shore, in order to better resolve the fine-scale grain size and bed level changes associated with incipient berm formation at the leading edge of the swash. More sensors in an across-shore configuration would also allow for the consideration of volume change in the swash region; e.g., is positive bed level change at the berm balanced by erosion from the mid-swash, or must material be sourced from the step region as well? The step has been shown to have an important influence on swash processes via its control on wave breaking, and has been suggested to be an important source of coarse material for swash zone morpho-sedimentary evolution in MSG settings. However, the methods employed for this study are not capable of directly observing processes at the step. To the knowledge of the author, no non-intrusive methods have been used to study the step in a field setting. 

Given the acknowledged influence of the full grain size distribution on mixed sediment transport dynamics \cite[e.g., the increased mobility of gravel-sized particles in the presence of a large sand fraction;][]{Wilcock_etal2001}, quantifying higher order moments of the grain size distribution is also of interest. It is possible that improvements upon the digital grain sizing results presented in this chapter could be obtained using a calibration-based approach \citep[e.g.,][]{Warrick_etal2009}. Other properties of the grains, namely particle shape, have also been demonstrated to play an important role in particle transport dynamics. The ability to digitally quantify particle shape at wave forcing timescales would be valuable, particularly in the context of coarse particle transport in swash flows.

