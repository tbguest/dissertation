\chapter{Conclusions}\label{Chapter:Conclusions}

This thesis presents observations of hydrodynamics, morphodynamics, grain size, and sediment dynamics at Advocate Beach, Nova Scotia -- a steep (1 in 10 slope), megatidal, mixed sand-gravel (MSG) beach. The beach is positioned near the head of the Bay of Fundy, and is generally fetch-limited, being forced by steep, locally generated waves with mean periods of 4-7 s, leading to an energetic shore break. The mean tide range is 10 m. The large tide range, steep slope, and broad, generally well-mixed grain size distribution make Advocate Beach advantageous as a study site for reasons both practical and scientific. The intertidal beach can be accessed during each low tide to deploy, retrieve, or maintain \textit{in situ} instrumentation. The selective transport properties of heterogeneous sediments lead to spatially and temporally varying sedimentation patterns often manifesting as complex three-dimensional morphologies (e.g., beach cusps) accompanied by grain size sorting. The large tide range leads to a reduction in the force applied per unit of beach surface area and to a heightened potential for observable morpho-sedimentary signals. However, because of the typically energetic shorebreak at Advocate Beach and at coarse-grained beaches in general, where impulsive wave forcing and large mobile grains impose a hazard to instrumentation, conventional \textit{in situ} sampling methods employed at sandy beaches are not suitable.

The observations were obtained during two field campaigns. The first campaign (Chapter \ref{Intro:Adv2015}) made use of beach-scale video monitoring and arrays of buried pressure sensors to observe beach cusp dynamics and the vertical pore pressure response of the bed under oscillatory and hydrostatic loading due to waves and tides. The second campaign (Chapter \ref{Intro:Adv2018}) employed GPS and image-based surveying to observe bed level and mean surficial grain size change on the scale of the intertidal zone, ultrasonic range sensors and digital imagery to obtain changes in bed level and mean grain size in the swash zone, and video-based Lagrangian tracking of natural cobble tracers in swash flows. The purpose of these studies was to gain insight into morpho-sedimentary, hydrodynamic, and hydraulic processes on mixed sand-gravel beaches on sub-tidal timescales. The datasets were used to address the following objectives: (1) characterise the depth dependence of oscillatory pore water pressures in intertidal sediments induced by surface gravity waves; (2) investigate the morpho- and hydrodynamical processes associated with mixed sand-gravel beach cusp evolution; and (3) investigate the coevolution of bed level and grain size in the swash and intertidal zones.

%[revise objectives]

\section{Key Findings}\label{section:KeyFindings}

% from intro:

%\begin{enumerate}
%	
%	\item Characterize the depth dependence of the phase lag and attenuation of oscillatory pore water pressures induced by surface gravity waves.
%	\begin{itemize}
%		\item Can the pore-pressure amplitude and phase be predicted using a poro-elastic bed response model?
%		\item What are the implications of the pore-pressure response for the efficacy of observing surface gravity wave amplitude and phase with buried pressure sensors?
%	\end{itemize}
%	
%	\item Investigate the morpho- and hydrodynamic processes associated with mixed sand-gravel beach cusp evolution.
%	\begin{itemize}
%		\item How do mixed sand-gravel cusps fit into the existing cusp literature?
%		\item What characterizes the timescale of cusp evolution at Advocate Beach?
%		\item What effect does large tidal range have on the timing of cusp events?
%		\item What is the role of sedimentology (e.g. grain-size distribution, sorting) in cusp morphological evolution?
%		%\item Can signatures of standing edge waves be observed during cusp formation at Advocate Beach?
%	\end{itemize}
%	
%	\item Investigate grain-size sorting processes and linkages to swash zone morphologic change.
%	\begin{itemize}
%		\item Can Lagrangian tracking of gravel and cobble size-fractions help us characterize sorting processes in the swash zone?
%		\item Are there correlations between properties of the grain-size distribution and ephemeral morphologic features on the beach surface?
%	\end{itemize}
%	
%\end{enumerate}


\begin{enumerate}	
	
	\item \textbf{Pore-trapped air plays a key role in the dynamics of pressure transmission through the sediment column at Advocate Beach.}
	
	During the Advocate 2015 field campaign, large and persistent phase lags and high degrees of attenuation of wave-induced pore pressure signals with depth in the upper 50 cm of the beach surface sediments were observed (Chapter \ref{Chapter:PorePressure}). The analytical poro-elastic bed response model from \citet{Yamamoto_etal1978} provided a good fit to the data. The saturation parameter, defined as the fractional volume of air in the interstitial fluid, was shown to be of first-order importance in determining the pressure response properties of the porous medium. Interstitial air contents of $O$(10\%) by pore volume were required to recreate the observed pore pressure attenuation and phase profiles. Variations in the hydraulic conductivity, resulting from disturbance of the sediment following instrument burial and subsequent reworking by storms, were also shown to influence the modelled pore pressure response. The persistence of the phase and attenuation properties suggests that pore-trapped air remains in the bed throughout both individual and spring-neap tidal cycles. 
		
	
	\item \textbf{Hydrostatic pressure alone does not serve to restore disturbed intertidal beach sediments to equilibrium, at least not on time scales of $O$(1 week).}
	
	The burial of the vertical pressure sensor array (Chapter \ref{Chapter:PorePressure}) coincided with a 4-day period of low energy wave forcing. The disturbance of the beach sediments associated with burying the pressure sensor array resulted in a prolonged factor-of-two difference in the oscillatory pore pressure phase and attenuation response. Though the array was subjected to 3-5 m of hydrostatic pressure at high tides throughout this time, the pore pressure response profiles did not return to their apparent `natural' state until the arrival of a storm event. The pore pressure response prior to the reworking of sediments by storm waves was characterised by reduced attenuation and phase shifting with depth in the sediment, suggesting that the disturbed sediments were more hydraulically conductive, and thus less effective at entrapping air.
	
	
	\item \textbf{Beach cusp dynamics and their location in the cross-shore are strongly influenced by the tide range via the shoreline position and its rate of change.}
	
	The formation of beach cusps observed during the Advocate 2015 field campaign was strongly influenced by the tides. Formation was favoured during high tide and early ebb, with no cusps being observed at the mid- or low-tide levels. During two instances of cusp field emergence that were examined in detail in Chapter \ref{Chapter:CuspDynamics}, the transition from relict to newly formed cusp fields occurred over \textit{ca}. 1 hour, with initial emergence and relict cusp decay timescales of 10-20 minutes, both visually and in an $e$-folding sense. The general absence of cusps at the mid-tide level on the beach face suggests a minimum formation timescale of $O$(10) minutes, or $O$(100) swash cycles, assuming nominal swash zone widths of 3-5 m at the times when the rate of shoreline translation was largest -- as much as 0.007 m s$^{-1}$ at mid-tide at Advocate Beach. The location and dynamics of cusp horns (i.e., erasure/re-emergence versus branching/merging/shifting) appeared to depend on the high water line and its location relative to any pre-existing cusp morphology. 
	
	
	\item \textbf{Timescales associated with the emergence of beach cusps on Advocate Beach are dependent on the beach surface grain size distribution.}
	
	Beach cusps at Advocate Beach are generally characterised by striking differences in grain size between the cusp horns (gravel-cobbles) and bays (sand). During one cusp event analysed in detail in Chapter \ref{Chapter:CuspDynamics}, the emergence of new cusps at high tide appeared to be inhibited by the lack of coarse material present on the beach surface. Cusp emergence occurred shortly after, during early ebb, where the beach surface consisted of a higher proportion of gravel and cobbles. Persistent correlations observed between changes in bed level and mean grain size (Chapter \ref{Chapter:MSDBeach}), which suggest a sedimentary feedback component to morphological change, support this finding.

	
	\item \textbf{A strong negative correlation exists between the mean surface grain size in the mid-intertidal zone and the forcing energy.} 
	
	During the Advocate 2018 field campaign, the mean surficial grain size on the intertidal beach exhibited fining in response to energetic waves, and coarsening in response to fairweather forcing (Chapter \ref{Chapter:MSDBeach}). This is the opposite of the expected grain size response on sand beaches. A similar response was previously observed at Advocate Beach by \citet[][]{Hay_etal2014}, as well as by other authors in the mixed sand gravel beach literature \citep{Nordstrom_Jackson1993, Pontee_etal2004, Curtiss_etal2009, Miller_etal2011}. %Based on the existing literature it is not clear whether 
	
	% though general comments to this effect, viz., the phenomenological response of mixed sand-gravel beach surface sediments to energetic waves
	
	
	\item \textbf{A positive correlation exists between changes bed level and mean grain size in the intertidal zone on tide-to-tide timescales.} 
	
	A persistent positive correlation was observed between changes in bed level and changes in the mean surface grain size from one low tide to the next during the Advocate 2018 field campaign (Chapter \ref{Chapter:MSDBeach}). The correlation was independent of the wave steepness or significant wave height. Though some correlation was apparent at the position of the characteristically coarse-grained high tide berm, the correlation was largest in the mid-intertidal zone -- the region seaward of the berm, typically characterised by a broader surficial grain size distribution.
	
	
	\item \textbf{Signals of bed level and mean grain size change with timescales of seconds to minutes are observable using acoustic bed level sensors and digital grain sizing in the intermittently exposed subaerial swash zone.}
	
	Coincident observations of bed level and mean grain size at timescales commensurate with the swash forcing (Chapter \ref{Chapter:MSDSwash}) revealed signals providing insights into the morpho-sedimentary dynamics. A positive qualitative correlation was observed over timescales on the order of hours. Finer-scale structure in the signal was observable over timescales of minutes, including signatures of bands of coarse-grained material that migrated shoreward with the leading edge of the swash prior to high tide berm formation.
	
	
	\item \textbf{The direction and magnitude of cobble transport in swash flows depended upon the cross-shore position of the cobble within the swash zone, as well as the composition of the underlying bed.}

	Lagrangian tracking of natural cobbles in swash flows was carried out using video imagery (Chapter \ref{Chapter:MSDSwash}). The net transport of cobbles was shoreward, with the smallest cumulative transport at stations where the substrate was predominantly coarse-grained, i.e., in the presence of the berm. When the substrate was predominantly fine-grained, i.e., at stations seaward of the coarse berm, the cobble tracers diverged from the mid-swash zone, being transported either to the shoreward edge of the swash zone, or seaward toward the step. The cumulative transport was greatest where the substrate was fine-grained, and the net transport was near zero. The longshore component of transport was small in comparison to the cross-shore component.
	
	
%	[refer to objectives and make sure im not laeving anything out. e.g. where do msg cusps fit in the overall cusp literature?]
	
\end{enumerate}


\section{Implications and Insights}\label{section:Insights}

%[including the pressure response of the intertidal beach as a poro-elastic medium; and interactions ]
%From the pore-pressure results of Chapter \ref{Chapter:PorePressure}, 

% and eventually bridge the gap between beach-scale responses and the small-scale underlying physics

Relative to pure sand or gravel beach settings, few field observations of mixed sand-gravel (MSG) beach dynamics exist in the literature, especially with regard to dynamics in the swash zone. A sound understanding of MSG beach phenomenology is required to inform meaningful parameterisations, and to improve our ability to predict the physical responses of MSG beaches to forcing. This thesis contributes to an observational basis of MSG beach dynamics from which a broader understanding of processes can be drawn. Some notable contributions include (1) observations of the vertical structure of the MSG bed at Advocate Beach as a poro-elastic medium, including the depth-dependent pressure response, grain size distribution, porosity, and hydraulic conductivity. Reported values of the hydraulic conductivity, in particular, of natural sand-gravel mixtures in the literature are few, despite the acknowledgement by \citet{Mason_Coates2001} of hydraulic conductivity as a leading order influence on the characteristics of MSG sediment transport; (2) observations of MSG beach cusps, about which few, if any, dynamical observations exist in the literature, but which display notable differences from cusps on sand beaches, namely, the important role played by the process of size segregation in cusp formation on MSG beaches; and (3) observations of morpho- and sedimentary dynamics at multiple scales in space and time, which to date are also largely unavailable in the MSG beach literature, particularly with regard to grain size dynamics. 

%Chief among these ... are the observations of sediment properties reported in Chapter \ref{Chapter:PorePressure}, including observations of the depth-dependent grain size distribution, porosity, and hydraulic conductivity of the natural mixed sand-gravel sediments at Advocate Beach. Reported values of the hydraulic conductivity, in particular, of sand-gravel mixtures in the literature are few, despite the acknowledgement by \cite{Mason_Coates2001} of hydraulic conductivity as a defining characteristic of MSG beaches and their dynamics.

In addition to the contributed observations, many of which may be of interest in their own right, the results of this thesis have provided insights into the hydro/morphodynamics and sediment dynamics in the intertidal zone at Advocate Beach. The inferred presence of a large and persistent trapped air component ($O$(10\%) by pore volume) in the upper 0.5 m of the sediment column has important implications for beach stability and potential mechanisms for sediment mobilisation: Rapid attenuation attributed to the presence of pore-trapped air limits the effects of cyclic pressure loading by wave action to a region near the bed surface. However, the presence of large phase lags in the oscillatory pore pressure signal with sediment depth may lead to instances of momentary sediment liquefaction, wherein vertical pressure gradients associated with the passage of wave troughs periodically oppose the local gravity vector, with magnitudes sufficient to overcome the static gravitational equilibrium of the bed. Whether, or to what extent, momentary sediment liquefaction is important as a sediment mobilisation mechanism is not known. 

In practice, due to the large and variable phase lags and high attenuation through the bed, attempts to obtain measurements of surface gravity wave kinematics using pressure sensors buried in intertidal sediments should include sediment-free passage between the sensors and the sediment-water interface (e.g., using rigid, perforated pipe). Phase coherent multi-sensor vertical arrays and poro-elastic bed response modelling are otherwise required. This may be true for a range of beach types, including pure sand \citep{Michallet_etal2009}, where sediments are intermittently aerated as a result of tidal action.

%suggesting the importance of morpho-sedimentary feedbacks in the evolution of beach surface features. This reinforces the importance of a morpho-sedimentary dynamics framework \citep{Buscombe_Masselink2006} in the study of coarse-grained beaches. bedform surrogacy

%One of the principle findings of this thesis is the persistent coevolution of morphology and grain size, suggesting the importance of sedimentary feedbacks on morphodynamic evolution in the swash zone...\\

%This thesis presents multiple avenues of evidence suggesting the importance of sedimentary feedbacks in the morphological evolution of Advocate Beach: (1)persistent correlation -- which in and of itself is not ... 

%An interesting implication of the morpho-sedimentary correlation analysis is the suggestion that feedbacks between fluid flow and beach surface texture influence the morphodynamical evolution of the swash zone, leading to spatially correlated morpho-sedimentary features which may be highly organised (e.g., cusps) or irregular (textural `patchiness' described in Chapter \ref{Chapter:MSDBeach}), but have observable signatures which may persist regardless of the forcing conditions. [break?] 

This thesis presents evidence of the morphodynamical importance of sedimentary feedbacks in the swash zone, leading to spatially correlated morpho-sedimentary features which may be highly organised (e.g., cusps) or irregular (textural `patchiness' described in Chapter \ref{Chapter:MSDBeach}), but which have observable signatures that persist regardless of the forcing conditions. The feedbacks manifest most strikingly through the persistent coevolution of morphology and grain size in space. However, multiple other findings support the role of feedbacks as a causative mechanism: (1) the apparent influence of the surface grain size distribution on the timing (i.e., cross-shore position) and/or timescale of beach cusp emergence; (2) the presence of relict cusp horns on the positioning of emergent cusps in the longshore; (3) the apparent lead time of the mean grain size maximum associated with the shoreward edge of the swash zone, relative to positive bed level change as a precursor to berm formation (see Chapter \ref{Chapter:MSDSwash}, Figs. \ref{fig:MSD_timeseries_19} and \ref{fig:MSD_timeseries_27}); and (4) the substrate-dependent transport characteristics of individual cobbles in the swash zone. %; and (5) ... .

%% what do I want to say here...
%- the spatially correlated features arise as a result of swash processes during ebb tide.
%- these features are necessarily the result of a short period of swash forcing -- $O$(10) mins based on the time- and space-scale arguments made in Chapter \ref{Chapter:CuspDynamics}. 
%- correlations are only possible (here) with subaerial observations
%- therefore it is not possible to observe signatures of feedbacks (with the methods described in this thesis) occurring (1) in the flood tide swash, or (2) seaward of the swash zone.
%- therefore it is not possible to report whether morphodynamically important sedimentary feedbacks occur anywhere but in the ebb tide swash. 
%- the fact that temporal correlations (from one tide to the next) are not significant indicates only that the sampling interval was not sufficiently short to resolve the processes.
%- The forcing associated with the passage of the shorebreak, surf zone, and flood tide swash zone leads to prolonged periods of sediment redistribution. 
%- correlation results give strong indication that sedimentary feedbacks are morphodynamically important in the swash zone.

%The necessity of subaerial observations to carry out the type of correlation analysis used in Chapter \ref{Chapter:MSDBeach} means that signatures of feedbacks occurring either in the flood tide swash or seaward of the swash zone are much more difficult to observe. 
%Though the observations presented in Chapter \ref{Chapter:MSDBeach} are specific to the ebb tide swash zone, 

The spatial correlations observed between changes in bed level and grain size are a result of processes in the swash zone during ebb tide. The ebb tide swash processes necessarily act over short timescales governed by the swash zone width and the rate of change of the mean shoreline position. The forcing associated with the passage of the shorebreak, surf zone, and flood tide swash zone leads to prolonged periods of sediment redistribution, resulting in spatial `smoothing' of relict morphological and sedimentary features in the intertidal zone, at least when observed with a sampling resolution of once per tide. The lack of significant temporal correlations between bed level and grain size (from one tide to the next) may only indicate that the sampling interval was not sufficiently short to resolve time-coherent processes that occur over timescales which are commensurate with the forcing ($O$(10) s), or with the evolution of the morpho-sedimentary features ($O$(100-1000) s). 

The emphasis in this thesis on correlating with subaerial observations means that it is much more difficult to observe signatures of feedbacks occurring either in the flood tide swash or seaward of the swash zone. However, the persistent spatial correlations between bed level and grain size that are associated with ebb tide swash processes give a strong indication that sedimentary feedbacks are morphodynamically important in the swash zone. Processes associated with the shorebreak and surf zone may still exhibit characteristics of morpho-sedimentary correlation, as in the cases of, for example, the beach step \citep{Austin_Buscombe2008}, and wave orbital-scale ripples \citep{Hay_etal2014}, though the mechanisms responsible for the formation of these features may not depend on a wide surficial grain size distribution. 

The role of sedimentary feedbacks on longer-term beach morphological responses (i.e., longer than a single tide) is unclear. However, swash-generated morpho-sedimentary features of the beach profile have been shown in some cases to influence the longer-term response, as demonstrated in Chapter \ref{Chapter:CuspDynamics}, under fairweather forcing, via the influence of relict cusps on the positioning of emergent cusp fields; as well as by \citet{Orford_Carter1984_cuspoverwash}, who reported that the locations of barrier overwash fans on a pure gravel beach during storm events were influenced by the locations of relict cusps on the upper beach. Given the inference from Advocate Beach that the surf and swash zones can be approximated as a closed sedimentary unit \citep[Chapter \ref{Chapter:MSDBeach}, based on anecdotal evidence from][]{Hay_etal2014}, it might be expected that coarse surficial sediments, in addition to one or more active or relict berms, would influence the response of the beach profile (e.g., the slope) to storm waves through the increased proportion of coarse particles. However, observations in support of this are not available from the current study.

%The morpho-sedimentary feedbacks leading to observable correlations between bed level and grain size appear to be specific to the swash zone. The forcing associated with the passage of the shorebreak and surf zone leads to prolonged periods of sediment redistribution. Meaningful correlations of .... are therefore not possible,  making observations of correlation between bed level and grain size through time less likely.   -- which may still exhibit characteristics of morpho-sedimentary correlation, as in the cases of, for example, the beach step \citep{Austin_Buscombe2008}, and wave orbital-scale ripples \citep{Hay_etal2014}. Swash-generated morpho-sedimentary features of the beach profile have been shown in some cases to influence the longer-term (i.e., longer than a single tide) response of the beach, as demonstrated in Chapter \ref{Chapter:CuspDynamics}, under fairweather forcing, via the influence of relict cusps on the positioning of emergent cusp fields; as well as by \citet{Orford_Carter1984_cuspoverwash}, who reported that the locations of barrier overwash fans on a pure gravel beach during storm events were influenced by the locations of relict cusps on the upper beach. Given the inference from Advocate Beach that the surf and swash zones can be approximated as a closed sedimentary unit \citep[Chapter \ref{Chapter:MSDBeach}, based on anecdotal evidence from][]{Hay_etal2014}, it might be expected that coarse surficial sediments, in addition to one or more active or relict berms, would influence the response of the beach profile (e.g., the slope) to storm waves through the increased proportion of coarse particles. However, observations in support of this are not available from the current study.

Some insight into the nature of the morpho-sedimentary feedbacks is also evident from the results of this thesis: namely, that pattern formation and coherent morpho-sedimentary change are heavily influenced by the coarse tail of the grain size distribution on the beach surface, or at least that the presence of a wide surficial grain size distribution, containing both fine and coarse material, is most likely to lead to observable correlations, and to the formation of cusps and cusp-like features. In the correlation analysis of Chapter \ref{Chapter:MSDBeach} (Table \ref{table:dz_dmgs}), an increase in the mean surficial grain size occurred during 86\% of the observed instances of positive bed level change (where bed level increased by greater than two centimetres) in the mid-intertidal zone. Bed fining corresponded to negative bed level change in only 56\% of cases. Similarly, in  Chapter \ref{Chapter:CuspDynamics}, the presence of coarse material at the beach surface accelerated (or, potentially, made possible) the emergence of beach cusps, as outlined above. %The lack of observations of other components of the grain size distribution (of comparable quality to the mean grain size), namely, a sorting parameter or other indication of the range of particle sizes present at a given time, represents a substantial shortcoming of the sedimentary datasets in this thesis.

The apparent dynamical importance of the coarse tail of the grain size distribution is in contrast to results from the sand beach literature, at least for the case of beach cusps, where variations in grain size associated with morphological pattern formation have been suggested not to be mechanistically important in the formation process \citep{VanGaalen_etal2011}. With regard to the coevolution of bed level and grain size on MSG beaches, it seems likely that conditions of fractional transport, wherein the coarsest component of the grain size distribution experiences a diameter-dependent likelihood of mobilisation, or no mobilisation at all, is important, as opposed to conditions of equal mobility, under which particles experience equivalent probabilities of mobilisation regardless of particle diameter. In other words, it is likely important that the threshold mobilisation shear stress for the coarsest grains is of the same order of magnitude as the bed shear stress under swash forcing.  % the tail wags the dog!

% It is also notable that the spatiotemporally averaged mean surface grain size for the 2018 field experiment (using data from Fig. \ref{fig:alltides_profiles}), corresponds to the upper XX percentile of the bulk distribution (see Figs. \ref{fig:bulk_grainsize} and \ref{fig:soilstats}) as observed during 2013 and 2015 experiments at Advocate Beach.

The presence of persistently observed correlations between bed level and grain size is supportive of the `intermediate case' of the morpho-sedimentary dynamics hypothesis, as described by \citet{Buscombe_Masselink2006}. In relation to the limiting cases, i.e., (1) that changes in grain size in dynamic sedimentary environments do not remain correlated through time or (2) that there does exist a temporal correlation or `persistence' in sedimentary data, \citeauthor{Buscombe_Masselink2006} describe the intermediate case as morphological change resulting in spatial variability in grain size displaying temporal persistence, but with variations which are stochastic, centred around a time-averaged grain size distribution. The intermediate case allows grain size characteristics to have a role in the morphodynamic response of the beach. 

MSG beaches are well-suited to morpho-sedimentary dynamical studies, their high degrees of grain size variability leading to relatively strong and observable sedimentary signals. Though morpho-sedimentary feedbacks undoubtedly influence the dynamics of pure gravel beaches as well, the narrower grain size distributions -- paired with the inherent variability of grain size estimates based on digital imaging methods -- may make it more difficult to observe sedimentary signals \citep[see][]{Austin_Buscombe2008, Masselink_etal2007}. Even on MSG beaches, constant redistribution of sediments by surf and swash action leads to the loss of time-coherent signals if the observation frequency is too low, i.e., much lower than the frequency of the forcing. Improvements in analysis strategies, particularly ones leading to the ability to more fully characterise broad and variable grain size distributions, may make observing meaningful morpho-sedimentary signals possible for a wider range of beach types in the future.

%Another important implication of this thesis is the demonstrated utility of recent improvements in remote sensing capability (i.e., over the last 10-20 years), including commercially available, low cost sensing options, to observe coastal morpho/hydrodynamical and sedimentary processes in coarse beach environments where conventional \textit{in situ} instrumentation has limited use. In particular, the use of acoustic range sensors for bed level monitoring combined with video or image-based grain sizing has potential to provide important insights into swash zone morpho-sedimentary dynamics and feature evolution.

%Unlike the self-organisation-based cusp model from \citet{Werner_Fink1993}, where topographic highs and lows both contribute to the emergence of a global pattern through constructive feedbacks, here, the dynamics of emergence appear governed by the presence and distribution of coarse particles (topographic highs) on the beach surface. This .. bedform surrogacy -- see above?

%[thinking about the process wrt 'sticky' coarse particles that are more difficult to 'smear', whereas the finer component moves around and responds. (in the lexicon of werner and kocurek, the tail does indeed wag the dog.)]


\section{Future Work}\label{section:FutureWork}

Despite substantial advances in observation techniques and the predictive skill and capacity of beach profile response models over the past several decades -- particularly when applied to pure sand or gravel beaches -- significant barriers to the predictability of physical beach processes remain. State of the art modelling efforts are not yet capable of reproducing mixed sand-gravel transport dynamics in the surf and swash zones. The confounding influences of inter-granular interactions, e.g., grain interlocking and sheltering by neighbouring grains leading to complex mobilisation characteristics, and the effects of spatial and temporal heterogeneity of grain size distributions on in/exfiltration and groundwater dynamics, in addition to near-bed turbulence from wave breaking and grain-scale roughness are difficult to parameterise, and represent substantial barriers to first principles-type modelling of mixed sand-gravel transport in swash flows. More field and laboratory data are required, for a broader range of beach types and forcing conditions, to guide our understanding of the fundamental phenomenological responses of mixed sand-gravel sediments at the shoreline. Based on the results of this thesis, there are several avenues for future research that would be of interest.

A logical and timely starting point would be the undertaking of a review of MSG beach dynamics, in order to supplement the informative MSG beach review papers by \citet{Kirk1980}, \citet{Mason_Coates2001}, and \citet{Pontee_etal2004}, as well as the gravel beach review paper by \citet{Buscombe_Masselink2006}, with the advances of the past 15-20 years, particularly regarding remote sensing methods.

With regard to pore pressure and the hydraulic and groundwater responses of mixed sand-gravel beach sediments, it seems likely that the $O$(10\%) trapped air component by pore volume inferred through the application of the \citet{Yamamoto_etal1978} poro-elastic bed response model should influence percolation processes and sediment stability at Advocate Beach. Though it is not possible to directly observe the bulk air content in the bed, it would be of interest to validate the $O$(10\%) air content estimate through the use of arrays of soil moisture sensors, which have been used elsewhere \citep[e.g.,][]{Heiss_etal2015} to infer volumes of pore-trapped air through gravimetric or volumetric water content. Multi-element coherent arrays of soil moisture and pressure sensors, buried at multiple locations both across-shore and vertically, could be used to better characterise spatial and temporal variability in the bulk compressibility of the sediment-fluid medium, and its influence on bed stability via mass mobilisation events. Multiple (vertical) observations of pressure and air content would enable the testing of a threshold criterion for momentary sediment liquefaction \citep[e.g.,][]{Qi_Gao2015}, helping to assess the relative importance of sediment mobilisation mechanisms under oscillatory or impulsive wave loading.

The finding that the mean beach surface grain size is closely correlated with the forcing, the wave height and steepness in particular (the mean grain size decreasing roughly 13 mm per metre of increased significant wave height at Advocate Beach, at least for mean grain sizes ranging from 15-25 mm), warrants testing with longer time series, and in a broader range of MSG settings. The analysis by \citet{Prodger_etal2016}, which made use of time histories of disequilibrium wave steepness, led to the prediction of mean surface grain sizes at a number of sand beach sites with considerable skill. It seems likely that a similar -- though perhaps fundamentally opposite -- relationship could be established for MSG beaches. 

This thesis provides compelling evidence that mixed sand-gravel morphodynamics and grain size dynamics are intrinsically linked through feedback processes, with implications for beach morphological evolution, at least under swash forcing. In considering future work, it would be of interest to make similar measurements with greater spatial coverage, particularly in the cross-shore, in order to better resolve the fine-scale grain size and bed level changes associated with incipient berm formation at the leading edge of the swash, as well as other processes related to the response of the beach profile. This study has demonstrated that low cost commercial range sensors are capable of identifying swash-timescale changes in bed level, at least in a coarse-grained setting. More precise instrumentation would reduce the need for added processing, and would be required where a greater proportion of the grain size distribution is less than \textit{ca}. 1 mm. More sensors in an across-shore configuration would also allow for the consideration of volume change in the swash region; e.g., is positive bed level change at the berm balanced by erosion from the mid-swash, or must material be sourced from the step region as well?

The lack of observations of other components of the grain size distribution (of comparable quality to the mean grain size), namely, a sorting parameter or other indication of the range of particle sizes present at a given time, represents a substantial shortcoming of the sedimentary datasets in this thesis. Given the acknowledged influence of the full grain size distribution on mixed sediment transport dynamics \citep[e.g., the increased mobility of gravel-sized particles in the presence of a large sand fraction;][]{Wilcock_etal2001}, quantifying higher order moments of the grain size distribution would be of interest. Other properties of the grain size distribution, namely particle shape, have also been demonstrated to play an important role in particle transport dynamics. The ability to digitally quantify particle shape at wave forcing timescales would be valuable, particularly in the context of coarse particle transport in swash flows. However, this problem remains to be solved.

Based on the findings in this thesis, future field studies attempting to observe signals of morphological and sedimentary coevolution should give close consideration to the site. In comparison to pure gravel beaches, the space- and time-varying surficial grain size distributions found on MSG beaches appear well-suited to the emergence of detectable signals. The presence of a large tide range is also beneficial, at least in terms of improved access to the intertidal zone, and likely also in terms of the improved signal-to-noise ratio resulting from decreased forcing duration per unit width of the beach. It would be of interest to conduct field studies similar to those outlined in Chapters \ref{Chapter:MSDBeach} and \ref{Chapter:MSDSwash} at mixed sand-gravel settings under a range of different tidal regimes. The presence of a large tide range is also useful for comprehending the phenomenological response of the beach: clear `before' and `after' bed states are separated by shorter timescales, especially near the high water line. Tide-to-tide changes in the morphology and sedimentary patterning of the bed in response to changes in forcing are therefore easier to observe -- often strikingly so.
