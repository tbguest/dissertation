\chapter{Background}\label{Chapter:Background}

In this section, a brief description of mixed sand-gravel (MSG) beaches is given, along with an outline of the recent MSG beach literature. Next, a review of the beach cusp literature is presented, with particular emphasis on mechanisms, timescales, and the role of grain-size segregation. Finally, leading order factors influencing sediment transport on MSG beaches are described.

\section{Mixed Sand-Gravel Beaches}\label{subsection:msg} 

%[be sure here to define characteristics of the beach profile]
%
%[more review of MSG lit]

Gravel beaches can be separated into three distinct morpho-sedimentary types, as outlined in the field-based classification scheme by \citet{Jennings_Shulmeister2002}: (1) pure gravel; (2) composite gravel, characterised by a steep gravel berm fronted by a sandy low-angle intertidal terrace; and (3) mixed sand-gravel, having moderate to steep slopes and well-mixed sediments both across-shore and vertically within the sediment column. Mixed sand-gravel beaches are common in the mid- to high-latitudes, where past glaciation or significant terrestrial weathering has provided an abundant source of coarse sediment. Typical grain sizes on MSG beaches vary from coarse sand to cobbles \citep{Mason_Coates2001}. 

Mixed sand-gravel beaches have features in common with both pure sand and gravel beaches: Like sand beaches, they are characterised by lower hydraulic conductivities -- i.e., the rate at which water can pass through the pore spaces of the medium, having units of distance per unit time -- than pure gravel. Similar to gravel beaches, MSG beaches are generally steep-sloped, and therefore are typically characterised by narrow surf zones leading to energetic shore breaks and morphodynamics dominated by processes in the swash zone -- the region of the beach intermittently covered and exposed by water following wave breaking. Despite any similarities, MSG beaches have been noted to be morphologically distinct from, and more complex than, both pure sand and gravel beaches \citep{Kirk1980}. In particular, the grain size dynamics of MSG beaches in response to variations in forcing are not well understood, and have in some cases been observed to be in fundamental opposition to the responses expected on sand beaches, as in the case of the fining of beach surficial sediments in response to energetic waves, observed by \citet{Nordstrom_Jackson1993}, \citet{Pontee_etal2004}, \citet{Curtiss_etal2009}, \citet{Miller_etal2011}, and \citet{Hay_etal2014}.

Much of our conceptual understanding of MSG beach morphology is derived from investigations of MSG beaches in New Zealand in the 1970s and 80s \citep[e.g.,][]{Mclean1970, Kirk1975, Kirk1980, Matthews1983}. Some illuminating field and laboratory experiments were also carried out by \citet{LonguetHiggins_Parkin1962}, treating the role of spatial and temporal hydraulic gradients (i.e., percolation) related to grain size segregation in the context of MSG beach cusp formation. 

%The wide distribution of grain sizes on MSG beaches makes them subject to (often pronounced) changes in bed surface composition in space and time (e.g., grain size-sorting, heterogeneity). The textural dependence of the bed surface on the grain size distribution leads to feedbacks on flow and transport \citep{Carter_Orford1993}. Consequently, the morphodynamic model employed for sandy environments describing the association between fluid flow and beach form does not fully describe morphological evolution on gravel and mixed beaches. In their gravel beach review paper, \citet{Buscombe_Masselink2006} urged a morpho-sedimentary dynamics framework for continued research and discussion around gravel beaches. Though \citeauthor{Buscombe_Masselink2006} discuss morpho-sedimentary dynamics primarily in context of pure gravel beaches, the framework is also well suited for discussion of MSG.

In the past 20 years, MSG beaches have seen an increase in research interest due in large part to the increasing use of sand-gravel mixtures in beach replenishment schemes \citep{Mason_Coates2001}. MSG beaches have also been acknowledged to be of interest in a purely scientific sense due to their complex dynamics, their generally well-mixed nature making them well-suited to investigations of sediment size segregation and pattern formation \citep{Hay_etal2014}. The majority of recent field studies have been concerned with MSG beach sediment and morphodynamics. Many of these studies have investigated sediment dynamics using Lagrangian tracers, through radio frequency identification (RFID) tagging \citep{Osborne2005, Allan_etal2006, Curtiss_etal2009, Dickson_etal2011, Miller_etal2011, Miller_Warrick2012, Bertoni_etal2012, Grottoli_etal2015, Grottoli_etal2019} or fluorescent painting of coarse natural sediments \citep{Ciavola_Castiglione2009, Stark_Hay2016}. These studies span a wide range of transport dynamics, e.g., with regard to magnitudes of longshore versus cross-shore transport and response to energetic or fairweather wave forcing conditions. Site-dependent considerations, including the characteristics of the forcing climate, coastline orientation, profile shape, and sediment type (e.g., size, shape, composition) are also important sources of variation.

% The range of transport dynamics observed in these studies is quite variable

% [particle shape \citet{Grottoli_etal2019,Grottoli_etal2015} \citet{Stark_etal2014} \citet{Stark_Hay2016}]
% \citet{Roberts_etal2013} (monthly) used remote methods to characterise the post-storm recovery of MSG beaches. 
% x and y also reported observations of sedimentology, though ... .
% Short-term morphological processes and hydrodynamcis were reported by \citet{Ivamy_Kench2006} and \citet{Hay_etal2014}, who, along with \citet{Curtiss_etal2009}, were also the only authors to report observations made using fixed instruments in the intertidal zone, where they would have been periodically exposed to the energetic shorebreak. 

Studies of MSG beach morphodynamics have been undertaken by \citet{Pontee_etal2004}, \citet{Horn_Walton2004}, \citet{Ivamy_Kench2006}, \citet{Miller_etal2011}, \citet{Miller_Warrick2012}, \citet{Bramato_etal2012}, \citet{Roberts_etal2013}, \citet{Hay_etal2014}, \citet{Almeida_etal2014}, and \citet{Grottoli_etal2017}. Of these, \citet{Pontee_etal2004}, \citet{Miller_etal2011}, \citet{Miller_Warrick2012}, \citet{Roberts_etal2013} and \citet{Hay_etal2014} also reported observations of grain size, though none of their sampling strategies were designed to evaluate collocated changes in bed level and grain size. With regard to the coevolution of morphology and the grain size distribution -- observations of which are required in order to evaluate the importance of sedimentary feedbacks in beach morphological evolution -- two studies have been undertaken \citep{Masselink_etal2007, Austin_Buscombe2008}, though both at pure gravel beach sites, with none treating MSG.

Only \citet{Horn_Walton2004} and \citet{Almeida_etal2014} reported observations of morphological processes in the swash zone at timescales commensurate with the wave forcing. \citeauthor{Horn_Walton2004} used a manual post-and-ruler method, sampling one-metre (cross-shore) spaced stations in the swash zone every 10-25 seconds over a three-hour period encompassing high tide. Their emphasis was on the magnitude of intra-swash timescale bed level oscillations, in comparison to magnitudes of bed level change over longer periods (i.e., hours). \citeauthor{Almeida_etal2014} used a terrestrial laser scanner which sampled the intertidal bed elevation at 2.5 Hz over the course of a tidal cycle characterised by energetic conditions. They noted a minimal morphological response to forcing in the intertidal zone in comparison to three other beaches surveyed (two pure gravel and one composite gravel), but did not emphasise morphological changes at wave timescales. 

Leading order factors influencing sediment transport on MSG beaches are outlined in section \ref{subsection:transport}. For a more complete overview of MSG beaches and their dynamics, refer to the review papers of \citet{Kirk1980} and \citet{Mason_Coates2001}. % [there are some other reveiew papers referenced in BM06].


%lab \citet{LopezSanRomanBlanco_etal2006}
%modelling studies \citet{Brown_etal2019} \citet{Bergillos_etal2017} \citet{PedrozoAcuna_etal2007}

\section{Beach Cusps}

%%%%%%%%%%%%%%%%%%%%%%%%%%%%%%%%%%%%%%%%%%%%%%%%%%%%%%%%%%%%%%%%%%%%%%%%%%%%%%%%%%%%%%%%%%%%%%%%%%%%%%%%%%%
%The striking regularity of beach cusp patterns has lead to decades of investigation into the prediction of cusp spacing as a function of the wave forcing conditions. The edge wave model (\cite{Guza_Inman1975}, \cite{Guza_Bowen1981}, others) assumes the presence of low mode subharmomic or synchronous standing edge waves, refractively trapped to the shoreline. In the case of mode 0 edge waves, cusp spacing is predicted by
%
%\begin{equation}\label{eq:ew}
%\lambda_c = m\frac{g}{\pi}T_i^2\sin\beta,
%\end{equation}
%
%in which $m$ is 1 or 0.5 for subharmomic or synchronous edge waves, respectively, $g$ is gravitational acceleration, $T_i$ is the incident wave period, and $\beta$ is the beach slope. This theory has largely given way to a self-organization based theory (\cite{Werner_Fink1993}; \cite{Coco_etal2000}; \cite{Sunamura2004}; \cite{Dodd_etal2008}), in which rhythmic cusp topography arises as a result of free behaviour characterized by asymmetric feedbacks between sediment and fluid flow. In this formulation, cusp spacing is given by
%
%\begin{equation}\label{eq:so}
%\lambda_c = fS,
%\end{equation}
%
%in which $S$ is the horizontal cross-shore swash excursion and $f$ is a nondimensional constant, empirically determined to be approximately 1.6.
%%%%%%%%%%%%%%%%%%%%%%%%%%%%%%%%%%%%%%%%%%%%%%%%%

% \cite{Sunamura_Aoki2000} observed gravel cusp formation apparently initiated by a large boulder, with the boulder providing a topographic high which diverted the shoreward flow, eventually leading to the formation of a cusp horn. This is consistent with the process suggested above.  

Beach cusps are spatially periodic sedimentary features common on beach foreshores. They consist of seaward-pointing sequences of ridges (horns), separated by topographic depressions (embayments), with quasi-regular spacings varying from $O$(10$^{-1}$) to $O$(10$^{2}$) m \citep{Coco_etal1999}. Cusps are generally acknowledged to be swash-generated features, though recent studies have drawn a distinction between beach cusps of short wavelength ($<$ 20 m), and the longer-wavelength `large beach cusps' which may be tied to surf- as well as swash-zone morphodynamic processes due to their larger cross-shore extent \citep{Garnier_etal2010}. The generation of cusps by swash processes links them closely to shoreline position, cusps generally being formed near the high water line. Cusps have been observed to form under a wide range of conditions and beach types, though they are acknowledged to form most readily during low energy, shore-normal incident wave forcing on reflective, medium- to coarse-grained beaches \citep{Holland1998}. 

The modern (\textit{ca}. 1970s to present) cusp literature has been largely focused on understanding the mechanism(s) responsible for cusp formation and establishing predictive relationships between cusp spacing and hydrodynamics. Efforts to establish relationships between spacing and wave forcing were initially based on the concept of a hydrodynamic template, provided by standing low mode subharmonic or synchronous edge waves -- alongshore-propagating waves trapped to the shoreline through a combination of reflection and refraction -- whose spatial structure would become imprinted on the underlying sediment \citep{Guza_Inman1975, Guza_Bowen1981, Holman_Bowen1982}. Later models are based on free, self-organising behaviour characterised by asymmetric feedbacks between hydrodynamics and morphology \citep{Werner_Fink1993, Coco_etal2000, Coco_etal2004, Sunamura2004, Dodd_etal2008}. Both mechanisms have been shown through laboratory and modelling studies to be capable of producing cusp-like patterns, and predict cusps of comparable wavelength. Criticisms of the edge wave model stem in part from a lack of observed edge wave presence during episodes of cusp formation \citep[e.g.,][]{Holland_Holman1996, Masselink1997}. \citet{Ciriano_etal2005} suggested that both mechanisms may play a role at different stages of cusp evolution, for example, spatially periodic perturbations being provided by standing edge waves, after which self-organising feedbacks become dominant. Many have noted that the difficulty in identifying the mechanism responsible for cusp formation arises as a result of the difficulty in verifying the presence of edge waves, which require multidimensional arrays of current meters or pressure sensors to observe \citep[e.g.,][]{Coco_etal1999}. 

Though the self-organisation model, which predicts cusp spacing as a linear function of swash zone width, has been used to successfully predict cusp spacings in a range of settings, \citet{Sunamura2004} noted that the linear model is not predictive; since both the cusp length scale and the swash zone width are dependent variables, the formula represents no causality. Based on this, \citeauthor{Sunamura2004} suggested a model to predict cusp spacing that depends upon wave height, period, and grain size, which in turn influence the beach slope and the swash zone width. The model predicts cusp spacing with accuracy similar to that of the edge wave and swash zone width models, but is the only length scale-predicting model that incorporates grain size.  

The focus on cusp initiation mechanisms and predictors of spacing has given way in the last decade to renewed emphasis on cusp morphology and characterisation of their dynamics \citep[][]{Almar_etal2008, VanGaalen_etal2011, Vousdoukas2012, Poate_etal2014}, based largely on the acknowledgement that predicting cusp spacing alone is an inadequate predictor of process \citep[e.g.,][]{Coco_etal1999, Almar_etal2008}. However, uncertainties remain about the nature of beach cusp morphological evolution. These include: varied and, in some cases, inconsistent conclusions regarding factors influencing cusp evolution timescales; the importance and/or role of grain size sorting; and a lack of consensus on the roles of accretion and erosion. 

There has been little emphasis on cusp formation timescale. This may be partly due to the difficulty inherent in establishing meaningful formation time metrics. The data indicate that beach cusp lengthscale and formation time are related. The observations by \citet{Komar1973} of beach cusps on a sandy lake shore often provide a lower limiting case in the literature for both cusp spacing and formation time -- as short as 11 cm and 10 mins, respectively, though cusp formation is generally reported to occur over tidal timescales (hours to days), with cusp spacings on the order of tens of metres. 

The relationship between timescale and beach composition is largely uninvestigated. Numerical modelling by both \citet{Coco_etal2000} and \citet{Dodd_etal2008} indicated a relationship between a dimensional transport constant in their sediment flux parameterisations and the number of swash cycles required for cusp formation. Fewer cycles were required when the transport constant was increased. As noted by \citet{Dodd_etal2008}, it is possible to approximate this constant as a function of median grain diameter using the \citet{vanRijn1984} sediment transport equation. The implication is that cusps would form more rapidly on beaches with a larger mean grain size. \citet{Dodd_etal2008} also concluded from their model results that beach permeability favours cusp development by enhancing the feedback mechanism (i.e., infiltration through the horns decreases the backwash volume, therefore decreasing offshore transport). For the case of mixed sand and gravel beaches, the gradation of fine and coarse material between bays and horns would be expected to further enhance feedbacks by increasing the asymmetry between erosion/depositional processes in bays and horns. Indeed, \citet{LonguetHiggins_Parkin1962} concluded from field observations that cusps form most easily given a vertical stratification of material: coarse sediments sitting atop a well-mixed, ``impermeable'' layer are more readily mobilised due to the reduced fluid infiltration. The heaping of sediments atop the horns makes them more permeable than the thinner layer in the bays, and therefore less subject to erosion. 

Despite their long history of study, previous studies of beach cusps have generally been limited to beach environments comprised of a single sediment type (i.e., sand or gravel), with little emphasis on beaches composed of mixed sand and gravel. \citet{Nolan_etal1999} presented the only study of which the author is aware from the last 20 years treating mixed sand-gravel cusps specifically. As their focus was on cusp morphometry, the dynamics of the formation process were not investigated. 

The role of size sorting in the process of cusp formation was investigated by \citet{VanGaalen_etal2011} for the case of a sandy, microtidal beach on the Atlantic coast of Florida. Using regular morphological observations from a terrestrial laser scanner and discrete surficial sediment sampling, they found no significant correlation between grain size and position between horn and embayment. This led them to suggest that horn coarsening and embayment fining occurs after cusp development, making them inherently different from other sorted bed forms whose formation depends on differential transport characteristics for different grain sizes. However, the sediment size distribution at their site was narrow, with typical grain sizes ranging from 0.17 to 0.35 mm. 

Results regarding the relative importance of accretion and erosion are varied. Instances of cusps formation have been reported for conditions dominated by sediment accretion, erosion, or a combination of both \citep[e.g., ][]{Antia1987, Masselink1997, Almar_etal2008}. In addition to their conclusions regarding size-sorting, \citet{VanGaalen_etal2011} demonstrated that the accretive/erosive properties of cusps observed on a sandy, microtidal beach were dependent upon the position of the cusps relative to larger-scale coastal morphology. The reader is referred to the \citeauthor{VanGaalen_etal2011} paper for a more complete review of past results treating cusp horn/bay sedimentation.


\section{Factors Affecting Mixed Sediment Transport} \label{subsection:transport} 

Though there are dynamic similarities between MSG and gravel beaches -- e.g., narrow surf zones, and emphasis on swash zone transport -- their responses to forcing may be markedly different. In particular, MSG beaches have percolation properties influenced by their high sand content, which may greatly reduce the hydraulic conductivity of the beach sediments \citep{Carter_Orford1984}. Hydraulic conductivity, sometimes referred to as the coefficient of permeability, is a measure of the ease with which a fluid travels through a porous medium, and has units of length over time. \citet{Mason_Coates2001} identify hydraulic conductivity as a leading order factor influencing mixed beach transport, along with infiltration and groundwater, wave reflection, and the threshold of sediment motion. Hydraulic conductivity acts as a primary control on beach profile via uprush/backwash asymmetry introduced by in/exfiltration effects (bed shear stresses are reduced during uprush through infiltration and/or frictional dissipation, leading to profile steepening), and influences the relative importance of groundwater dynamics. Groundwater dynamics have been widely demonstrated to play in important role in swash zone sediment transport, most notably via the influence on swash uprush-backwash asymmetry \citep[e.g.][]{LonguetHiggins_Parkin1962, Packwood1983}, but also by altering the boundary layer thickness and the vertical force associated with fluid flow into or out of the bed \citep[see: seepage forces,][p. 101]{Nielsen1992}. The dynamics of substrate and groundwater response to wave loading also have implications for bed stability and mobilisation via, for example, impulsive pressure forces associated with wave breaking \citep{PedrozoAcuna_etal2008} or periodic, momentary liquefaction events \citep{Michallet_etal2009}.

Initiation of motion and transport of mixed sediments has been studied extensively in the context of fluvial transport \citep[e.g.,][]{Wilcock_McArdell1997, Wilcock_etal2001, Wilcock_Crowe2003, Buscombe_Conley2012}. However, most studies involve steady unidirectional flow over a mildly sloping bed. MSG transport in the swash zone is complicated by percolation effects, fluid acceleration, and turbulence (e.g., following bore collapse). The limited availability of field observations also poses a challenge, especially in shallow aerated flows, where accurate non-intrusive observations of sediment transport remain beyond the state-of-the-art \citep{PedrozoAcuna_etal2007}.

%[some of this may be well suited for motivation/intro]:
Incipient sediment motion is typically described in terms of a threshold fluid velocity above which particles are mobilised \citep[e.g., the Shields parameter,][]{Nielsen1992}. The typical dependence of threshold parameters on bulk properties of the sediment (e.g., median grain diameter, $D_{50}$) is problematic for cases of wider grain-size distributions or sediment bimodality. For mixed sediments, interactions between size-fractions complicate the interpretation of mobilisation thresholds: larger particles are heavier, thus more difficult to move, but also protrude into the flow, and are therefore subject to larger lift and drag forces. Smaller grains are more readily entrained, but may be sheltered by the larger grains. Similarly, the ability of the flow to roll a grain depends upon its size relative to the substrate, the angle of pivot required for mobilisation being larger when a grain is nestled among grains of similar or larger size. These properties lead to mobilisation thresholds for mixed sediments which are different from estimates made assuming a bed of uniform composition \citep{Wilcock_McArdell1993}. The use of `hiding' functions, empirically-derived correction factors that adjust thresholds based on the sediment distribution, circumvent this problem to some extent, but do not account for changes in transport properties resulting from size sorting (e.g., armouring, sorted bedforms, cusps) or spatial and temporal sediment heterogeneity. 

The hiding-sheltering and rollability effects described above are intrinsic to the processes of armouring and overpassing, which have been used to describe size-sorting phenomena \citep{Moss1962}. In the process of armouring, aggregations of large grains create a substrate that readily `accepts' grains of similar size, leading to bed surface coarsening and preventing further transport of coarse grains locally, while finer grains infiltrate through pore spaces (kinetic sieving). This is in contrast to overpassing, during which the inertia and rollability of larger grains cause them to be `rejected' by a finer substrate. Armouring has been investigated in the fluvial literature for cases of unidirectional flow \citep[e.g.,][]{Gomez1983}, as well as in the swash zone literature \citep[e.g.,][]{Isla1993}. 

On macrotidal beaches, morphodynamics are influenced by the tidal state. The position of the shoreline and swash zone translate in the across-shore in phase with changing water level, varying the relative influence of in/exfiltration, reflectivity, and surficial grain-size distribution on sediment transport. Interactions between tide and groundwater have been demonstrated to be capable of influencing beach profiles \citep[e.g.,][]{Turner1995}, and in the case of a nonplanar beach face, reflection coefficients may vary with tidal stage \citep{Davidson_etal1994}. In the case of changing mean grain diameter in the cross-shore, rates of frictional dissipation may be affected. It is common for gravel and mixed beaches to exhibit coarsening of mean grain size shoreward \citep{Mason_Coates2001}. 

The effects of spatial and temporal variation in hydraulic conductivity, differential and distribution-dependent mobilisation thresholds, and grain size sorting on MSG sediment transport are difficult to capture in a predictive framework, and highlight the importance of an empirically-based, phenomenological understanding of processes. 




%Most transport models parameterize grain size as its mean value ($D_{50}$); this is problematic for cases of larger grain-size distribution or sediment bimodality. For mixed sediments, interactions between size-fractions complicate the interpretation of mobilization thresholds: larger particles are heavier, thus more difficult to move, but protrude into the flow, increasing lift and drag forces. Smaller grains are more readily entrained, but may be sheltered by the larger grains. Similarly, the ability of the flow to roll a grain depends upon its size relative to the substrate, the angle of pivot required for mobilization being larger when a grain is nestled among grains of similar or larger size. These properties lead to an underestimation of coarse sediment mobilization by a threshold-type parameter \cite{xx}. The use of `hiding' functions -- empirically-derived correction factors that adjust thresholds based on the distribution [rethink this] -- circumvent this problem to some extent, but do not account for changes in transport properties resulting from size sorting (e.g., armouring, sorted bedforms, cusps) or spatial and temporal sediment anisotropy. Existing models have had some success in reproducing observed changes in beach profile under wave forcing \cite{pedrozo acuna, others?}. However, these have treated only 2-d response under storm-representative conditions, and made use of substantial simplifications to the dynamics. 

%Useful insights can be obtained from the fluvial sediment transport literature. In a flume experiment treating varied mixtures of sand and gravel in unidirectional flow, \cite{Wilcock_etal2001} observed orders of magnitude increases in gravel transport rates when sand was introduced, despite a decrease in the overall proportion of gravel. They observed maximum gravel transport rates when sand content was between 15 and 27\%, and suggested that these ratios roughly corresponded to the transition from a framework- to a matrix-supported bed. 

%Interactions between grain size-sorting and sediment transport result from changes in grain size distributions at the bed surface. Varied transport regimes for differing size fractions can lead to bed armouring, preventing further transport locally.

%Differential hydraulic conductivities, thresholds of motion, and sorting properties pose a substantial challenge for modelling MSG transport processes. Most transport models parameterize grain size as its mean value ($D_{50}$), and it remains unclear on a phenomenological level how to best deal with interactions between grain size populations (e.g. fractionation)

%Imporatnt factors to outline:
%- Hydraulic conductivity *
%- infiltration and groundwater * [these should help motivate paper 1, along with obs challenges from intro]
%- threshold of motion
%- tidal range 
%- armouring

