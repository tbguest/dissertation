\begin{acknowledgements}
	
	
% On stochastic and deterministic complexity:
%
% In the process of writing my PhD dissertation, I wound up with a small collection of ruminations on / my perspectives of complexity. Most of these were not fit to be included in the thesis document, lest I distract from or obscure the boundaries of my (stated) crystalline objectives.
%
% The short version: they are two sides of the same coin. They are differences in perspective, which allow us two fundamentally different ways to comprehend complexity or (apparent) randomness.
%
%The world behaves as it does... as far as we can tell, according to the (human constructed) laws of classical mechanics -- at least for the things we can see. The physics of the very small -- the domain of quantum mechanics -- is a different kettle of fish that I will leave for someone more qualified than me to discuss.
% 
% from thesis:
%\textit{deterministic} \citep[e.g.,][]{Lorenz1963}, wherein nonlinear interactions between system elements are represented by relatively simple systems of deterministic equations, but with sensitive dependence on initial conditions leading to complex outcomes, or \textit{stochastic}, wherein complexity arises through the cumulative impacts of numerous process-response mechanisms, or through multiple controls on process-response relationships operating over ranges of spatial and temporal scales \citep{Phillips1992}.
%
% Consider the movement of celestial bodies -- the muse of Newton and ... . Two bodies in space ... . However, the introduction of a third body changes the problem considerably.
%
% Amazingly, the theoretical behaviour of fluids is quite well understood. Using the calculus of infinitesimals to build systems of equations that relate the position of a `parcel' of fluid to its rate of change in time and space, the properties of fluid motion can be (more or less) completely described. This makes such things possible as weather forecasting, ocean modelling, and any number of other human achievements. The science of fluid mechanics can be understood purely in the language of classical mechanics.  
%
% So why then, if both fluid and ballistic motion are individually well-understood, is there any problem at all in understanding the movements of sand grains at the beach?
%
% [nonlinearity, turulence, parameterisation]
%
% as we learned from Kurt Godel: sometimes even the most diligently constructed foundations -- something as fundamental and irrefutable as the axioms of logic underpinning modern mathematics -- fail to lead us everywhere with certainty. Some truths cannot be proven, and beginning with the fundamental does not always ensure understanding of a system's behaviour. For example: though there is no doubt much to be learned from examining neurons, how far will their observation get us on the road to comprehending thought, or ego (the concept of `me')? The camparison of neuron to thought is a long one from sand grains to coastlines (where we can largely ignore the dreaded -- to a physicist -- influence of `biology'), the same questions and challenges apply: Can starting from the bottom up or the top down lead us to the same understanding? Or will we at some point, in either direction, be barred from reaching the terminus -- whether by the never-quite-adequate thresholds of our certainty, or by our lack of observational capacity?
%
% Doug Hofsdadter
% Kurt Godel
% E N Lorenz
% Dan Buscombe
% James Gleick

	
I am very grateful to have been able to spend these past years considering interesting science questions, and to have been exposed to such wonderful and powerful techniques and tools for approaching them. I am a more enthusiastic and effective problem solver than when I began, and that alone makes the experience worthwhile.

There are many people who helped make my run at academia possible and enjoyable. First and foremost, I acknowledge my graduate supervisor, Alex Hay, who took me in as a wayward math major, put a shovel in my hands, and sent me off to dig the first of many holes in Advocate Beach. I have great admiration for Alex's commitment to quality science, and to exactness in meaning through language. Alex, you have spared the world another `woolly' writer (to whatever extent I have absorbed your lessons, at least).

I acknowledge my advisory committee: David Barclay, Tony Bowen, and Mike Dowd. I hold the three of them in very high regard, and I am thankful for their productive and encouraging guidance. Insightful comments from Tony Bowen led to improvements in the thesis, as well as to shifts in my perspectives of field oceanography, models, and processes at the shoreline.% Conversations with Tony often left me feeling that I was on the right track with regard to the direction of the thesis, despite my occasional misgivings.

Thanks to Richard Cheel for helping me cut my teeth both at the Matlab prompt and at the soldering station. His mentorship -- particularly in the early days of my graduate studies -- was invaluable, as was his assistance in carrying out the field work described in this thesis. The benefits of Richard's presence when one needs to retrieve something heavy from the seafloor cannot be overstated. Thanks, also, to the rest of the Hay Lab, past and present, for support, discussions, feedback, and friendship. Jenna Hare in particular has my thanks for always being willing to chat, and for her contributions to the camaraderie shared by the students of our department. I am sure that I am not alone in thinking this.
	
Thanks to the oceanography skate crew: Kevin, Matt, Ian, and (honorary member) Riley for many nourishing lunchtime sessions. Those sunny days at the bowl stand out as a true highlight of my graduate student experience.

Of course, I am grateful to my family and friends for their support and inspiration, which came in many forms: from my grandmother, Patricia, who brightened my Tuesdays with our phone calls, and who made sure I was eating enough through frequent lunches and snack deliveries; from my parents, who have been genuinely interested since the beginning, and who helped in absolutely any way they could fathom; from my brother, by being a model of commitment to the things that one loves; and from the rest of the gang for shared meals, post-work ales, river runs, bike rides, and other shenanigans of all kinds. 

My partner Laura has been a pillar of support in the daily grind of the past five years. She has witnessed the ebb and flow of my enthusiasm, and listened with patience and understanding as I summarised each snag, conundrum, solution, and bit of excitement that the days presented. It has been truly helpful. Thank you, Laura.

Finally, thanks to E. N. Lorenz for giving language and direction to a field of mathematics that inspired me long ago to pursue study in the mathematical and physical sciences, and whose work I have been led back to at many intervals in the course of my education. Perhaps I would be somewhere else were it not for the flick of his pen sixty years ago.

\end{acknowledgements}
 
