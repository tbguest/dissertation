\begin{abstract}

%The Faculty of Graduate Studies will reject MSc theses with abstracts exceeding
%150 words, and PhD theses with abstracts that either exceed 350 words or span
%more than a single page.

%Sediment transport processes on mixed sand-gravel beaches are not well studied in comparison to those on sandy beaches. Hydraulic effects, along with nontrivial interactions between morphology, sedimentology, and flow, pose a challenge for the modelling of mixed sand-gravel transport. Often energetic shorebreaks capable of entraining gravel- and cobble-sized grains limit the utility of \textit{in situ} instrumentation, contributing to a scarcity of observational data. In this thesis, morpho-sedimentary, hydrodynamic, and hydraulic surf- and swash-zone processes are investigated through a series of field studies at Advocate Beach, Nova Scotia. Four principal content chapters are presented. First, the vertical structure of surface gravity wave-induced pore pressure is investigated using a coherent vertical array of pressure sensors buried in the intertidal zone. Observations of the change of the phase lag and amplitude attenuation of oscillatory pore pressure signals with sediment depth are compared to a poroelastic bed response model in order to evaluate the efficacy of using buried pressure sensors to observe surface gravity wave amplitude and phase. Second, video observations are used to characterise beach cusp morphodynamics with high temporal resolution. Timescales of cusp evolution are emphasised, along with the role of the surface grain-size distribution and changing water level due to tides. Third, correlations between bed level and the surficial mean grain size are investigated using GPS- and photograph-based survey data at the scale of the intertidal beach. Fourth, and finally, the coevolution of bed level and mean grain size is investigated in the swash zone using an array of collocated acoustic range sensors and cameras. Lagrangian tracking of painted cobbles in swash flows is utilised to characterise processes leading to grain size segregation. The results are discussed in the context of a morpho-sedimentary dynamics framework, which emphasises the intrinsic interrelationships between beach morphology, flow, and the surficial grain size distribution.

Sediment dynamics on mixed sand-gravel (MSG) beaches have received much less attention in the literature than on sandy beaches. The steep slopes characteristic of MSG beaches result in an energetic shorebreak, accompanied by ballistic transport of gravel- and cobble-sized grains. The associated risks of damage to \textit{in situ} instrumentation have contributed to the relative scarcity of observational data. A central goal of this thesis is to contribute new knowledge and understanding of morphodynamic responses to wave forcing on MSG beaches through the use of innovative, inexpensive sensing systems not exposed to the rigours of the shorebreak. The studies were carried out at Advocate Beach, Nova Scotia, a 1:10 slope megatidal MSG beach at the head of the Bay of Fundy. The principal results are presented in four chapters. First, the vertical structure of surface gravity wave-induced pore pressure in the intertidal zone is investigated using a coherent array of buried pressure sensors.  A key finding is that the phase of the pore pressure lags the pressure at the sediment surface. This phase lag is shown to be due to the presence of bubbles within the sediment column, which has implications for using buried pressure sensors for surface gravity wave measurement in the intertidal zone on MSG beaches. Second, video observations are used to characterise beach cusp morphodynamics at high temporal resolution. The timescale of cusp evolution is shown to be $O$(10) minutes. Importantly, the cusps exhibit pronounced bay/horn size segregation, indicating strong feedback between the hydrodynamics and cusp formation on MSG beaches. In the third main chapter, correlations between bed level and the surficial mean grain size are investigated using GPS and photographic surveys of the intertidal beach. Finally, the coevolution of bed level and grain size is investigated in the swash zone using an array of collocated acoustic range sensors and cameras, and Lagrangian tracking of painted cobbles is utilised to study grain size segregation at the swash scale. The results are discussed in the context of a morpho-sedimentary dynamics framework, emphasising the intrinsic interrelationships between morphology, flow, and the broad surficial grain size distribution.

\end{abstract}
